\documentclass[12pt]{article}
\usepackage[finnish]{babel}
\usepackage[T1]{fontenc}
\usepackage[utf8]{inputenc}
\usepackage{amssymb}
\usepackage{amsmath}
\usepackage{hyperref}
\newcommand{\pat}{\partial}
\newcommand{\be}{\begin{equation}}
\newcommand{\ee}{\end{equation}}
\newcommand{\bea}{\begin{eqnarray}}
\newcommand{\eea}{\end{eqnarray}}

% roman letters
\newcommand{\ci}{{\rm i}}
\newcommand{\rmd}{{\rm d}}
\newcommand{\rme}{{\rm e}}



\newcommand{\abf}{{\bf a}}
\newcommand{\Zcal}{{\cal Z}_{12}}
\newcommand{\zcal}{z_{12}}
\newcommand{\Acal}{{\cal A}}
\newcommand{\Fcal}{{\cal F}}
\newcommand{\Ucal}{{\cal U}}
\newcommand{\Vcal}{{\cal V}}
\newcommand{\Ocal}{{\cal O}}
\newcommand{\Rcal}{{\cal R}}
\newcommand{\Scal}{{\cal S}}
\newcommand{\Lcal}{{\cal L}}
\newcommand{\Hcal}{{\cal H}}
\newcommand{\hsf}{{\sf h}}
\newcommand{\half}{\frac{1}{2}}
\newcommand{\Xbar}{\bar{X}}
\newcommand{\xibar}{\bar{\xi }}
\newcommand{\barh}{\bar{h}}
\newcommand{\Ubar}{\bar{\cal U}}
\newcommand{\Vbar}{\bar{\cal V}}
\newcommand{\Fbar}{\bar{F}}
\newcommand{\zbar}{\bar{z}}
\newcommand{\wbar}{\bar{w}}
\newcommand{\zbarhat}{\hat{\bar{z}}}
\newcommand{\wbarhat}{\hat{\bar{w}}}
\newcommand{\wbartilde}{\tilde{\bar{w}}}
\newcommand{\barone}{\bar{1}}
\newcommand{\bartwo}{\bar{2}}
\newcommand{\nbyn}{N \times N}
\newcommand{\repres}{\leftrightarrow}
\newcommand{\Tr}{{\rm Tr}}
\newcommand{\tr}{{\rm tr}}
\newcommand{\ninfty}{N \rightarrow \infty}
\newcommand{\unitk}{{\bf 1}_k}
\newcommand{\unitm}{{\bf 1}}
\newcommand{\zerom}{{\bf 0}}
\newcommand{\unittwo}{{\bf 1}_2}
\newcommand{\holo}{{\cal U}}
\newcommand{\bra}{\langle}
\newcommand{\ket}{\rangle}
\newcommand{\muhat}{\hat{\mu}}
\newcommand{\nuhat}{\hat{\nu}}
\newcommand{\rhat}{\hat{r}}
\newcommand{\phat}{\hat{\phi}}
\newcommand{\that}{\hat{t}}
\newcommand{\shat}{\hat{s}}
\newcommand{\zhat}{\hat{z}}
\newcommand{\what}{\hat{w}}
\newcommand{\sgamma}{\sqrt{\gamma}}
\newcommand{\bfE}{{\bf E}}
\newcommand{\bfB}{{\bf B}}
\newcommand{\bfM}{{\bf M}}
\newcommand{\cl} {\cal l}
\newcommand{\ctilde}{\tilde{\chi}}
\newcommand{\ttilde}{\tilde{t}}
\newcommand{\ptilde}{\tilde{\phi}}
\newcommand{\utilde}{\tilde{u}}
\newcommand{\vtilde}{\tilde{v}}
\newcommand{\wtilde}{\tilde{w}}
\newcommand{\ztilde}{\tilde{z}}


\hoffset 0.5cm
\voffset -0.4cm
\evensidemargin -0.2in
\oddsidemargin -0.2in
\topmargin -0.2in
\textwidth 6.3in
\textheight 8.4in

\begin{document}

\normalsize

\baselineskip 14pt

\begin{center}
{\Large {\bf FYMM/MMP IIIb 2020 \ \ \  Solutions to Problem Set 4}}
Jake Muff
\end{center}

\bigskip


\bigskip

\begin{enumerate}

%\item Consider the compact manifold $S^1$, and recall that it can be given an atlas with two charts $(U_i,\varphi_i)$, $i=1,2$, such that for $i=1,2$ we set
% \[
%  U_i =\{\rme^{\ci \theta} | \theta\in I_i\}\,,\quad  
%  \varphi^{-1}_i: I_i \to U_i\,, \quad \varphi^{-1}_i(\theta)=\rme^{\ci \theta}\,,
% \]
% using the parameter domains $I_1=(-\pi,\pi)$ and $I_2=(0,2\pi)$.
% Define for every $p=\rme^{\ci \theta}\in U_1$ the vectors
% \[
%  X_{p} = \sin^2 (\theta)\, \partial_\theta\,,\qquad
%  Y_{p} = \theta\, \partial_\theta\,,
% \]
% in $T_pS^1$,
% using the coordinate presentation on the chart $U_1$.  Show that there is a unique
% choice of $X_{p}$ for $p=\rme^{\ci \pi}$ which makes $X$ into a (smooth) vector field $X$ on $S^1$.  In contrast, show that it is not possible to extend $Y$ into a smooth vector field on $S^1$.
% (These results should become obvious once you tranform the vectors for $p\in U_1\cap U_2$ into the coordinate system of $U_2$).  
% Compute the (complete) flow map $\sigma:\mathbb{R}\times S^1\to S^1$ generated by $X$.  Compute also all {\em fixed points\/} of the flow, i.e., points $p\in S^1$ for which $\sigma(t,p)=p$ for all $t$.

% \item Show that under a coordinate transformation $(x_1, \ldots, x_n) \mapsto (y_1({\vec x}), \ldots, y_n({\vec x}))$ in $\mathbb{R}^n$ the $n$-form
% \begin{equation*}
% dy^1\wedge dy^2\wedge\cdots \wedge dy^n = J({\vec y},{\vec x}) dx^1\wedge dx^2\wedge\cdots\wedge dx^n,
% \end{equation*}
% where $J$ is the Jacobian determinant
% \begin{equation*}
% J({\vec y},{\vec x}) = \mathrm{det}\left(\frac{\partial y^i}{\partial x^j}\right).
% \end{equation*}
% {\em Hint:} you may need the following determinant formula
% \begin{equation}
% \det\left(\frac{\partial y^{i}}{\partial x^{j}}\right) = \varepsilon^{j_{1}\ldots j_{n}}\frac{\partial y^{1}}{\partial x^{j_{1}}}\ldots\frac{\partial y^{n}}{\partial x^{j_{n}}}
% \end{equation}
% which uses the completely antisymmetric $\varepsilon$-symbol. This is the general determinant formula
% \begin{equation}
% \det M = \det (M_{ij}) =  \varepsilon^{j_{1}\ldots j_{n}} M_{1j_1}\cdots M_{nj_n} 
% \end{equation}
% applied to the Jacobian matrix. You can find the formula for the determinant of an $n\times n$ matrix e.g. in Wikipedia.

\item Compact manifold $S^1$ on the chart $U_1$, this has coordinate presentation 
$$ (x,y) = (\cos(\theta), \sin(\theta)) $$
Where 
$$ U_1 = \{ e^{i \theta} | \theta \in I_i \} $$
$$ I_1 = ( - \pi, \pi ) $$
$$ I_2 = (0, 2\pi) $$
Like in the lecture notes we need to partition the unity, so for $x = \cos (\theta), \theta = \cos^{-1} (x)$ and we have 
$$ \frac{\partial}{\partial \theta} = \frac{\partial \theta}{\partial x} \frac{\partial}{\partial x} = \frac{\partial}{\partial x} \cos^{-1} (x) \frac{\partial}{\partial x} $$
$$ = - \frac{1}{\sqrt{1-x^2}} \frac{\partial}{\partial x} $$
This comes from the trig identitity $\sin^2 + \cos^2 =1$ which gives 
$$ \cos (\theta) = \pm \sqrt{1-\sin^2 (\theta)} $$
$$ x = \pm \sqrt{1-y^2} \rightarrow (\sqrt{1-y^2}, y) $$
And 
$$ \sin (\theta) = \pm \sqrt{1- \cos^2 (\theta) } $$
$$ y = \pm \sqrt{1- x^2} \rightarrow (x, \pm \sqrt{1-x^2}) $$
So for $y = \sin (\theta) \rightarrow \theta = \sin^{-1} (y) $
$$ \frac{\partial}{\partial \theta} = \frac{\partial \theta}{\partial y} \frac{\partial}{\partial y} = \frac{\partial}{\partial y} \sin^{-1} (y) \frac{\partial}{\partial y} $$
$$ = \frac{1}{\sqrt{1-y^2}} \frac{\partial}{\partial y} $$
For $X_p = \sin^2 ( \theta) \partial_{\theta}$
$$ X_p = \sin^2 ( \sin^{-1} (y)) \frac{1}{\sqrt{1-y^2}} \frac{\partial}{\partial y} $$
$$ = \frac{y^2}{\sqrt{1-y^2}} \frac{\partial}{\partial y} $$
For the missing point we take the right and left limits s.t $p = e^{i \theta} \rightarrow p(- \pi) = p( \pi) = (-1,0)$ 
$$ \lim_{y \to 0^+} X_p =0 $$
$$ \lim_{y \to 0^-} X_p =0 $$
In order for $X_p$ to be continuous $X_p (\pi)$ has to be equal to 0. So there is only 1 unique choice for $X_p$ which makes it into a smooth vector field. 
\\
For $Y$ we have the limits as $\theta \to \pi$ or $\theta \to - \pi$ 
$$ \lim_{\theta \to \pi} \theta \partial_{\theta} = \pi \partial_{\theta} $$
$$ \lim_{\theta \to - \pi} \theta \partial_{\theta} = - \pi \partial_{\theta} $$
For both of these they correspond to the same point as $\cos(\pi), \sin(\pi) = \cos(-\pi), \sin(-\pi)$. 
For the vector field to be smooth on $S^1$ we need to have 
$$ \frac{\partial a}{\partial \theta} |_{\theta = \pi} \neq 0 $$
Where $a$ is a smooth function. So we have 
$$ \lim_{\theta \to \pi} Y_p a = \pi \partial_{\theta} a $$
$$ \lim_{\theta \to - \pi} Y_p a = - \pi \partial_{\theta} a $$ 
These two are not equal to each other so there is no value of $Y_p$ for which $\theta = \pi$, thus $Y_p$ cannot be extended into a smooth vector field on $S^1$ 

\item From the lecture notes we have 
$$ dy^{\nu} = \frac{\partial y^{\nu} }{\partial x^{\mu}} dx^{\mu} = \partial_{\mu} y^{\nu} dx^{\mu} $$
So 

$$  dy^1 \wedge dy^2 \wedge \ldots \wedge dy^n = \partial_{\mu_1} y^1 dx^{\mu_1} \wedge \partial_{\mu_2} y^2 dx^{\mu_2} \wedge \ldots \wedge \partial_{\mu_n} y^n dx^{\mu_n} $$ 
The indicies of wedge products are totally anti-symmetric so 
$$ \varepsilon^{1 \ldots n} d x^{\mu_1} \wedge dx^{\mu_2} \wedge \ldots dx^{\mu_n} = \varepsilon^{\mu_1 \ldots \mu_n} dx^1 \wedge dx^2 \wedge dx^n $$
Therefore the equation above simplifies to 
$$ \varepsilon^{1 \ldots n} dy^1 \wedge dy^2 \wedge \ldots \wedge dy^n = \varepsilon^{\mu_1 \ldots \mu_n} \partial_{\mu_1} y^1 dx^1 \wedge \partial_{\mu_2} y^2 dx^2 \wedge \ldots \wedge \partial_{\mu_n} y^n dx^n $$
$$ = \varepsilon^{\mu_1 \ldots \mu_n} \frac{\partial y^1}{\partial x^{\mu_1}} dx^1 \wedge \frac{\partial y^2}{\partial x^{\mu_2}} dx^2 \wedge \ldots \wedge \frac{\partial y^n}{\partial x^{\mu_n}} dx^n $$  
Now using the definition of $J({\vec y},{\vec x})$ down using the determinant equation gives 
$$ \varepsilon^{1 \ldots n} dy^1 \wedge dy^2 \wedge \ldots \wedge dy^n = J({\vec y},{\vec x}) dx^1 \wedge dx^2 \wedge \ldots \wedge dx^n $$
Where $\varepsilon^{1 \ldots n} =1$

\item Let $\omega_q$ be a $q$- form 
$$
   \omega_q = \frac{1}{q!} \omega_{\mu_1\mu_2\cdots \mu_q} dx^{\mu_1}\wedge dx^{\mu_2} \wedge \cdots \wedge dx^{\mu_q} \ ,
 $$

 \begin{enumerate}
   \item Showing that $\omega_q \wedge \eta_r = (-1)^{qr} \eta_r \wedge \omega_q$. Suppose, for example we have 
   $$ \omega_q = \omega_1 \wedge \omega_2 \wedge \ldots \wedge \omega_q $$
   Where $\omega_n = d x^{\mu_n}$ for example. $\eta_r$ will be an $r$-form version 
   $$ \eta_r = \eta_1 \wedge \eta_2 \wedge \ldots \wedge \eta_r $$
   Where $\eta_n = dx^{\alpha_n}$ for example. This means that 
   $$ \omega_q \wedge \eta_r =  \omega_1 \wedge \omega_2 \wedge \ldots \wedge \omega_q \wedge \eta_1 \wedge \eta_1 \wedge \ldots \wedge \eta_r $$
   To now get $ \eta_r \wedge \omega_q$ we swap each of the $\omega$'s and $\eta$'s by moving each $\eta_n$ over $q$ times. By doing each swap each time we have a negative sign which totally gives $(-1)^{qr}$, the $r$ part comes from us having $r$, $\eta$'s so that it takes $qr$ swaps. Therefore we have 
   $$ \omega_q \wedge \eta_r = (-1)^{qr} \eta_r \wedge \omega_q $$

   \item $\omega_q \wedge \eta_r = - \omega_q \wedge \eta_r$ implies that $\omega_q \wedge \omega_q = - \omega_q \wedge \omega_q$ meaning that 
   $$ \omega_q \wedge \omega_q = 0 $$

 \end{enumerate}

 \item Following differential forms in $\mathbb{R}^3$: 
 $$ \alpha = x dx + ydy + zdz $$
$$ \beta = z dx + x dy + y dz $$
$$ \gamma = dy dz $$
\begin{enumerate}
  \item $\alpha$ is closed because we can define a function which has gradient equal to $\alpha$ such that 
  $$ \nabla \times \alpha = 0 $$
  For $\alpha$ as it is, this is very simple 
  $$ \frac{1}{2} ( x^2 + y^2 + z^2) \nabla \times \alpha = 0 $$
  $$ = \frac{1}{2} ( 2x dx + 2y dy + 2z dz ) \times \alpha = (xdx + ydy + z dz ) \times \alpha = 0 $$
  For $\gamma$ at (1,1,1), $\nabla \times \vec{F} = <x,-y, 0 >$ where $F_1 =0, F_2 =0, F_3 = xy \rightarrow \vec{F} = <F_1, F_2, F_3 > $. So $\gamma$ is not closed as 
  $$ \nabla \times \vec{F} \neq 0 $$
  \textbf{N.B} To calculate curls and gradients I used an online calculator. 

  \item $\alpha \wedge \beta$, so from question 3 we have 
  $$ \alpha \wedge \beta = (x dx + ydy + zdz) \wedge (zdz + x dy + y dz) $$
  $$ = (x^2 - yz) dx \wedge dy + (y^2 - zx) dy \wedge dz + (z^2 - xy) dz \wedge dx $$
  And for $(\alpha + \gamma) \wedge ( \alpha + \gamma)$ this would simply be 
  $$ (\alpha + \gamma) \wedge ( \alpha + \gamma) = 0 $$
\end{enumerate}

\item $M = \mathbb{R}^3, \omega = \omega_x dx + \omega_y dy + \omega_z dz$ 
$$ \int_S (\nabla \times \vec{\omega})\cdot d\vec{S}= \oint_C \vec{\omega}\cdot d\vec{s} $$
Where $\vec{\omega} = (\omega_x, \omega_y, \omega_z ) $. Working from the LHS to the RHS by expanding out the curl on $\vec{\omega}$. As $\omega$ is a 2-form it can be written as 
$$ \omega = \omega_x dx \wedge dy + \omega_y dy \wedge dz + \omega)z dz \wedge dx $$
Such that $d \omega$ can be written as 
$$ d \omega = \Big( \frac{\partial \omega_y}{\partial x} - \frac{\partial \omega_z}{\partial y} \Big) dx \wedge dy $$
$$ + \Big( \frac{\partial \omega_z}{\partial y} - \frac{\partial \omega_y}{\partial z} \Big) dy \wedge dz $$
$$ + \Big( \frac{\partial \omega_z}{\partial x} - \frac{\partial \omega_x }{\partial z } \Big) dx \wedge dz $$ 
Therefore, the curl is 
$$ \nabla \times \vec{\omega} = \frac{\partial \omega_y}{\partial x} dx \wedge dy - \frac{\partial \omega_y}{\partial z} dy \wedge dz $$
To compute this is much easier if we have the same wedge products. Rewriting 
$$ \nabla \times \vec{\omega} = \frac{\partial \omega_y}{\partial x} dx \wedge dy + \frac{\partial \omega_y}{\partial z} dz \wedge dy $$
So that the LHS integral is 
$$ \int_S \nabla \times \vec{\omega} d \vec{s} = \int_S \Big( \frac{\partial \omega_y}{\partial x} dx \wedge dy + \frac{\partial \omega_y}{\partial z} \wedge dy \Big) d \vec{s} $$
This can now be put into the form needed on the RHS of eq (3) from PS by noting that 
$$ A = \omega_y d s $$
Such that 
$$ dA = \frac{\partial \omega_y}{\partial x} dx \wedge dy + \frac{\partial \omega_y}{\partial z} dz \wedge dy $$
So the above integral is 
$$ \int_S dA \cdot d \vec{s} $$
Which from the lecture notes and assuming that we have a single chart, this is equivalent to the closed curve integral 
$$ \int_S dA \cdot d \vec{s} = \oint_C A \cdot d \vec{s} $$
$$ = \oint_C \omega_y dy \cdot d \vec{s} $$
$$ = \oint_C \vec{\omega}\cdot d\vec{s} $$

Second part of this question not answered.


% \item Show that the exterior product satisfies the two properties below. For our purpose you may 
% use the basis expansions, and direct computations. Let $\omega_q$ be a $q$-form
% $$
%   \omega_q = \frac{1}{q!} \omega_{\mu_1\mu_2\cdots \mu_q} dx^{\mu_1}\wedge dx^{\mu_2} \wedge \cdots \wedge dx^{\mu_q} \ ,
% $$
% and  $\eta_r$ be a $r$-form. Show that
% \begin{enumerate}
% \item $\omega_q\wedge\eta_r=(-1)^{qr}\eta_r\wedge\omega_q$
% \item $\omega_q\wedge\omega_q=0$, when $q$ is odd
% \end{enumerate}
% {\em Hint:} (a) implies (b).

% %\newpage

% \item Consider the following differential forms in $\mathbb{R}^3$:
% $$
% \alpha = xdx + ydy + zdz \ ; \ \beta = zdx+ xdy + ydz \ ; \ \gamma = xydz \ .
% $$
% \begin{enumerate}
% \item Is $\alpha$ closed or exact? Is $\gamma$ closed or exact?
% \item Calculate $\alpha \wedge \beta$ and  $(\alpha + \gamma)\wedge (\alpha + \gamma)$.\\ ({\em Hint:} the previous problem may be helpful.)
% \end{enumerate}

% \item (Nakahara 6.2) Let $M=R^3$, $\omega =\omega_x dx +\omega_y dy+\omega_z dz$. Show that Stokes' theorem
% implies
% \be\label{s1}
% \int_S (\nabla \times \vec{\omega})\cdot d\vec{S}= \oint_C
% \vec{\omega}\cdot d\vec{s} \ ,
% \ee
% where
% $\vec{\omega}=(\omega_x,\omega_y,\omega_z )$ and $C$ is the boundary of a surface $S$.
% In a similar vein, for $\psi=\frac{1}{2}\psi_{\mu\nu}dx^\mu\wedge dx^\nu$, show
% \be\label{s2}
% \int_V \nabla \cdot \vec{\psi}~dV = \oint_S \vec{\psi}\cdot
% d\vec{S} \ ,
% \ee
% where the components of the vector $\vec{\psi}$ are
% $\psi^\lambda=\varepsilon^{\lambda\mu\nu}\psi_{\mu\nu}$ and $S$ is the boundary
% of a volume $V$. 

% ({\em N.B.} This problem is meant to be non-trivial. By using the definition of integral of an $n$-form, motivate how the integrals are reduced to the usual integrals 
% in $\mathbb{R}^3$. No need to consider  several charts. Wikipedia might be useful.)




\end{enumerate}

\end{document}


