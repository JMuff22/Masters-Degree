
\documentclass[12pt]{article}


\usepackage{latexsym}
\usepackage[english,finnish]{babel}
\usepackage{bbm}
\usepackage{mathrsfs}
\usepackage{ifthen}
\usepackage{url}
\usepackage{enumerate}
\usepackage{fancyhdr}
% AMS packages:
\usepackage{amsbsy}
\usepackage{amsfonts}
\usepackage{amsmath}
\usepackage{amssymb}
\usepackage{amsthm}
\usepackage{amsxtra}
% for comments to work
\usepackage{verbatim}
\usepackage{graphicx}

\newcommand{\pat}{\partial}
\newcommand{\be}{\begin{equation}}
\newcommand{\ee}{\end{equation}}
\newcommand{\bea}{\begin{eqnarray}}
\newcommand{\eea}{\end{eqnarray}}
\newcommand{\abf}{{\bf a}}
\newcommand{\Zcal}{{\cal Z}_{12}}
\newcommand{\zcal}{z_{12}}
\newcommand{\Acal}{{\cal A}}
\newcommand{\Fcal}{{\cal F}}
\newcommand{\Ucal}{{\cal U}}
\newcommand{\Vcal}{{\cal V}}
\newcommand{\Ocal}{{\cal O}}
\newcommand{\Rcal}{{\cal R}}
\newcommand{\Scal}{{\cal S}}
\newcommand{\Lcal}{{\cal L}}
\newcommand{\Hcal}{{\cal H}}
\newcommand{\hsf}{{\sf h}}
\newcommand{\half}{\frac{1}{2}}
\newcommand{\Xbar}{\bar{X}}
\newcommand{\xibar}{\bar{\xi }}
\newcommand{\barh}{\bar{h}}
\newcommand{\Ubar}{\bar{\cal U}}
\newcommand{\Vbar}{\bar{\cal V}}
\newcommand{\Fbar}{\bar{F}}
\newcommand{\zbar}{\bar{z}}
\newcommand{\wbar}{\bar{w}}
\newcommand{\zbarhat}{\hat{\bar{z}}}
\newcommand{\wbarhat}{\hat{\bar{w}}}
\newcommand{\wbartilde}{\tilde{\bar{w}}}
\newcommand{\barone}{\bar{1}}
\newcommand{\bartwo}{\bar{2}}
\newcommand{\nbyn}{N \times N}
\newcommand{\repres}{\leftrightarrow}
%\newcommand{\Tr}{{\rm Tr}}
%\newcommand{\tr}{{\rm tr}}
\newcommand{\ninfty}{N \rightarrow \infty}
\newcommand{\unitk}{{\bf 1}_k}
\newcommand{\unitm}{{\bf 1}}
\newcommand{\zerom}{{\bf 0}}
\newcommand{\unittwo}{{\bf 1}_2}
\newcommand{\holo}{{\cal U}}
\newcommand{\ket}[1]{\vert{#1}\rangle}
\newcommand{\bra}[1]{\langle{#1}\vert}
\newcommand{\muhat}{\hat{\mu}}
\newcommand{\nuhat}{\hat{\nu}}
\newcommand{\rhat}{\hat{r}}
\newcommand{\phat}{\hat{\phi}}
\newcommand{\that}{\hat{t}}
\newcommand{\shat}{\hat{s}}
\newcommand{\zhat}{\hat{z}}
\newcommand{\what}{\hat{w}}
\newcommand{\sgamma}{\sqrt{\gamma}}
\newcommand{\bfE}{{\bf E}}
\newcommand{\bfB}{{\bf B}}
\newcommand{\bfM}{{\bf M}}
\newcommand{\cl} {\cal l}
\newcommand{\ctilde}{\tilde{\chi}}
\newcommand{\ttilde}{\tilde{t}}
\newcommand{\ptilde}{\tilde{\phi}}
\newcommand{\utilde}{\tilde{u}}
\newcommand{\vtilde}{\tilde{v}}
\newcommand{\wtilde}{\tilde{w}}
\newcommand{\ztilde}{\tilde{z}}

% misc
\newcommand{\vc}[1]{\mathbf{#1}}
\newcommand{\defem}[1]{{\em #1\/}}
\newcommand{\vep}{\varepsilon}
\newcommand{\wto}{\overset{{\rm w}}{\to}}
\newcommand{\sto}{\overset{{\rm s}}{\to}}
\DeclareMathOperator{\wlim}{w-lim}
\DeclareMathOperator{\slim}{s-lim}
\DeclareMathOperator{\tr}{Tr}

\newcommand{\qand}{\quad\text{and}\quad}

% To define sets:
\newcommand{\defset}[2]{ \left\{ #1 \left|\, #2\makebox[0pt]{$\displaystyle\phantom{#1}$}\right.\!\right\} }

\newcounter{alplisti}
\renewcommand{\thealplisti}{\alph{alplisti}}
\newenvironment{alplist}[1][(\thealplisti)]{\begin{list}{{\rm #1}\ }{ %
      \usecounter{alplisti} %
    \setlength{\itemsep}{0pt}
    \setlength{\parsep}{0pt}  %
%    \setlength{\leftmargin}{5em} %
%    \setlength{\labelwidth}{5em} %
%    \setlength{\labelsep}{1em} %
%    \settowidth{\labelwidth}{(DR2)}
     \setlength{\topsep}{0pt} %
}}{\end{list}}

% Norms:
\newcommand{\abs}[1] {\lvert #1 \rvert}
\newcommand{\norm}[1]{\lVert #1 \rVert}
\newcommand{\floor}[1] {\lfloor {#1} \rfloor}
\newcommand{\ceil}[1]  {\lceil  {#1} \rceil}

% Basic spaces
\newcommand{\R} {\mathbb{R}}
\newcommand{\C} {{\mathbb{C}}}
\newcommand{\Rd} {{\mathbb{R}^{d}}}
\newcommand{\N} {\mathbb{N}}
\newcommand{\Z} {\mathbb{Z}}
\newcommand{\Q} {\mathbb{Q}}
\newcommand{\K} {\mathbb{K}}
\newcommand{\T} {\mathbb{T}}


\hoffset 0.5cm
\voffset -0.4cm
\evensidemargin -0.2in
\oddsidemargin -0.2in
\topmargin -0.2in
\textwidth 6.3in
\textheight 8.4in

\begin{document}

\normalsize

\baselineskip 14pt

\begin{center}
{\Large {\bf FYMM/MMP IIIb \ \ \ \  Solutions to Exam Homework\ \  Dec 15 2020}} \\
Jake Muff
Student number: 015361763
\end{center}

\bigskip

\enlargethispage*{2cm}

\section*{Exercise 1}
\begin{enumerate}
  \item $M_1 = \mathbb{R}^2 \backslash \{ (0,0) \} $. \\
  For this $S^1$ is a retract of $M_1$. A topological group is simple connected if and only if $X$ is path connected and the fundamental group is trivial, $\pi_1 (X) = \{ e \}$ \\
  For $M_1$ we are on a 2D space. $\mathbb{R}^2$ is contractible and its fundamental group is $\{ e \}$, i.e trivial. But for $M_1$ we have $\mathbb{R}^2$ minus a point at $(0,0)$. The $\mathbb{R}^2 \backslash \{ (0,0) \} $ deformation retracts on $S^1$ via the homotopy $H: (\mathbb{R}^2 \backslash \{ (0,0) \} \times [0,1] ) \rightarrow S^1$. As a result 
  $$ \Pi_1 (\mathbb{R}^2 \backslash \{ (0,0) \}) = \Pi_1 (S^1) = \mathbb{Z} $$
  
  \item $M_2 = \mathbb{R}^2 \backslash \{ (0,0), (1,0), (0,1) \} $ \\
  If we consider some standard coordinates in $\mathbb{R}^2$ i.e $(x_1 \ldots x_n) \in \mathbb{R}^3 $ and that $(0,0) \in \{ x_1 > 0 \}$ and $ (1,0) \in \{ x_1 < 0 \} $. And we have a variable $\varepsilon >0$ such that 
  $$ H = \{ (x_1 \ldots x_3) \in \mathbb{R}^2 : |x_1| < \varepsilon \} $$
  Which is disjoint from the points. Consider the open subsets of $\mathbb{R}^2 \backslash \{ (0,0), (1,0), (0,1) \} $
  $$ A = \{ (x_1, x_2) \in \mathbb{R}^2 : x_1 > - \varepsilon \} \backslash \{ (0,0), (1,0), (0,1) \} $$
  $$ B = \{ (x_1, x_2) \in \mathbb{R}^2 : x_1 < \varepsilon \} \backslash \{ (0,0), (1,0), (0,1) \} $$
  Clearly the union of these two subsets is $\mathbb{R}^2 \backslash \{ (0,0), (1,0), (0,1) \}$. If we denote the points in $A$ and $B$ as $k_A, k_B$ since none of the points is contained in $A \cap B$ so $k_A + k_B = k$. We can now use Van Kampen's theorem for fundamental groups to say that 
  $$ \Pi_1 (\mathbb{R}^2 \backslash \{ (0,0), (1,0), (0,1) \} ) = \mathbb{Z} \times \mathbb{Z} \times \mathbb{Z} $$

  \item $M_3 = S^2$. On a sphere we can continuously deform a loop around it so 
  $$ \Pi_1 (S^2) = \{ e \} $$ 
  
  \item $M_4 = S^2 \backslash \{ (1,00) \} $. This is a sphere with a point removed at the equator. A sphere and a point are simply connected. 
  Removing a point doesn't change the fact that you can go around the hole as there are two directions for the band to move across so it can be made into a single point thus 
  $$ \Pi_1 (S^2 \backslash \{ (1,0,0) \} ) = \{ e \} $$

  \item $M_5 = S^2 \backslash \{ (0,0,-1), (0,0,1) \} $
  $$ \Pi_1 (M_5) = \mathbb{Z} $$
  This manifold is simply connected. 
\end{enumerate}


\section*{Exercise 2}
Assuming $f$ is a smooth function. A tensor of type (0,1) is a vector field. Applying $\nabla_X$ to a tensor yields a tensor of the same type. The exterior derivative is a map $\Omega^r (M) \rightarrow \Omega^{r+1} (M)$ and has the property that $d^2 =0 \rightarrow d_{r+1} d_r = 0 $
$$ T = \nabla_X (df) \rightarrow (1,1) q=1, r=1 $$
$$ U = d( \nabla_X f) \rightarrow (1,1) q=1, r=1 $$
$$ V = d( L_X (df)) = L_X (d^2 f) = L_X 0 = 0 \rightarrow (0,0), q=0, r=0 $$
$$ W = L_X (d(df)) = L_x (0f) = 0 \rightarrow (0,0), q=0, r=0 $$


\section*{Exercise 3}
\begin{enumerate}
  \item $X$ is a smooth vector field on $M$ because we can assign a vector $v (p)$ to every point on the manifold i.e $X$ is a linear map 
  $$ X: C^{\infty} (M) \rightarrow C^{\infty} (M) $$ 
  However, byy adding the (removal of) North and South pole points to the manifold, we cannot have a smooth vector field as we no longer have continuous derivatives i.e it is not differentiable everywhere now. $M$ is of course a smooth manifold. 

  \item To compute the flow generated by the vector field we need to solve  an ODE in coordinates. So, suppose we have some points $p = (\theta \sigma, \phi \sigma)$ and we want the integral curve $y$ with $y (\sigma) = p$, i.e $y(t) = (\theta (t), \phi (t) )$. The condition that $y$ be an integral curve gives us 
  $$ y' (t) = X_{y(t)} $$
  Therefore 
  $$ \theta' (t) = 0 $$
  And 
  $$ \phi' (t) = \frac{1}{\sin(\theta (t))} $$
  Now $\theta$ must be constant $\theta \sigma$ so 
  $$ \phi' (t) = \frac{1}{\sin(\theta \sigma)} $$
  is constant. And 
  $$ \phi (t) = \frac{t}{\sin(\theta \sigma) + \phi \sigma} $$
  This means that the integral curve at $p = (\theta \sigma, \phi \sigma)$ is 
  $$ y(t) = (\theta \sigma, \frac{t}{\sin (\theta \sigma) } + \phi \sigma) $$
  The flow is then 
  $$ F(p, t) = (\theta, \frac{t}{\sin(\theta)} + \phi ) $$
  I belive this is the maximal flow if $y(t,p) =  \phi_p (t)$ for $(t,p)$ in an open subset of $\mathbb{R} \times M$, which we have. The vector field is well defined for $\sin (\theta \sigma) = \sigma $, therefore the maximal flow is not complete. 
\end{enumerate}

\section*{Exercise 4}
\begin{enumerate}
  \item $$ M = \{ (u,v) | uv <1 \} $$
  $$ ds^2 = - \frac{1}{1-uv)^2} ( du \otimes dv + dv \otimes dv ) $$
  A pseudo-Riemannian manifold has a pseudo-Riemannian metric which is a metric which is symmetric and non-degenerate. As we can see this is a manifold with signature 0 and therefore is pseudo-riemannian. I think it may actually be Lorentizan as it is 2 dimensional. 

  \item The Levi-Civita connection $\nabla$ associated with the metric has the property that $\nabla_g g = 0$ and the non-zero Christoffel symbols are 
  $$ \Gamma^u_{uu} = - \frac{v}{1-uv} $$
  $$ \Gamma^u_{uv} = - \frac{u}{1-uv} $$
  $$ \Gamma^u_{vv} = \frac{v}{1-uv} $$
  $$ \Gamma^v_{uu} = \frac{u}{1-uv} $$
  $$ \Gamma^v_{uv} = - \frac{v}{1-uv} $$
  $$ \Gamma^v_{vv} = - \frac{u}{1-uv} $$
  I have skipped those that can be deduced by symmetry. 
  \item Which of the following curves are geodesics 
  $$ c(t) = (0,t), c(t) = (t, -t), c(t) = (e^t, -e^{-t}) $$
  For this I calculated the associated tangent vector fields for each curve w.r.t the manifold and checked to see if 
  $$ \nabla_V V = 0 $$
  $$ \frac{d \nu^i}{dt} + \Gamma^i_{jk} \nu^i \nu^k = 0 $$
  The curve is a geodesic if $\gamma (t) =0$. 
  $$ c(t) = (0,t) \Rightarrow  \gamma'_1= \frac{\partial}{\partial v} $$
  $$ c(t) = (t, -t) \Rightarrow \gamma'_2 = \frac{\partial}{\partial u} - \frac{\partial}{\partial v} $$
  $$ c(t) = (e^t, -e^{-t}) \Rightarrow \gamma'_3 = e^t \frac{\partial}{\partial u} + e^{-t} \frac{\partial}{\partial v} $$
  So we have 
  $$ \gamma_1 (t) = -t \frac{\partial}{\partial u} $$
  $$ \gamma_2 (t) = (- \frac{2t}{t^2 +1} ) \frac{\partial}{\partial u} + (\frac{2t}{t^2 +1}) \frac{\partial}{\partial v} $$
  $$ \gamma_3 (t) = (\frac{1}{2} e^{-3t} + \frac{3}{2} e^t) \frac{\partial}{\partial u} + ( - \frac{1}{2} e^{3t} - \frac{3}{2} e^{-t} ) \frac{\partial}{\partial v} $$
  From my calculations, none of these are geodesics. For this question I heavily used Mathematica referencing the Problem Set 5 solutions to calculate the coefficients and various derivatives. 
\end{enumerate}



\section*{Exercise 5}
Describe the classifications of all irreducible unitary finite-dimensional representations of the simple Lie algebra $\mathfrak{su}$(3) (or $\mathfrak{su}$(2)) of the simple connected compact simple Lie group $SU(3)$ or $SU(2)$. I am not entirely sure what the classifications are, however, I will attempt to answer as best I can, after reading appendix A.4 in the lecture notes. I am not sure but I believe the theorem of the highest weight plays a substantial part, however this may be for semi-simple Lie groups. 
\\
% We can build  $\mathfrak{su}$(2) irreducible representations  (denoted $D$) with tensor products by expressing the heighest weight $\vec{\mu}$ of the lie algebra, which can be expanded 
% $$ \vec{\mu} = q_1 \vec{\mu_1} + \ldots q_m \vec{\mu_m} $$
% Where $q_i$ are integers and the components $\vec{\mu_1}$ are often denoted as "fundamental weights". For the lie algebra $\mathfrak{su}$(2), this corresponds to the spin $\frac{1}{2}$ representation, thus, we have 
% $$ j = q \cdot \frac{1}{2} $$
% where $j$ is the highest weight of the spin $J$ representation. From the fundamental representation we can build other representations using tensor products. Consider the two irreducible representations spin $j_1$ and $j_2$
% $$ \ket{j_1,\alpha_1}, \ket{j_2, \alpha_2} $$
% For $\alpha_1 = \alpha_1 \ldots - \alpha_1$ and $\alpha_2 = \alpha_2 \ldots - \alpha_2$. With tensor product 
% $$ \ket{j_1, \alpha_1} \otimes \ket{j_2, \alpha_2} $$
% If we now let the $\mathfrak{su}$(2) generators act on these we get 
% $$ J_a = J^1_a \otimes \mathbb{I}_2 + \mathbb{I}_1 \otimes J^2_a $$
% $$ J_a = J^1_a + J^2_a $$


% Let us build the irreducible representations with tensor products. The fundamental representations are the 3 (1,0) representations with 3 weights $\ket{1/2, 1/2\sqrt{3}}, \ket{-1/2, 1/2 \sqrt{3} }, \ket{0, -1/\sqrt{3}}$ and the $\bar{3}$ antifundamental (0,1) representation with weights $\ket{-1/2, -1/2 \sqrt{3}}, \ket{1/2, -1/2 \sqrt{3}}, \ket{0, 1/ \sqrt{3}}$. The $\mathfrak{su}$(3) generators use upper and lower indices given by 
% $$ (T_a)^i_j = \frac{1}{2} ( \lambda_a)_{ij} $$
% Where $\lambda_a$ are the Gell-Mann matrices. These act on the fundamental basis vector as 
% $$ (T^a)^j_i \ket{j} $$
% And on the anti fundamental basis vectors 
% $$ - (T^a)^i_j \ket{j} $$
% Using the summation convention a generic vector in the fundamental representation acts on the generators by 

To describe the classifications of all the irreducible representations we first need the Cartan matrix. As the defining representation of $\mathfrak{su}$(3) consists of 3x3 trace 0 unitary matrices we can easily find the cartan matrix. The basis of generators $T^a$ is normalzied such that $Tr (T^a T^b) = \frac{1}{2} \delta^{ab}$ for $a = 1 \ldots 8$. The Cartan generators $H_m$ are easily found 
$$ H_1 = \frac{1}{2}\begin{pmatrix}
  1&0&0\\0&-1&0\\0&0&0
\end{pmatrix} = \frac{1}{2} \lambda_3 $$
Where $\lambda_a$ are the Gell-Mann matrices. 
$$ H_2 = \frac{1}{2 \sqrt{3}}\begin{pmatrix}
  1&0&0\\0&1&0\\0&0&2
\end{pmatrix} = \frac{1}{2 \sqrt{3}} \lambda_8 $$
In the definining representation we have basis vectors 
$$ \begin{pmatrix}
  1\\0\\0
\end{pmatrix}, \begin{pmatrix}
  0\\1\\0
\end{pmatrix}, \begin{pmatrix}
  0\\0\\1
\end{pmatrix} $$
To find the weights $\mu$ we find the eigenvalues of the Cartan generators in the basis of the defining representation, giving 
$$ \mu^1 = (\frac{1}{2}, \frac{1}{2 \sqrt{3} }) $$
$$ \mu^2 = (-\frac{1}{2}, \frac{1}{2 \sqrt{3} }) $$
$$ \mu^3 = (0, - \frac{1}{\sqrt{3} }) $$
The raising and lowering operators of SU(3) 
$$ E^1_{\alpha} = i(\lambda_1 + i \lambda_2 ) = \begin{pmatrix}
  0&1&0 \\ 0&0&0 \\ 0&0&0 
\end{pmatrix} $$
$$ E^2_{\alpha} = i(\lambda_4 + i \lambda_5 ) = \begin{pmatrix}
  0&0&1 \\ 0&0&0 \\ 0&0&0 
\end{pmatrix} $$
$$ E^3_{\alpha} = i(\lambda_6 + i \lambda_7 ) = \begin{pmatrix}
  0&0&0 \\ 0&0&1 \\ 0&0&0 
\end{pmatrix} $$
$$ E^1_{- \alpha} = i(\lambda_1 - i \lambda_2 ) = \begin{pmatrix}
  0&0&0 \\ 1&0&0 \\ 0&0&0 
\end{pmatrix} $$
$$ E^2_{- \alpha} = i(\lambda_4 - i \lambda_5 ) = \begin{pmatrix}
  0&0&0 \\ 0&0&0 \\ 1&0&0 
\end{pmatrix} $$
$$ E^3_{- \alpha} = i(\lambda_6 - i \lambda_7 ) = \begin{pmatrix}
  0&0&0 \\ 0&0&0 \\ 0&1&0 
\end{pmatrix} $$
To find the roots (denoted $\alpha$ ) of SU(3) we find the weight states of the adjoint representation 
$$ [ H_1, E^1_{\alpha} ] = E^1_{\alpha} $$
$$ [H_2, E^1_{\alpha}] = 0 $$
Therefore the roots for $\alpha^1_+ = (1,0)$ and $\alpha_-^1 = (-1,0) $. For the second generator 
$$ [H_1, E_{\alpha}^2] = \frac{1}{2} E^2_{\alpha} $$
$$ [H_2, E_{\alpha}^2 ] = \frac{\sqrt{3}}{2} E^2_{\alpha} $$
Therefore the roots are $\pm \alpha^2 = \pm (\frac{1}{2}, \frac{\sqrt{3}}{2} ) $ and $\pm \alpha^3 = \pm (- \frac{1}{2}, \frac{\sqrt{3}}{2})$. From these we can find the simple roots (denoted $\vec{\alpha}$) 
$$ \vec{\alpha}^1 = (\frac{1}{2}, \frac{\sqrt{3}}{2} ) $$
$$ \vec{\alpha}^2 = (\frac{1}{2}, - \frac{\sqrt{3}}{2} ) $$
The fundamental weights (denoted $\vec{q}$) are found from 
$$ 2 \frac{\alpha^i \vec{q}^j}{|\alpha|^2} = \delta^{ij}  $$
On SU(3) these represent the $3$and $\bar{3}$ irreducible representations so we have 
$$ (\frac{1}{2}, \frac{1}{2 \sqrt{3}} ), (\frac{1}{2}, - \frac{1}{2 \sqrt{3}} ) $$
From these the cartan matrix can be found 
$$ 2 \vec{\alpha}_1 \vec{\alpha}_1 = -2 $$
$$ 2 \vec{\alpha}_1 \vec{\alpha}_2 = -1 $$
$$ 2 \vec{\alpha}_2 \vec{\alpha}_2 = -2 $$
The cartan matrix is then 
$$ C = \begin{pmatrix}
  2 & -1 \\ -1 & 2 
\end{pmatrix} $$
We can build another unitary representation from this by choosing another $H_m$ and $E_{\pm \alpha}$ such that 
$$ H_1 = \frac{1}{2} \begin{pmatrix}
  1&-&0 \\ 0&-1&0\\0&0&0
\end{pmatrix} $$
$$ H_2 = \frac{1}{2} \begin{pmatrix}
  0&0&0 \\ 0&1&0 \\ 0&0&-1
\end{pmatrix} $$
$$ E_{\alpha_1} = \begin{pmatrix}
  0&1&0 \\ 0&0&0 \\ 0&0&0 
\end{pmatrix}, E_{\alpha_2} = \begin{pmatrix}
  0&0&1 \\ 0&0&0 \\ 0&0&0 
\end{pmatrix}, E_{\alpha_3} = \begin{pmatrix}
  0&0&0 \\ 0&0&1 \\ 0&0&0 
\end{pmatrix} $$
WIth weights 
$$ \mu^1 = (\frac{1}{2}, 0), \mu^2 = (- \frac{1}{2}, \frac{1}{2} ), \mu^3 = (0, - \frac{1}{2} ) $$
And roots 
$$ \alpha_1 = (1, - \frac{1}{2} ), \alpha_2 = (\frac{1}{2}, \frac{1}{2} ), \alpha_3 = (- \frac{1}{2}, 1) $$
The simple roots are 
$$ \alpha^{12} = \alpha_1 - \alpha_2 = (\frac{1}{2}, 1), \alpha^{23} = \alpha_2 - (- \alpha_3) = (0, - \frac{3}{2} ) $$
The fundamental highest weights $q^j$ are given by 
$$ q^1 = (\frac{1}{2}, 0), q^2 = (0, - \frac{1}{2} ) $$
In terms of the classification theorem (Appendix A.4) I believe these are represented by the $A_n$ and this corresponds to the same Dynkin diagram given in the notes. 


\end{document}

