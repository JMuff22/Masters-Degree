
\documentclass[12pt]{article}
%\usepackage[finnish]{babel}
\usepackage[T1]{fontenc}
\usepackage[utf8]{inputenc}
\usepackage{amssymb}
\usepackage{amsmath}
\usepackage{hyperref}
\newcommand{\pat}{\partial}
\newcommand{\be}{\begin{equation}}
\newcommand{\ee}{\end{equation}}
\newcommand{\bea}{\begin{eqnarray}}
\newcommand{\eea}{\end{eqnarray}}
\newcommand{\abf}{{\bf a}}
\newcommand{\Zcal}{{\cal Z}_{12}}
\newcommand{\zcal}{z_{12}}
\newcommand{\Acal}{{\cal A}}
\newcommand{\Fcal}{{\cal F}}
\newcommand{\Ucal}{{\cal U}}
\newcommand{\Vcal}{{\cal V}}
\newcommand{\Ocal}{{\cal O}}
\newcommand{\Rcal}{{\cal R}}
\newcommand{\Scal}{{\cal S}}
\newcommand{\Lcal}{{\cal L}}
\newcommand{\Hcal}{{\cal H}}
\newcommand{\hsf}{{\sf h}}
\newcommand{\half}{\frac{1}{2}}
\newcommand{\Xbar}{\bar{X}}
\newcommand{\xibar}{\bar{\xi }}
\newcommand{\barh}{\bar{h}}
\newcommand{\Ubar}{\bar{\cal U}}
\newcommand{\Vbar}{\bar{\cal V}}
\newcommand{\Fbar}{\bar{F}}
\newcommand{\zbar}{\bar{z}}
\newcommand{\wbar}{\bar{w}}
\newcommand{\zbarhat}{\hat{\bar{z}}}
\newcommand{\wbarhat}{\hat{\bar{w}}}
\newcommand{\wbartilde}{\tilde{\bar{w}}}
\newcommand{\barone}{\bar{1}}
\newcommand{\bartwo}{\bar{2}}
\newcommand{\nbyn}{N \times N}
\newcommand{\repres}{\leftrightarrow}
\newcommand{\Tr}{{\rm Tr}}
\newcommand{\tr}{{\rm tr}}
\newcommand{\ninfty}{N \rightarrow \infty}
\newcommand{\unitk}{{\bf 1}_k}
\newcommand{\unitm}{{\bf 1}}
\newcommand{\zerom}{{\bf 0}}
\newcommand{\unittwo}{{\bf 1}_2}
\newcommand{\holo}{{\cal U}}
\newcommand{\bra}{\langle}
\newcommand{\ket}{\rangle}
\newcommand{\muhat}{\hat{\mu}}
\newcommand{\nuhat}{\hat{\nu}}
\newcommand{\rhat}{\hat{r}}
\newcommand{\phat}{\hat{\phi}}
\newcommand{\that}{\hat{t}}
\newcommand{\shat}{\hat{s}}
\newcommand{\zhat}{\hat{z}}
\newcommand{\what}{\hat{w}}
\newcommand{\sgamma}{\sqrt{\gamma}}
\newcommand{\bfE}{{\bf E}}
\newcommand{\bfB}{{\bf B}}
\newcommand{\bfM}{{\bf M}}
\newcommand{\cl} {\cal l}
\newcommand{\ctilde}{\tilde{\chi}}
\newcommand{\ttilde}{\tilde{t}}
\newcommand{\ptilde}{\tilde{\phi}}
\newcommand{\utilde}{\tilde{u}}
\newcommand{\vtilde}{\tilde{v}}
\newcommand{\wtilde}{\tilde{w}}
\newcommand{\ztilde}{\tilde{z}}


\hoffset 0.5cm
\voffset -0.4cm
\evensidemargin -0.2in
\oddsidemargin -0.2in
\topmargin -0.2in
\textwidth 6.3in
\textheight 8.4in

\begin{document}

\normalsize

\baselineskip 14pt

\begin{center}
{\Large {\bf FYMM/MMP IIIb 2020 \ \ \ Solutions to  Problem Set 2}}
Jake Muff 
7/11/20
\end{center}


\begin{enumerate}

\item %\setminus 
If we have 
$$ \mathbb{R} \setminus \{p_1\} \  \textrm{and} \ \mathbb{R}^2 \setminus \{ p_2 \} $$
Which are not homeomorphic as they are connect connected so 
$$ \mathbb{R} \setminus \{ p \} $$
is not connected but 
$$ \mathbb{R}^2 \setminus \{ p \} $$
is connected. So, from the theorem $\mathbb{R}$ must not be homeomorphic to $\mathbb{R}^2$.


\item $\pi_2 (M)$ is a two-loop. Need to find the fundamental groups. 
\begin{enumerate}
    \item Not answered. 
    
    \item $M=T^2$, $T^2 = S^1 \times S^1$. The fundamental group of $S^1$ is $\{ e\}$ so we have 
    $$ \pi_2 \{ T^2 \} = \{e \} \times \{ e\} = \{ e\} $$
    So $\pi_2 (M) = \{ e \} $
\end{enumerate}

\item $S^1$ in cartesian coordinates is 
$$ x^2 + y^2 =1 $$
Such that $(x,y)$ is a point in $\mathbb{R}^2$. So $S^1$ is a subset of $\mathbb{R}^2$: $S^1 = \{ x \in \mathbb{R}^2 | \sum_i^2 (x^i)^2 =1 \} $. 
So we have 2 sets of coordinate neighbourhoods 
$$ U_{1+} \equiv \{ (x,y) \in S^1 | x > 0 \} $$
$$ U_{1-} \equiv \{ (x,y) \in S^1 | x < 0 \} $$
$$ U_{2+} \equiv \{ (x,y) \in S^1 | y > 0 \} $$
$$ U_{2-} \equiv \{ (x,y) \in S^1 | y < 0 \} $$
With coordinate functions 
$$ V_{1 \pm} (x,y) =y $$
Where $V_{1 \pm}$ maps $U_{1 \pm} \rightarrow (-1,1)$. And 
$$ V_{2 \pm} (x,y) = x $$
Where $V_{2 \pm}$ maps $U_{2 \pm} \rightarrow (-1,1)$. 
\\
These are coordinate neighbourhoods and functions if they are open and a homeomorphism. These can be shown to be homeomorphisms by construction onverses from the transistion functions: 
$ \psi_{U \pm}$ maps $(-1,1) \rightarrow S^1$ 
$$ \psi_{V_{1 \pm}} (x,y) = ( \pm \sqrt{1-x^2}, x) $$
$$ = (\pm \sqrt{1-y^2}, y) $$
$$ \psi_{V_{2 \pm}} (x,y) = (x,\pm \sqrt{1-x^2}) $$
$$ = (y, \pm \sqrt{1-y^2}) $$
\item Express the vector field
$$
V = \frac{\partial}{\partial x} +  \frac{\partial}{\partial y} +  \frac{\partial}{\partial z}
$$
in {\em paraboloidal} coordinates $(u,v,\varphi)$, the coordinate transformation is
$$
x = uv\cos \varphi \ , \ y = uv \sin \varphi \ , \ z = \half (u^2 - v^2 ) .
$$
So I need to find functions of $u, v$ and $\varphi$. If we divide $y$ by $x$ we get a function of $\varphi$
$$ \frac{y}{x} = \frac{uv \sin(\varphi)}{uv \cos(\varphi)} = \tan(\varphi) $$
And that 
$$ x^2 + y^2 = (uv \cos(\varphi))^2 + (uv \sin(\varphi))^2 $$
$$ = u^2 v^2 \cos^2 (\varphi) + u^2 v^2 \sin^2 (\varphi) $$
Because $\sin^2 + \cos^2 =1$ this can be written as
$$ = u^2 v^2 $$
Notice that 
$$ \sqrt{u^2 v^2 + (\frac{1}{2} (u^2 - v^2) )^2} = \frac{u^2 +v^2}{2} $$
So adding $z$ and taking another square root gives $u$ and adding $-z$ and another square root gives $v$ (derived from wolffram alpha) 
$$ u = \sqrt{\sqrt{u^2 v^2 + (\frac{1}{2} (u^2 - v^2))^2 }+ (\frac{1}{2} u^2 - v^2)} $$
$$ = \sqrt{\sqrt{x^2 + y^2 + z^2 } + z} $$
And 
$$ v = \sqrt{\sqrt{x^2 + y^2 + z^2 }-z} $$
From the lecture notes, a general transformation of coordinates is given by 
$$ X = X^{\mu} \frac{\partial}{\partial x^{\mu}} = Y^{\mu} \frac{\partial}{\partial y{\mu}} $$
Where 
$$ X^{\mu} \frac{\partial}{\partial x^{\mu}} = X^{v} \frac{\partial y^{\mu} }{\partial x^{v}} \frac{\partial}{\partial y^{\mu}} $$
From the chain rule. The version we need is 
$$ \frac{\partial y^{\mu} }{\partial x^v} \Rightarrow y^{\mu} = (u,v,\varphi) $$
So that we have 3 partial fractions for $u$, 3 for $v$ and 3 for $\varphi$ for $x,y,z$ 
$$ \frac{\partial u}{\partial x} = \frac{x}{2 \sqrt{\sqrt{x^2 + y^2 + z^2 } + z} \cdot \sqrt{x^2 + y^2 +z^2}} $$
Subsituting $u$ and $v$ 
$$ = \frac{uv \cos(\varphi)}{\sqrt{\sqrt{u^2 v^2 + ( \frac{1}{2} (u^2 - v^2))^2} + \frac{1}{2} (u^2 - v^2) } \sqrt{u^2 v^2 + (\frac{1}{2} (u^2 -v^2))^2}} $$
$$ = \frac{ v \cos (\varphi)}{u^2 + v^2} $$
Doing this for the rest then gives 
$$ \frac{\partial u}{\partial y} = \frac{ v \sin (\varphi)}{u^2 + v^2} $$
$$ \frac{\partial u}{\partial z} = \frac{ u }{u^2 + v^2} $$
$$ \frac{\partial v}{\partial x} = \frac{ u \cos (\varphi)}{u^2 + v^2} $$
$$ \frac{\partial v}{\partial y} = \frac{ u \sin (\varphi)}{u^2 + v^2} $$
$$ \frac{\partial v}{\partial z} = -\frac{ v }{u^2 + v^2} $$
$$ \frac{\partial \varphi}{\partial x} = - \frac{ \sin (\varphi)}{uv} $$
$$ \frac{\partial \varphi}{\partial y} = - \frac{ \cos (\varphi)}{uv} $$
$$ \frac{\partial \varphi }{\partial z} = 0 $$
All of which were evaluated using wolffram alpha. So that the vector field in paraboloidal coordinates is 
$$ X = \frac{u + v( \sin(\varphi) + \cos(\varphi) ) }{u^2 + v^2} \frac{\partial}{\partial u} + \frac{-v + u(\sin(\varphi) + \cos(\varphi) )}{u^2 + v^2} \frac{\partial}{\partial v} + \frac{\cos(\varphi) - \sin(\varphi) }{uv} \frac{\partial}{\partial \varphi} $$


\end{enumerate}


\end{document}


