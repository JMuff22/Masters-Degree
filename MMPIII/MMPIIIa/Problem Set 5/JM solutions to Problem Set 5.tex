
\documentclass[12pt]{article}
\usepackage[finnish]{babel}
\usepackage[T1]{fontenc}
\usepackage[utf8]{inputenc}
\usepackage{amssymb}
\usepackage{bbm}
\newcommand{\pat}{\partial}
\newcommand{\be}{\begin{equation}}
\newcommand{\ee}{\end{equation}}
\newcommand{\bea}{\begin{eqnarray}}
\newcommand{\eea}{\end{eqnarray}}
\newcommand{\abf}{{\bf a}}
\newcommand{\Zcal}{{\cal Z}_{12}}
\newcommand{\zcal}{z_{12}}
\newcommand{\Acal}{{\cal A}}
\newcommand{\Fcal}{{\cal F}}
\newcommand{\Ucal}{{\cal U}}
\newcommand{\Vcal}{{\cal V}}
\newcommand{\Ocal}{{\cal O}}
\newcommand{\Rcal}{{\cal R}}
\newcommand{\Scal}{{\cal S}}
\newcommand{\Lcal}{{\cal L}}
\newcommand{\Hcal}{{\cal H}}
\newcommand{\hsf}{{\sf h}}
\newcommand{\half}{\frac{1}{2}}
\newcommand{\Xbar}{\bar{X}}
\newcommand{\xibar}{\bar{\xi }}
\newcommand{\barh}{\bar{h}}
\newcommand{\Ubar}{\bar{\cal U}}
\newcommand{\Vbar}{\bar{\cal V}}
\newcommand{\Fbar}{\bar{F}}
\newcommand{\zbar}{\bar{z}}
\newcommand{\wbar}{\bar{w}}
\newcommand{\zbarhat}{\hat{\bar{z}}}
\newcommand{\wbarhat}{\hat{\bar{w}}}
\newcommand{\wbartilde}{\tilde{\bar{w}}}
\newcommand{\barone}{\bar{1}}
\newcommand{\bartwo}{\bar{2}}
\newcommand{\nbyn}{N \times N}
\newcommand{\repres}{\leftrightarrow}
\newcommand{\Tr}{{\rm Tr}}
\newcommand{\tr}{{\rm tr}}
\newcommand{\ninfty}{N \rightarrow \infty}
\newcommand{\unitk}{{\bf 1}_k}
\newcommand{\unitm}{{\bf 1}}
\newcommand{\zerom}{{\bf 0}}
\newcommand{\unittwo}{{\bf 1}_2}
\newcommand{\holo}{{\cal U}}
\newcommand{\bra}{\langle}
\newcommand{\ket}{\rangle}
\newcommand{\muhat}{\hat{\mu}}
\newcommand{\nuhat}{\hat{\nu}}
\newcommand{\rhat}{\hat{r}}
\newcommand{\phat}{\hat{\phi}}
\newcommand{\that}{\hat{t}}
\newcommand{\shat}{\hat{s}}
\newcommand{\zhat}{\hat{z}}
\newcommand{\what}{\hat{w}}
\newcommand{\sgamma}{\sqrt{\gamma}}
\newcommand{\bfE}{{\bf E}}
\newcommand{\bfB}{{\bf B}}
\newcommand{\bfM}{{\bf M}}
\newcommand{\cl} {\cal l}
\newcommand{\ctilde}{\tilde{\chi}}
\newcommand{\ttilde}{\tilde{t}}
\newcommand{\ptilde}{\tilde{\phi}}
\newcommand{\utilde}{\tilde{u}}
\newcommand{\vtilde}{\tilde{v}}
\newcommand{\wtilde}{\tilde{w}}
\newcommand{\ztilde}{\tilde{z}}
\newcommand{\C}{\mathbb{C}}
\newcommand{\R}{\mathbb{R}}

\newtheorem{theorem}{Theorem}

\hoffset 0.5cm
\voffset -0.4cm
\evensidemargin -0.2in
\oddsidemargin -0.2in
\topmargin -0.2in
\textwidth 6.3in
\textheight 8.4in

\begin{document}

\normalsize

\baselineskip 14pt

\begin{center}
{\Large {\bf FYMM/MMP IIIa 2020 \ \ \  Solutions to Problem Set 5}}
Jake Muff
\end{center}

\bigskip

\noindent

\begin{enumerate}

\item Show that
$$
  P_n \equiv \{ a_0 +a_1 z + a_2 z^2 +\cdots + a_n  z^n | a_0,a_1,\ldots ,a_n \in \mathbb{C}^n \}
$$
is a vector space.
\\
From the vector space axioms we can show that this is a vector space. 
\begin{enumerate}
  \item Closure under addition 
  $$ a_n Z^n + b_n Z^n = (a_n + b_n) Z^n $$
  \item Closure under multiplication
  $$ c a_n Z^n = (c a_n ) Z^n $$
  \item Associativity
  $$ (a_n Z^n + b_n Z^n) + c_n Z^n = (a_n + b_n +c_n)Z^n $$
  $$ = a_n Z^n + c b_n Z^n + c_n Z^n $$
  \item Identity element of addition 
  $$ a_n = 0 $$
  for all $n$ 
  $$ a_n Z^n + 0 = a_n Z^n $$
  \item Inverse elements of addition 
  $$ (a_n^{-1} = - a_n) a_n Z^n + (-a_n) Z^n $$
  $$ = (a_n -a_n)Z^n = 0 $$
  \item Commutivity of addition 
  $$ a_n Z^n + b_n Z^n = (a_n + b_n ) Z^n $$
  $$ = (b_n + a_n) Z^n = b_n Z^n + a_n Z^n $$
  \item Distributivity of scalar multiplication with respect to vector addition
  $$ c (a_n Z^n + b_n Z^n) = c (a_n + b_n) Z^n $$
  $$ = (c a_n)Z^n + (c b_n) Z^n $$
  \item Distributivity of scalar multiplication with respect to field addition. 
  $$ (c + d) a_n Z^n = c a_n Z^n + D a_n Z^n $$
  \item Identity element of scalar multiplication 
  $$ 1 \cdot a_n Z^n = a_n Z^n $$ 
  \item Associativity of scalar multiplication 
  $$ (cd)a_n Z^n = (c d a_n )Z^n = c (d a_n ) Z^n $$

\end{enumerate}
The dimension in the complex field is 
$$ dim P_n = n(n+1) $$ 

\item Find a faithful representation of $\mathbb{Z}_6$ in $\mathbb{R}^2$, thinking of group elements generated by anticlockwise 60 degree rotations.
\\
60 degree $ = \frac{\pi}{3}$. Rotation denoted $R_\theta$. Representation 
$$ D: \mathbb{Z}_6 \rightarrow Aut(\mathbb{R}^2) $$
Where 
$$ \mathbb{Z}_6 = \{e,a,a^2,a^3,a^4,a^5\} $$
$$ a^n \rightarrow R_{n \frac{\pi}{3}} $$
Because we have a rotation matrix
$$ D(a^n \cdot a^m) = D (a^{(n+m) \ mod \ 6}) $$
So that 
$$ R_{\frac{\pi}{3} [ (n+m) \ mod \ 6 ]} $$
$$ = R_{n \frac{\pi}{3}+m \frac{\pi}{3} \ mod \ 2 \pi} $$
$$ = R_{n \frac{\pi}{3}}R_{m \frac{\pi}{3}} = D(a^n) D(a^m) $$
So that $D$ is a homomorphism. 
$$ ker D = \{ a^n \in \mathbb{Z}_6 | R_{n \frac{\pi}{3}} = \mathbbm{1} \} $$
$$ = \{ a^n \in \mathbb{Z}_6 | n = 0 \ mod \ 6 \} $$
$$ = \{ a^0\} = \{ e\} $$
So $D$ is a faithful representation 

\item Show that $SL(n,\mathbb{R})$ is a normal subgroup of  $GL(n,\R)$, and identify the quotient group $GL(n,\R)/SL(n,\R)$.
\\ The first isomorphism theorem:
\begin{theorem}
  Let $G$ and $H$ be two groups and $\phi : G \rightarrow H$ be a group homomorphism. Then $ker \phi$ is a normal subgroup of $G$ and 
  $$ G / ker \phi \cong Im \phi $$
\end{theorem}
So we have 
$$ \phi : GL(n, \mathbb{R}) \rightarrow \mathbb{R} \backslash \{\vec{0}\} $$
Is a determinant map such that $A \rightarrow det(A)$. Determinants of $GL(,\mathbb{R})$ are non zero and determinats of $\mathbb{R}$ are real so $\phi$ works as a map here. So we have 
$$ \phi(AB) = det(AB) = det(A) det(B) = \phi(A) \phi(B) $$ 
So $\phi : GL(n, \mathbb{R}) \rightarrow \mathbb{R} \backslash \{\vec{0}\}$ is a homomorphism and 
$$ ker \phi = \{ A \in GL(n,\mathbb{R}) | det(A) =1 \} $$
which is equivalent to the special linear transform group $SL(n,\mathbb{R})$. By the first isomorphism theorem then 
$$ GL(,\mathbb{R}) \backslash ker \phi = GL(n, \mathbb{R}) \backslash SL(n, \mathbb{R}) \cong \mathbb{R} \backslash \{\vec{0}\} $$


\item Show that all group elements belonging to the same conjugacy class have the same order of element.
\\
In a group $G$, two elements $h$ and $g$ are conjugate when 
$$ h = x g x^{-1} $$
where $ x \in G$. To show that they have the same order we need to show that $g$ and $x g x^{-1} $ have the same order.
\\
In a group where $(x g x^{-1})^n = xg^n x^{-1}$ for $n>0$
$$ (x g x^{-1})^n = x g^n x^{-1} \forall n \in \mathbb{Z}^+$$ 
If $g^n =1$ then $(xgx^{-1})^n = xg^n x^{-1} = x x^{-1} = e$, and if $(x g x^{-1})^n =1$ then 
$$ x g^n x^{-1} = e$$
so 
$$ g^n = x x^{-1} = e $$ 
so 
$$ (x g x^{-1})^n =1$$
If and only if $g^n =1$ so $g$ and $x g x^{-1} $ have the same order. 

\item Show that a linear map $L: V\rightarrow V$ is an automorphism if and only if ${\rm Ker}~L= \{0\}$.
\\
\\
Let $L \in Aut(V)$ and $x \in ker L$ such that $L(x) =0$ and $L(x)=L(0)$. L is an automorphism meaning it is an injection and $x = 0$ and $ker L = \{0\}$. 
\\
Now suppose that $L: V \rightarrow V$ is an automorphism, which means proving that it is a bijection, thus proving surjectivity and injectivity:
\\
For surjectivity suppose we have $\forall y \in Y \exists x \in X s.t f(x) = y$. Suppose we have a basis for $V$ as $\{v_i\}$ then $L(v_i)$ is also a basis of $V$ as 
$$ \sum_{i=1}^n a_i L(v_i) = L \Big( \sum_{i=1}^n a_i v_i\Big) $$
$$ = \sum_{i=1}^n a_i v_i = 0 $$
$a_i$ is such that $a_i =0$. If we have a $u \in V $ such that $u_i$ is in the basis $\{v_i\}$ and $L(v_i)$ then 
$$ U = \sum_{i=1}^n u_i L(v_i) = L \Big(\sum_{i=1}^n u_i v_i\Big) $$
So that $u_i v_i \in V$ 
\\
For injectivity suppose we have a function $f(x) \neq f(x') \forall x \neq x' $ and we have $ x,y \in V s.t L(x) = L(y) $ and $L(x-y)=0$ due to $x-y \in ker L$ and $ker L = \{0\}$ so $x-y =0$ and $x=y$.
\\
There is injection and surjection proved so there exists a bijection and $L:V \rightarrow V$ is a automorphism. 

\end{enumerate}
\end{document}
