
\documentclass[12pt]{article}
\usepackage[finnish]{babel}
\usepackage[T1]{fontenc}
\usepackage[utf8]{inputenc}
\usepackage{delarray,amsmath,bbm,epsfig,slashed}
\newcommand{\pat}{\partial}
\newcommand{\be}{\begin{equation}}
\newcommand{\ee}{\end{equation}}
\newcommand{\bea}{\begin{eqnarray}}
\newcommand{\eea}{\end{eqnarray}}
\newcommand{\abf}{{\bf a}}
\newcommand{\Zmath}{\mathbf{Z}}
\newcommand{\Zcal}{{\cal Z}_{12}}
\newcommand{\zcal}{z_{12}}
\newcommand{\Acal}{{\cal A}}
\newcommand{\Fcal}{{\cal F}}
\newcommand{\Ucal}{{\cal U}}
\newcommand{\Vcal}{{\cal V}}
\newcommand{\Ocal}{{\cal O}}
\newcommand{\Rcal}{{\cal R}}
\newcommand{\Scal}{{\cal S}}
\newcommand{\Lcal}{{\cal L}}
\newcommand{\Hcal}{{\cal H}}
\newcommand{\hsf}{{\sf h}}
\newcommand{\half}{\frac{1}{2}}
\newcommand{\Xbar}{\bar{X}}
\newcommand{\xibar}{\bar{\xi }}
\newcommand{\barh}{\bar{h}}
\newcommand{\Ubar}{\bar{\cal U}}
\newcommand{\Vbar}{\bar{\cal V}}
\newcommand{\Fbar}{\bar{F}}
\newcommand{\zbar}{\bar{z}}
\newcommand{\wbar}{\bar{w}}
\newcommand{\zbarhat}{\hat{\bar{z}}}
\newcommand{\wbarhat}{\hat{\bar{w}}}
\newcommand{\wbartilde}{\tilde{\bar{w}}}
\newcommand{\barone}{\bar{1}}
\newcommand{\bartwo}{\bar{2}}
\newcommand{\nbyn}{N \times N}
\newcommand{\repres}{\leftrightarrow}
\newcommand{\Tr}{{\rm Tr}}
\newcommand{\tr}{{\rm tr}}
\newcommand{\ninfty}{N \rightarrow \infty}
\newcommand{\unitk}{{\bf 1}_k}
\newcommand{\unitm}{{\bf 1}}
\newcommand{\zerom}{{\bf 0}}
\newcommand{\unittwo}{{\bf 1}_2}
\newcommand{\holo}{{\cal U}}
%\newcommand{\bra}{\langle}
%\newcommand{\ket}{\rangle}
\newcommand{\muhat}{\hat{\mu}}
\newcommand{\nuhat}{\hat{\nu}}
\newcommand{\rhat}{\hat{r}}
\newcommand{\phat}{\hat{\phi}}
\newcommand{\that}{\hat{t}}
\newcommand{\shat}{\hat{s}}
\newcommand{\zhat}{\hat{z}}
\newcommand{\what}{\hat{w}}
\newcommand{\sgamma}{\sqrt{\gamma}}
\newcommand{\bfE}{{\bf E}}
\newcommand{\bfB}{{\bf B}}
\newcommand{\bfM}{{\bf M}}
\newcommand{\cl} {\cal l}
\newcommand{\ctilde}{\tilde{\chi}}
\newcommand{\ttilde}{\tilde{t}}
\newcommand{\ptilde}{\tilde{\phi}}
\newcommand{\utilde}{\tilde{u}}
\newcommand{\vtilde}{\tilde{v}}
\newcommand{\wtilde}{\tilde{w}}
\newcommand{\ztilde}{\tilde{z}}

% David Weir's macros


\newcommand{\nn}{\nonumber}
\newcommand{\com}[2]{\left[{#1},{#2}\right]}
\newcommand{\mrm}[1] {{\mathrm{#1}}}
\newcommand{\mbf}[1] {{\mathbf{#1}}}
\newcommand{\ave}[1]{\left\langle{#1}\right\rangle}
\newcommand{\halft}{{\textstyle \frac{1}{2}}}
\newcommand{\ie}{{\it i.e.\ }}
\newcommand{\eg}{{\it e.g.\ }}
\newcommand{\cf}{{\it cf.\ }}
\newcommand{\etal}{{\it et al.}}
\newcommand{\ket}[1]{\vert{#1}\rangle}
\newcommand{\bra}[1]{\langle{#1}\vert}
\newcommand{\bs}[1]{\boldsymbol{#1}}
\newcommand{\xv}{{\bs{x}}}
\newcommand{\yv}{{\bs{y}}}
\newcommand{\pv}{{\bs{p}}}
\newcommand{\kv}{{\bs{k}}}
\newcommand{\qv}{{\bs{q}}}
\newcommand{\bv}{{\bs{b}}}
\newcommand{\ev}{{\bs{e}}}
\newcommand{\gv}{\bs{\gamma}}
\newcommand{\lv}{{\bs{\ell}}}
\newcommand{\nabv}{{\bs{\nabla}}}
\newcommand{\sigv}{{\bs{\sigma}}}
\newcommand{\notvec}{\bs{0}_\perp}
\newcommand{\inv}[1]{\frac{1}{#1}}
%\newcommand{\xv}{{\bs{x}}}
%\newcommand{\yv}{{\bs{y}}}
\newcommand{\Av}{\bs{A}}
%\newcommand{\lv}{{\bs{\ell}}}
\newcommand{\Rmath}{\mbf{R}}

%\newcommand\bsigma{\vec{\sigma}}
\hoffset 0.5cm
\voffset -0.4cm
\evensidemargin -0.2in
\oddsidemargin -0.2in
\topmargin -0.2in
\textwidth 6.3in
\textheight 8.4in

\begin{document}

\normalsize

\baselineskip 14pt

\begin{center}
{\Large {\bf Quantum Information A \ \ Fall 2020 }} \\
{\Large Solutions to Problem Set 3} \\
Jake Muff

\end{center}


\begin{enumerate}

\item What is the Polar Decomposition of a positive matrix P? Of a Unitary Matrix U? Of a Hermitian Matrix H? \\

\begin{enumerate}
    \item A positive matrix, P,  is diagonalizable, meaning
    $$ P = \sum_i \lambda_i \ket{i} \bra{i} $$
    Where $\lambda_i \geq 0$. A matrix J can be viewed as 
    $$ J = \sqrt{P^\dagger P}  = \sqrt{PP} = \sqrt{P^2} $$
    $$ = \sum_i \sqrt{\lambda_i^2} \ket{i} \bra{i} = \sum_i \lambda_i \ket{i} \bra{i} =P $$
    So the polar decomposition is therefore 
    $$ P = UP ;\ \forall P $$ 
    Since U is a unitary matrix i.e $ U^\dagger U = I$ and $U=I$ then $P=P$ 
    \item For a unitary matrix U
    \\ 
    Lets say that U is decomposed such that $U = KJ$ where K is unitary and J is a positive operator such that $J= \sqrt{U^\dagger U} = \sqrt{I} = I$.\\
    Unitary matrices/operators are invertible such that $K = UJ^{-1} = U I^{-1} = UI = U$.
    \\
    The polar decomposition of U is then $U=U$ 
    \item For a Hermitian matrix H such that $H=UJ$ where $J= \sqrt{H^\dagger H} = \sqrt{HH} = \sqrt{H^2} $, which can be re-written as 
    $$ H = U\sqrt{H^2} $$ 
    From spectral decomposition we know that $H \neq \sqrt{H^2} $, the proof is as follows
    $$ H = \sum_i \lambda_i \ket{i} \bra{i} $$
    Where $\lambda \in \mathbbm{R}$ :
    $$ \sqrt{H^2} = \sqrt{\sum_i \lambda_i^2 \ket{i} \bra{i}} = \sum_i \sqrt{\lambda_i^2} \ket{i} \bra{i} = \sum_i | \lambda_i | \ket{i} \bra{i} \neq H$$
\end{enumerate}


\item Find the left and right Polar Decompositions of the matrix 
$$ A =  \left( \begin{array}{cc} 1 & 0 \\ 1 & 1\end{array} \right) $$
$$ A^\dagger A = \left( \begin{array}{cc} 1 & 1 \\ 0 & 1\end{array} \right)  \left( \begin{array}{cc} 1 & 0 \\ 1 & 1\end{array} \right) =  \left( \begin{array}{cc} 2 & 1 \\ 1 & 1\end{array} \right) $$
The left polar decomposition is 
$$ A = UJ $$
And the right polar decomposition 
$$ A = KU $$
Where $J$ and $K$ are positive operators. 
\\
To find the matrices for $K$ and $J$ we use spectral decomposition and notice that $U=AJ^{-1} $
\\
Lets find the eigenvalues for $A^\dagger A $:
$$ det(A^\dagger A - \lambda I ) = |  \left( \begin{array}{cc} 2- \lambda & 1 \\ 1 & 1- \lambda \end{array} \right) = 0 $$
$$ = (2- \lambda ) (1- \lambda) - 1 $$
$$ = 2- 2\lambda -\lambda + \lambda^2 -1 $$
$$ 1 - 3\lambda + \lambda^2 =0 $$
So $$\lambda_{\pm} = \frac{3 \pm \sqrt{5}}{2}$$
With associated eigenvectors 
$$ \ket{\lambda_+} = \left( \begin{array}{cc} \frac{\sqrt{5}+1}{2} \\ 1\end{array} \right)$$
$$ \ket{\lambda_-} = \left( \begin{array}{cc} -\frac{\sqrt{5}+1}{2} \\ 1\end{array} \right)$$
Now the spectral decomposition of $A^\dagger A$ is 
$$ A^\dagger A = \lambda_+ \ket{\lambda_+}\bra{\lambda_+} + \lambda_- \ket{\lambda_-}\bra{\lambda_-} $$
So when $J = \sqrt{A^\dagger A} $ the spectral decomposition is
$$ \sqrt{A^\dagger A} = \sqrt{\lambda_+} \ket{\lambda_+}\bra{\lambda_+} + \sqrt{\lambda_-} \ket{\lambda_-}\bra{\lambda_-} $$
Which works out to be 
$$ J = \left( \begin{array}{cc} 5.854 & 1.618 \\ 1.618 & 2.236 \end{array} \right) $$ 
Now $U = AJ^{-1} $, where 
$$J^{-1} = \left( \begin{array}{cc} 0.214 & -0.155 \\ -0.155 & 0.559 \end{array} \right) $$ 
So that 
$$ U = \left( \begin{array}{cc} 0.214 & -0.155 \\ 0.059 & 0.405 \end{array} \right) $$ 
$K$ is calculated from the spectral decomposition of $\sqrt{AA^\dagger}$  such that
$$ K = \left( \begin{array}{cc} 4.227 & 1.618 \\ 3.611 & 3.851 \end{array} \right) $$ 
The left polar decomposition is then 
$$ A = UJ = \left( \begin{array}{cc} 0.214 & -0.155 \\ 0.059 & 0.405 \end{array} \right) \left( \begin{array}{cc} 5.854 & 1.618 \\ 1.618 & 2.236 \end{array} \right) $$
And the right polar decomposition is 
$$ A = KU = \left( \begin{array}{cc} 4.227 & 1.618 \\ 3.611 & 3.851 \end{array} \right)\left( \begin{array}{cc} 0.214 & -0.155 \\ 0.059 & 0.405 \end{array} \right) $$
These values may be a little off as the calculator had some rounding error when it came to the surds and couldn't represent some decimals as fractions or surds. To keep thinks consistent I kept to 3 decimal places. 
\item Show that $\vec{v} \cdot \vec{\sigma} $ has eigenvalues $\pm 1$ and that the projectors onto the corresponding eigenspaces are given by $P_{\pm} = \frac{I \pm \vec{v} \cdot \vec{\sigma}} {2} $
\\
This question is similar to exercise 2.35 from the book. 
\\
We know $\vec{v} \cdot \vec{\sigma}$ is hermitian and that $(\vec{v} \cdot \vec{\sigma})^2 = I$. So we can say
$$ (\vec{v} \cdot \vec{\sigma})^2 \ket{\lambda} = I \ket{\lambda} = \lambda $$
Where $\ket{\lambda}$ is the eigenvector with $\lambda$ eigenvalue. Therefore we have 
$$ \lambda^2 \ket{\lambda} = \ket{\lambda} $$ 
Thus $\lambda^2 =1 ; \ \lambda = \pm 1$
Or using the determinant. 
$$\vec{v} \cdot \vec{\sigma} = \sum_i^3 v_i \sigma_i $$
$$ = \left( \begin{array}{cc} v_3 & v_1 -iv_2 \\ v_1 + iv_2 & -v_3 \end{array} \right)$$
Computing the determinant 
$$| \left( \begin{array}{cc} v_3-\lambda & v_1 -iv_2 \\ v_1 + iv_2 & -v_3-\lambda \end{array} \right)|$$
$$ = \lambda^2 -1 =0 $$
Note that $|\vec{v} | =1 $.
\\ 
So $\lambda = \pm 1$.
\\
Now we know from eq 2.35 in the book that the projector is equal to $P = \sum_i \ket{i} \bra{i} $. To answer the second part of the question we can show that the outer products of the eigenvectors will be what we're looking for. However, I will also provide a proof the projector is related to this outer product, thus tying them together to show that $P_{\pm} = \frac{I \pm \vec{v} \cdot \vec{\sigma}} {2} $.
\\
For $\lambda = 1$ the outer product is 
$$ \ket{\lambda_1}\bra{\lambda_1} = \frac{1+v_3}{2} \left( \begin{array}{cc} 1  \\ \frac{1-v_3}{v_1 -iv_2} \end{array} \right)\left( \begin{array}{cc} 1 & \frac{1-v_3}{v_1 -iv_2} \end{array} \right) $$
$$ = \frac{1}{2} \Big(I +  \left( \begin{array}{cc} v_3 & v_1-iv_2  \\ v_1+iv_2 & -v_3 \end{array} \right)\Big) $$
$$ = \frac{I+ \vec{v}\cdot \vec{\sigma}}{2} $$
For $\lambda = -1$ 
$$ \ket{\lambda_{-1}}\bra{\lambda_{-1}} = \frac{1-v_3}{2} \left( \begin{array}{cc} 1  \\ -\frac{1+v_3}{v_1 -iv_2} \end{array} \right)\left( \begin{array}{cc} 1 & -\frac{1+v_3}{v_1 -iv_2} \end{array} \right) $$
$$ = \frac{1}{2} \Big(I -  \left( \begin{array}{cc} v_3 & v_1-iv_2  \\ v_1+iv_2 & -v_3 \end{array} \right)\Big) $$
$$ = \frac{I- \vec{v}\cdot \vec{\sigma}}{2} $$
Now we need to prove that 
$$ P_{\pm} = \ket{\lambda_{\pm 1}} \bra{\lambda_{\pm 1}} $$ 
And thus 
$$ P_{\pm} = \frac{I\pm \vec{v}\cdot \vec{\sigma}}{2} $$
Suppose that $\ket{\phi} \in \mathbbm{C}^2$ such that
$$ \bra{\phi} (P_{\pm} - \ket{\lambda_{\pm 1}} \bra{\lambda_{\pm 1}}) \ket{\phi} = 0 $$
As $\vec{v} \cdot \vec{\sigma}$ is hermitian (proven previously) thus the eigenvectors are orthonormal, $\ket{\phi}$ can be written as a linear combination 
$$ \ket{\phi} = A \ket{\lambda_{\pm 1}} + B \ket{\lambda_{\pm 1}} $$
Where $|A|^2 + |B|^2 =1 $ and theyre both in $\mathbbm{C}$ (Seen this before) we can write 
$$ \bra{\phi} (P_{\pm} - \ket{\lambda_{\pm}} \bra{\lambda_{\pm}}) \ket{\phi} $$
$$ = \bra{\phi} P_{\pm} \ket{\phi} - \langle \phi | \lambda_{\pm} \rangle \langle \lambda_{\pm} | \phi \rangle $$
So 
$$ \bra{\phi} P_{\pm} \ket{\phi} = \bra{\phi} \frac{1}{2} ( I \pm \vec{v} \cdot \vec{\sigma}) \ket{\phi} $$
$$ = \frac{1}{2} \pm \frac{1}{2} \bra{\phi} \vec{v} \cdot \vec{\sigma} \ket{\phi} $$
$$ = \frac{1}{2} \pm \frac{1}{2} (|A|^2 - |B|^2 ) = \frac{1}{2} \pm \frac{1}{2} (2|A|^2 - 1) $$
Where subsituting the previous values $\langle \phi | \lambda_1 \rangle \langle \lambda_1 | \phi \rangle = |A|^2 $ and $\langle \phi | \lambda_{-1} \rangle \langle \lambda_{-1} | \phi \rangle = |B|^2 $
\\
So we have that 
$$ \langle \phi | P_{\pm} - \ket{\lambda_{\pm 1}} \bra{\lambda_{\pm 1}} | \phi \rangle  =0 $$
And 
$$ P_{\pm} = \ket{\lambda_{\pm 1}} \bra{\lambda_{\pm 1}}.$$
So that the projector is equal to the outer product of the eigenvectors. 

\item Calculate the probability of obtaining the result $+1$ for a measurement of $\vec{v} \cdot \vec{\sigma}$, given that the state prior to measurement is $\ket{0}$. What is the state of the system after measurement if $+1$ is obtained? 
\\
Probability of $+1$ state after measurement 
$$ P(+1 | \ket{0})$$
$+1$ state corresponds to $\lambda_1$ therefore we have 
$$ P(\langle \lambda_1  \ket{0}) = \langle \lambda_1 | 0 \rangle \langle 0 | \lambda_1 \rangle $$ 
$$ = \langle 0 | \lambda_1 \rangle \langle \lambda_1 | 0 \rangle $$
$$ = \langle 0 | \frac{1}{2} ( I + \vec{v} \cdot \vec{\sigma}) | 0\rangle $$
$$ = \frac{1}{2} (1 + v_3) $$
Given that outcome $m$ occured, the state after measurement is 
$$ \frac{P_m \ket{\psi}}{\sqrt{P(m)}} $$
So we get 
$$ \frac{\ket{\lambda_1} \langle \lambda_1 | 0 \rangle}{\sqrt{\langle 0 | \lambda_1 \rangle \langle \lambda_1 | 0 \rangle}} $$
$$ = \frac{1}{\sqrt{\frac{1}{2}(1+v_3)}} \cdot \frac{1}{2} \left( \begin{array}{cc} 1+v_3 \\ v_1+iv_2 \end{array} \right)$$
$$ = \sqrt{\frac{1+v_3}{2}} \left( \begin{array}{cc} 1 \\ \frac{1-v_3}{v_1 -iv_2} \end{array} \right)$$
$$ = \ket{\lambda_1} $$ 
From the previous exercise. 
\item If you have an orthonormal basis ${e_1,\ldots ,e_n}$ of a vector space $V$ chosen so that the first $1\leq k<n$ vectors are a basis of a $k$-dimensional subspace $W$,
a projection operator $P$ that projects to W is simply
$$
P = \sum^k_{i=1} e_i e^\dagger_i \ .  
$$
(In ket notation with $e_i = \ket{i}$, $P = \sum^k_{i=1} \ket{i}\bra{i}$.)  What if you have a basis which is not even orthogonal? Consider the vectors
$$
u_1 = \left( \begin{array}{c} 1 \\ 0 \\ 0 \end{array} \right) \ ;  \  u_2 = \left( \begin{array}{c} 1 \\ 1 \\ 0 \end{array} \right) 
$$
spanning a two-dimensional subspace $W$ (the xy-plane) of $\Rmath^3$. Note that $u_1,u_2$ are not orthogonal. In this case one can construct a projection operator $P$ which
projects to $W$ as follows. Construct a $3\times 2$ matrix
$$
A = [u_1 u_2] \ , 
$$ 
the notation means that the two vectors $u_1,u_2$ are the columns of $A$. Then 
$$
P = A (A^TA)^{-1} A^T
$$
is a projection operator to $W$. Verify this:  show that in general the above $P^2=P$, and by using the given $u_1,u_2$ calculate the matrix $P$ explicitly and verify that
it projects to $W$ by showing that the vector $Pv$ with an arbitrary 
$$
v= \left( \begin{array}{c} v_x \\ v_y \\ v_z \end{array} \right) 
$$
is in $W$.  Next, show that the matrix $G\equiv A^TA$ is in fact a Gram matrix with $G_{ij}=u_i\cdot u_j$.
\\
\underline{\emph{\bf{Answer}}}: 
\\
$$ A = \left( \begin{array}{ccc} 1 & 1 \\ 0 & 1 \\ 0 & 0 \end{array} \right)$$
$$ P = A(A^T A)^{-1} A^T $$
In general (for any $P$), P satisfies 
$$ P^2 = P $$ 
Because 
$$P = \sum_i \ket{i} \bra{i} $$ 
$$ P^2 = (\sum_i \ket{i} \bra{i}) (\sum_j \ket{j} \bra{j} )$$
$$ = \sum_{i,j} \ket{i} \bra{i} \ket{j} \bra{j} $$
$$ = \sum_{i,j} \ket{i} \bra{j} \delta_{i,j} $$
$$ = \sum_i \ket{i} \bra{i} = P $$
Calculating the matrix $P$ explicitly 
$$ P = \left( \begin{array}{ccc} 1 & 1 \\ 0 & 1 \\ 0 & 0 \end{array} \right) \Bigg[ \left( \begin{array}{ccc} 1 & 0 & 0 \\ 1 & 1 & 0 \end{array} \right)\left( \begin{array}{ccc} 1 & 1 \\ 0 & 1 \\ 0 & 0 \end{array} \right)\Bigg]^{-1} \left( \begin{array}{ccc} 1 & 0 & 0 \\ 1 & 1 & 0 \end{array} \right) $$
$$ = \left( \begin{array}{ccc} 1 & 1 \\ 0 & 1 \\ 0 & 0 \end{array} \right) \left( \begin{array}{ccc} 1 & 1 \\ 1 & 2 \end{array} \right)^{-1} \left( \begin{array}{ccc} 1 & 0 & 0 \\ 1 & 1 & 0 \end{array} \right) $$
$$ = \left( \begin{array}{ccc} 1 & 0 & 0 \\ 0 & 1 & 0 \\ 0 & 0 & 0 \end{array} \right)$$
Vector $Pv$ with 
$$ v = \left( \begin{array}{ccc} v_x \\ v_y \\ v_z \end{array} \right)$$
Such that 
$$ Pv = \left( \begin{array}{ccc} v_x \\ v_y \\ 0 \end{array} \right)$$
$v_2$ spans the 2D subspace $W$ of $\mathbbm{R}^3$ so $Pv$ is in $W$
$$ G = A^T A $$
$$ G = \left( \begin{array}{ccc} 1 & 0 & 0 \\ 1 & 1 & 0 \end{array} \right) \left( \begin{array}{ccc} 1 & 1 \\ 0 & 1 \\ 0 & 0 \end{array} \right)$$
$$ = \left( \begin{array}{ccc} 1 & 1 \\ 1 & 2 \end{array} \right) $$
$$ G_{ij} = u_i \cdot u_j $$
$$ G_{ij} = u_i^\dagger u_j $$
$$ =  \left( \begin{array}{ccc} 1 & 1 \\ 0 & 1 \\ 0 & 0 \end{array} \right)^\dagger  \left( \begin{array}{ccc} 1 & 1 \\ 0 & 1 \\ 0 & 0 \end{array} \right) $$
$$ = \left( \begin{array}{ccc} 1 & 1 \\ 1 & 2 \end{array} \right) $$
With inner product $\langle u_i , u_j \rangle$ so that $G_{12} = 1$ verifies it. 


\item Suppose Bob is given a quantum state chosen from a set $\ket{\psi_1} \ldots \ket{\psi_m}$ of linearly independent states. Construct a POVM $\{E_1, E_2, \ldots , E_{m+1}\}$ such that if outcome $E_i$ occurs, $1 \leq i \leq m$, then Bob knows with certainty that he was given the state $\ket{\psi_i}$. The POVM must be such that $\langle \psi_i | E_i | \psi_i \rangle > 0$ for each $i$. 
\\
\\
To construct the POVM we want that $\langle \psi_i | E_j | \psi_i \rangle = 0$ for every $i \neq j$ and we have a condition that $\langle \psi_i | E_i | \psi_i \rangle > 0$. We're looking to find $\langle u_j | \phi_i \rangle = 0 \ i \neq j$ for such $E_j = a \ket{u_j}\bra{u_j}$, which as we know can be rewritten in the form of a projector as $E_j = a P_j$ where $P_j$ is the projector onto an orthogonal complement $U_j$ of the set of linearly independent states. 
For all $j$ from $j = 1, \ldots , m$ we know that the sum $\sum_j E_j = I$, which is called the completeness condition. Pulling these two facts together we can construct $E_{m+1}$ 
$$ E_{m+1} = I - a \sum_{j=1}^m P_j = I - a \sum_{j=1}^m \ket{j} \bra{j}  = I - \sum_{j=1}^m E_j$$ 
As the states are linearly independent the trace must be greater than 0, $Tr(P_j \psi_j) > 0$ and that $E_{m+1}$ is positive. 
\\
Our POVM elements are then 
$$ E_j = a P_j = a \ket{j}\bra{j} = \frac{1}{m} P_j $$
$$ E_{m+1} = I - a \sum_{j=1} \ket{j} \bra{j} = I - \sum_{j=1}^m E_j $$
\\
If we take $a = 1/m$  we can prove that this satisifies the condition that $\langle \psi_i | E_i | \psi_i \rangle > 0$ 
$$ \langle \psi_i | E_{m+1} | \psi_i \rangle = \langle \psi_i | (I - a \sum_{j=1}^m \ket{u_j}\bra{u_j}) | \psi_i \rangle $$
$$ = \langle \psi_i | \psi_i \rangle - \frac{1}{m} \sum_{j=1}^m \langle \psi_i | u_j \rangle \langle u_j | \psi_i \rangle $$ 
$$ \geq 1 - \frac{1}{m} \sum_{j=1}^m 1 $$
$$ \geq 1- \frac{1}{m} (m \times 1) = 0 $$
Therefore $ E_{m+1} \geq 0$ 


\end{enumerate}


\end{document}

