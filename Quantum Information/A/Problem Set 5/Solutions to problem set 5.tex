
\documentclass[12pt]{article}
\usepackage[finnish]{babel}
\usepackage[T1]{fontenc}
\usepackage[utf8]{inputenc}
\usepackage{delarray,amsmath,bbm,epsfig,slashed}
\usepackage{bbold}
\usepackage{listings}
\newcommand{\pat}{\partial}
\newcommand{\be}{\begin{equation}}
\newcommand{\ee}{\end{equation}}
\newcommand{\bea}{\begin{eqnarray}}
\newcommand{\eea}{\end{eqnarray}}
\newcommand{\abf}{{\bf a}}
\newcommand{\Zmath}{\mathbf{Z}}
\newcommand{\Zcal}{{\cal Z}_{12}}
\newcommand{\zcal}{z_{12}}
\newcommand{\Acal}{{\cal A}}
\newcommand{\Fcal}{{\cal F}}
\newcommand{\Ucal}{{\cal U}}
\newcommand{\Vcal}{{\cal V}}
\newcommand{\Ocal}{{\cal O}}
\newcommand{\Rcal}{{\cal R}}
\newcommand{\Scal}{{\cal S}}
\newcommand{\Lcal}{{\cal L}}
\newcommand{\Hcal}{{\cal H}}
\newcommand{\hsf}{{\sf h}}
\newcommand{\half}{\frac{1}{2}}
\newcommand{\Xbar}{\bar{X}}
\newcommand{\xibar}{\bar{\xi }}
\newcommand{\barh}{\bar{h}}
\newcommand{\Ubar}{\bar{\cal U}}
\newcommand{\Vbar}{\bar{\cal V}}
\newcommand{\Fbar}{\bar{F}}
\newcommand{\zbar}{\bar{z}}
\newcommand{\wbar}{\bar{w}}
\newcommand{\zbarhat}{\hat{\bar{z}}}
\newcommand{\wbarhat}{\hat{\bar{w}}}
\newcommand{\wbartilde}{\tilde{\bar{w}}}
\newcommand{\barone}{\bar{1}}
\newcommand{\bartwo}{\bar{2}}
\newcommand{\nbyn}{N \times N}
\newcommand{\repres}{\leftrightarrow}
\newcommand{\Tr}{{\rm Tr}}
\newcommand{\tr}{{\rm tr}}
\newcommand{\ninfty}{N \rightarrow \infty}
\newcommand{\unitk}{{\bf 1}_k}
\newcommand{\unitm}{{\bf 1}}
\newcommand{\zerom}{{\bf 0}}
\newcommand{\unittwo}{{\bf 1}_2}
\newcommand{\holo}{{\cal U}}
%\newcommand{\bra}{\langle}
%\newcommand{\ket}{\rangle}
\newcommand{\muhat}{\hat{\mu}}
\newcommand{\nuhat}{\hat{\nu}}
\newcommand{\rhat}{\hat{r}}
\newcommand{\phat}{\hat{\phi}}
\newcommand{\that}{\hat{t}}
\newcommand{\shat}{\hat{s}}
\newcommand{\zhat}{\hat{z}}
\newcommand{\what}{\hat{w}}
\newcommand{\sgamma}{\sqrt{\gamma}}
\newcommand{\bfE}{{\bf E}}
\newcommand{\bfB}{{\bf B}}
\newcommand{\bfM}{{\bf M}}
\newcommand{\cl} {\cal l}
\newcommand{\ctilde}{\tilde{\chi}}
\newcommand{\ttilde}{\tilde{t}}
\newcommand{\ptilde}{\tilde{\phi}}
\newcommand{\utilde}{\tilde{u}}
\newcommand{\vtilde}{\tilde{v}}
\newcommand{\wtilde}{\tilde{w}}
\newcommand{\ztilde}{\tilde{z}}

% David Weir's macros


\newcommand{\nn}{\nonumber}
\newcommand{\com}[2]{\left[{#1},{#2}\right]}
\newcommand{\mrm}[1] {{\mathrm{#1}}}
\newcommand{\mbf}[1] {{\mathbf{#1}}}
\newcommand{\ave}[1]{\left\langle{#1}\right\rangle}
\newcommand{\halft}{{\textstyle \frac{1}{2}}}
\newcommand{\ie}{{\it i.e.\ }}
\newcommand{\eg}{{\it e.g.\ }}
\newcommand{\cf}{{\it cf.\ }}
\newcommand{\etal}{{\it et al.}}
\newcommand{\ket}[1]{\vert{#1}\rangle}
\newcommand{\bra}[1]{\langle{#1}\vert}
\newcommand{\bs}[1]{\boldsymbol{#1}}
\newcommand{\xv}{{\bs{x}}}
\newcommand{\yv}{{\bs{y}}}
\newcommand{\pv}{{\bs{p}}}
\newcommand{\kv}{{\bs{k}}}
\newcommand{\qv}{{\bs{q}}}
\newcommand{\bv}{{\bs{b}}}
\newcommand{\ev}{{\bs{e}}}
\newcommand{\gv}{\bs{\gamma}}
\newcommand{\lv}{{\bs{\ell}}}
\newcommand{\nabv}{{\bs{\nabla}}}
\newcommand{\sigv}{{\bs{\sigma}}}
\newcommand{\notvec}{\bs{0}_\perp}
\newcommand{\inv}[1]{\frac{1}{#1}}
%\newcommand{\xv}{{\bs{x}}}
%\newcommand{\yv}{{\bs{y}}}
\newcommand{\Av}{\bs{A}}
%\newcommand{\lv}{{\bs{\ell}}}
\newcommand{\Rmath}{\mbf{R}}

%\newcommand\bsigma{\vec{\sigma}}
\hoffset 0.5cm
\voffset -0.4cm
\evensidemargin -0.2in
\oddsidemargin -0.2in
\topmargin -0.2in
\textwidth 6.3in
\textheight 8.4in

\begin{document}

\normalsize

\baselineskip 14pt

\begin{center}
{\Large {\bf Quantum Information A \ \ Fall 2020 \ \  Solutions to Problem Set 5}} \\
Jake Muff \\
04/10/20
\end{center}

\bigskip
\section{Answers}



\begin{enumerate}
    \item Exercise 3.13 from Nielsen and Chaung. Show that $c^n$ is $\Omega (n^{log n})$ for any $c > 1$ but that $n^{log n} $ is never $\Omega(c^n)$. We need to show that as in the book $f(n)$ is $\Omega (g(n)) $ if $c g(n) \leq f(n)$ for all $n \geq n_0$, so 
    $$ c n^{log n} \leq c^n $$ 
    For this I used a very rudimentary and brute force method python script to check 
    \begin{lstlisting}[language=Python]
        import numpy as np

        n = 100000
        c = 2
        count =0

        #r is right hand side
        #l is left hand side
        for c in range(0,1000):
            l = c * (n**np.log(n))
            r = c**n 
            if l <= r:
                print('c = %d, YES'%(c))
                count +=1
            else:
                print('c = %d, NO'%(c))
        print(count)
        #Outputs YES for all values c > 1
    \end{lstlisting}
    And to prove that $n^{log n}$ is never $\Omega (c^n)$ 
    \begin{lstlisting}[language=Python]
        import numpy as np

        n = 100000
        c = 2
        count =0

        #r is right hand side
        #l is left hand side
        for c in range(0,1000):
            l = c * c**n
            r = n**(np.log(n)) 
            if l <= r:
                print('c = %d, YES'%(c))
                count +=1
            else:
                print('c = %d, NO'%(c))
        print(count)
        #Outputs NO for all values c > 1
    \end{lstlisting}
    I know this can be shown graphically I just didn't really know how to represent the processing time with the function to show that the lower bound is $\Omega (n^{log n})$ 

    \item Exercise 4.4 from Nielsen-Chaung. Express the Hadamard gate H as a product of $R_x$ and $R_z$ rotation and $e^{i \psi}$ for some $\psi$. 
    \\
    If $\hat{n} = (n_x, n_y, n_z) \in \mathbb{R}^3$ such that 
    $$ R_{\hat{n}} (\theta) = \exp (-i \theta \hat{n} \cdot \vec{\sigma}/2) $$
    $$ = cos(\theta /2)\textbf{I} - isin(\theta /2 )(n_x \textbf{X} + n_y \textbf{Y} + n_z \textbf{Z}) $$
    Where $\vec{\sigma}$ are the Pauli Matrices with associated gates. 
    $$  \textbf{X} = \left(\begin{array}{cc} 0 & 1 \\  1 & 0\end{array}\right), \textbf{Y} = \left(\begin{array}{cc} 0 & -i \\  i & 0\end{array}\right)$$
    $$ \textbf{Z} = \left(\begin{array}{cc} 1 & 0 \\  0 & -1\end{array}\right), \textbf{I} = \left(\begin{array}{cc} 1 & 0 \\  0 & 1\end{array}\right) $$
    The Pauli matrices have eigenvectors which are 
    $$ \ket{\textbf{X}} = \frac{1}{\sqrt{2}} \left(\begin{array}{cc} 1 \\  1\end{array}\right), \frac{1}{\sqrt{2}} \left(\begin{array}{cc} -1 \\  1\end{array}\right)$$
    $$ \ket{\textbf{Y}} = \frac{1}{\sqrt{2}} \left(\begin{array}{cc} 1 \\  i\end{array}\right), \frac{1}{\sqrt{2}} \left(\begin{array}{cc} i \\  1\end{array}\right)$$
    $$ \ket{\textbf{Z}} = \left(\begin{array}{cc} 1 \\  0\end{array}\right), \left(\begin{array}{cc} 0 \\  1\end{array}\right)$$
    Where the eigenvectors are the different poles of the bloch sphere. The Hadamard gate has the effect on the Pauli gates of 
    $$ \ket{\textbf{Z}; +} \rightarrow \ket{\textbf{X}; +} $$
    $$ \ket{\textbf{Z}; -} \rightarrow \ket{\textbf{X}; -} $$
    $$ \ket{\textbf{Y}; +} \rightarrow \ket{\textbf{Y}; -} $$
    In terms of the bloch sphere or geometrically these can be seen as rotations
    $$ H = R_x (\pi) R_y(\frac{\pi}{2}) e^{i \psi} $$
    Where 
    $$ R_x (\pi) = cos(\frac{\pi}{2})\textbf{I} - isin(\frac{\pi}{2})(n_x \textbf{X} + n_y \textbf{Y} + n_z \textbf{Z}) $$
    $$ = \left(\begin{array}{cc} cos(\frac{\pi}{2}) & -isin(\frac{\pi}{2}) \\  -isin(\frac{\pi}{2}) & cos(\frac{\pi}{2})\end{array}\right) $$
    $$ R_y(\frac{\pi}{2}) = cos(\frac{\pi}{4}) \textbf{I} - isin(\frac{\pi}{4})(n_y \textbf{Y}) $$
    $$ = \left(\begin{array}{cc} cos(\frac{\pi}{4}) & -isin(\frac{\pi}{4}) \\  -isin(\frac{\pi}{4}) & cos(\frac{\pi}{4})\end{array}\right) $$
    So 
    $$ R_x (\pi) R_y (\frac{\pi}{2}) = \left(\begin{array}{cc} -i \frac{\sqrt{2}}{2} & -i \frac{\sqrt{2}}{2} \\  -i \frac{\sqrt{2}}{2} & i \frac{\sqrt{2}}{2}\end{array}\right) $$
    $$ = -i \frac{1}{\sqrt{2}} \left(\begin{array}{cc} 1 & 1 \\  1 & -1\end{array}\right) $$
    Where $e^{i \psi} = e^{-i 3 \pi /2} $


    \item Exercise 4.9 from Nielsen-Chaung. Explain why any single qubit unitary operator may be written in the form 
    $$ U = e^{i(\alpha - \beta /2 - \delta /2)} cos(\frac{\gamma}{2}) $$
    \\
    A Unitary operator is such that 
    $$ U U^\dagger = U^\dagger U = \textbf{I} $$ 
    If we have 
    $$ U =  \left(\begin{array}{cc} a & b \\  c & d\end{array}\right)$$
    And 
    $$ U^\dagger = \left(\begin{array}{cc} a^* & b^* \\  c^* & d^*\end{array}\right)$$
    Then 
    $$ U^\dagger U = \left(\begin{array}{cc} aa^* + bb^* & (a \vec{i}+ b \vec{j}) \cdot (c \vec{i} + d \vec{j})  \\ (c \vec{i}+ d \vec{j}) \cdot (a \vec{i} + b \vec{j})   & cc^* + dd^*\end{array}\right)$$
    $$ =  \left(\begin{array}{cc} 1 & 0 \\  0 & 1\end{array}\right) = \textbf{I} $$
    So it can be shown that 
    $$ aa^* + bb^* = e^0 sin^2 (\frac{\gamma}{2}) + e^0 cos^2 (\frac{\gamma}{2}) = 1$$
    With this and the fact that any single qubit operator will have the form 
    $$ U = e^{ia} R_{\hat{n}} (\theta) $$
    Expanded gives 
    $$ U = e^{ia}  \left(\begin{array}{cc} cos(\frac{\theta}{2})-isin(\frac{\theta}{2})(n_x + n_z) & -n_y sin(\frac{\theta}{2})  \\  n_y sin(\frac{\theta}{2}) & cos(\frac{\theta}{2}) -isin(\frac{\theta}{2})(n_x - n_z)\end{array}\right)$$
    We can show that this equals eq 4.12 by making substitutions such as $a = \alpha, \theta = \gamma /2, n_y = cos(\beta /2 - \delta /2 )$

    \item Exercise 4.15 from Nielsen-Chaung. 
    \begin{equation}
        c_{12} = c_1 c_2 - s_1 s_2 \hat{n}_1 \cdot \hat{n}_2 
    \end{equation}
    \begin{equation}
        s_{12}\hat{n}_{12} = s_1 c_2 \hat{n}_1 + c_1 s_2 \hat{n}_2 + s_1 s_2 \hat{n}_2 \times \hat{n}_1 
    \end{equation}
    We have a rotation around $\hat{n}_1$ through $\theta = \beta_1$ \emph{then} a rotation around $\hat{n}_2$ through $\theta = \beta_2$. Need to find the rotation around $\hat{n}_{12}$ through $\beta_{12}$ or the equations (1) and (2). The book makes the substitution that $c_i = cos(\beta_i /2), s_i = sin(\beta_i /2), c_{12} = cos(\beta_{12}/2), s_{12} = sin(\beta_{12} /2)$ 
    An important indentity is needed which expands out a double dot product in terms of the cross product. 
    \begin{equation} \label{iden}
         (\vec{a} \cdot \vec{\sigma}) \cdot (\vec{b} \cdot \vec{\sigma}) = \vec{a} \cdot \vec{b} + i(\vec{a} \times \vec{b}) \cdot \vec{\sigma} 
    \end{equation}
    We have 
    $$ U_{\hat{n}_2} (\beta_1) U_{\hat{n}_1}(\beta_1) = e^{-i \frac{\beta_2}{2} \hat{n}_2 \cdot \vec{\sigma}} e^{-i \frac{\beta_1}{2} \hat{n}_1 \cdot \vec{\sigma}} $$
    $$ = \Big[ cos(\frac{\beta_2}{2}) - i(\hat{n}_2 \cdot \vec{\sigma}) sin(\frac{\beta_2}{2})\Big]\Big[cos(\frac{\beta_1}{2}) -i(\hat{n}_1 \cdot \vec{\sigma})sin(\frac{\beta_1}{2})\Big] $$
    $$ = cos(\frac{\beta_1}{2}) cos(\frac{\beta_2}{2}) - sin(\frac{\beta_1}{2})sin(\frac{\beta_2}{2}) (\hat{n}_1 \cdot \vec{\sigma})(\hat{n}_2 \cdot \vec{\sigma}) $$
    $$ -i \Big[ cos(\frac{\beta_1}{2}) sin(\frac{\beta_2}{2})(\hat{n}_2 \cdot \vec{\sigma}) + sin(\frac{\beta_1}{2}) cos(\frac{\beta_2}{2}) (\hat{n}_1 \cdot \vec{\sigma})\Big] $$
    Using the identity in equation (\ref{iden}) 
    $$ U_{\hat{n}_2} (\beta_2) U_{\hat{n}_1}(\beta_1) = cos(\frac{\beta_1}{2}) cos(\frac{\beta_2}{2}) -sin(\frac{\beta_1}{2})sin(\frac{\beta_2}{2}) \hat{n}_1 \cdot \hat{n}_2 $$
    $$ -i \Big[ sin(\frac{\beta_1}{2})cos(\frac{\beta_2}{2}) \hat{n}_1 + cos(\frac{\beta_1}{2})sin(\frac{\beta_2}{2}) \hat{n}_2 - sin(\frac{\beta_1}{2})sin(\frac{\beta_2}{2}) \hat{n}_1 \times \hat{n}_2 \Big] \cdot \vec{\sigma} $$
    The overall rotation is simply just 
    $$ U_{\hat{n}_{12}}(\beta_{12}) = e^{-i \frac{\beta_{12}}{2} \hat{n}_{12} \cdot \vec{\sigma}} $$
    $$ = cos (\frac{\beta_{12}}{2}) - isin(\frac{\beta_{12}}{2}) \hat{n}_{12} \cdot \vec{\sigma} $$
    Now we set $U_{\hat{n}_2} (\beta_2) U_{\hat{n}_1} (\beta_1) = U_{\hat{n}_{12}} (\beta_{12}) $. Which gives us
    $$ cos(\frac{\beta_{12}}{2}) = cos(\frac{\beta_{1}}{2})cos(\frac{\beta_2}{2})-sin(\frac{\beta_1}{2})sin(\frac{\beta_2}{2}) \hat{n}_1 \cdot \hat{n}_2 $$
    $$ sin(\frac{\beta_{12}}{2}) \hat{n}_{12} = sin(\frac{\beta_1}{2}) cos (\frac{\beta_2}{2}) \hat{n}_1 + cos(\frac{\beta_1}{2})sin(\frac{\beta_2}{2}) \hat{n}_2 + sin(\frac{\beta_1}{2})sin(\frac{\beta_2}{2}) \hat{n}_1 \times \hat{n}_2 $$
    Which is equivalent to equations (1) and (2). 
    \\
    For the second part we can simply set $\beta_1 = \beta_2 = \beta$ and $\hat{n}_1 = \hat{z}$ and $\hat{n}_2 = \hat{n}_2$ to subsitute and expand out so we get 
    $$ cos (\frac{\beta_{12}}{2}) = cos^2(\beta) - sin^2 (\beta) \hat{z} \cdot \hat{n}_2 $$
    $$ sin (\frac{\beta_{12}}{2}) \hat{n}_{12} = sin(\beta) cos(\beta) (\hat{z} + \hat{n}_2 ) - sin^2 (\beta) \hat{n}_2 \times \hat{z} $$
    
    \item Exercise 4.20 from Nielsen-Chaung. 
    $$ \ket{\pm} \equiv (\ket{0} \pm \ket{1}) / \sqrt{2} $$
    $CNOT $ with 2nd qubit as control and 1st qubit as the target can be labelled as $CNOT_{2,1}$. A $CNOT$ with the 1st qubit as control and the 2nd qubit as the target can be labelled $CNOT_{1,2}$. The LHS of the diagram shows 
    $$ (H \otimes H) (\ket{0} \bra{0} \otimes \textbf{I} + \ket{1} \bra{1} \otimes \textbf{X}) (H \otimes H) $$
    With 
    $$ H \otimes H = \frac{1}{\sqrt{2}} \left(\begin{array}{cc} H & H \\  H & -H\end{array}\right) =\frac{1}{\sqrt{2}} \left(\begin{array}{cc} 1 & 1 \\  1 & -1\end{array}\right) $$
    $\ket{0}\bra{0} \otimes \textbf{I} + \ket{1}\bra{1} \otimes \textbf{X}$ has matrix representation 
    $$ =  \left(\begin{array}{cc} \textbf{I} & 0 \\  0 & \textbf{X} \end{array}\right)$$
    So we have 
    $$ \frac{1}{\sqrt{2}} \left(\begin{array}{cc} H & H \\  H & -H\end{array}\right)\left(\begin{array}{cc} \textbf{I} & 0 \\  0 & \textbf{X} \end{array}\right)\frac{1}{\sqrt{2}} \left(\begin{array}{cc} H & H \\  H & -H\end{array}\right) $$
    $$ = \frac{1}{2} \left(\begin{array}{cccc} 1 & 1 & 1 & 1 \\  1 & -1 & 1 & -1 \\ 1 & 1 & -1 & -1 \\ 1 & -1 & -1 & 1\end{array}\right) $$
    $$ =  \left(\begin{array}{cccc} 1 & 0 & 0 & 0 \\  0 & 0 & 0 & 1 \\ 0 & 0 & 1 & 0 \\ 0 & 1 & 0 & 0\end{array}\right) $$
    Which is $CNOT_{2,1}$. 
    \\ 
    As we know from the previous exercises 
    $$ H \ket{0} = \ket{+}, H \ket{0} = \ket{-} $$
    And 
    $$ H^2 = \textbf{I} $$
    So 
    $$ CNOT_{1,2} \ket{++} = (H \otimes H)^2 CNOT_{1,2} (H \otimes H) \ket{00} $$
    $$ = (H \otimes H) CNOT_{2,1} \ket{00} $$
    $$ = (H \otimes H) \ket{00} = \ket{++} $$
    $$ CNOT_{1,2} \ket{-+} = (H \otimes H)^2 CNOT_{1,2} (H \otimes H) \ket{10} $$ 
    $$ = (H \otimes H) CNOT_{2,1} \ket{10} $$
    $$ = (H \otimes H) \ket{10} = \ket{-+} $$
    $$ CNOT_{1,2} \ket{+-} = (H \otimes H)^2 CNOT_{1,2} (H \otimes H) \ket{01} $$
    $$ = (H \otimes H) CNOT_{2,1} \ket{01} $$
    $$ (H \otimes H) \ket{11} = \ket{--} $$
    $$ CNOT_{1,2} \ket{--} = (H \otimes H)^2 CNOT{1,2} (H \otimes H) \ket{11} $$ 
    $$ = (H \otimes H) CNOT_{2,1} \ket{11} $$
    $$ = (H \otimes H) \ket{01} = \ket{+-} $$

    \item Exercise 4.23 from Nielsen-Chuang. Constructing the $C^1(U)$ gate for $U = R_x (\theta)$ and $U = R_y (\theta)$ using $CNOT$ and single qubit gates. 
    $$ ABC = \textbf{I} $$
    $$ U = \exp(i \alpha) AXBXC$$ 
    So $\exp (i \alpha) AXBXC = R_x (\theta) \ or \ R_y (\theta) $
    \begin{figure}
        \includegraphics[width = 20cm]{Photo 05-10-2020, 20 13 16.jpg}
        \centering
        \caption{Page of notes for Question 6 or Ex 4.23 from NC. Circuit diagram shown.}
    \end{figure}
The determinant of $U$ is $e^{i \alpha}$. Starting with $U = R_y (\theta)$ 
$$ U = R_y (\theta) = e^{i \theta \textbf{Y}} \rightarrow \alpha =0 $$
If $A = \textbf{I}$ then $B = e^{-i \theta \textbf{Y} /2}$ and $C = e^{-i \theta \textbf{Y} /2}$ so that $ABC = BC = \textbf{I}$ so $AXBXC = XBXC = e^{i \theta \textbf{Y}}$. This means that we can represent this as 2 single qubit gates.
\\
However if we have 
$$ U = R_x (\theta) $$
Then it seems like the minimum number of 1 qubit gates needed is 3 if we set 
$$ A = H, B = e^{-i \theta \textbf{Z}/2}, C = e^{i \theta \textbf{Z}/2}H $$
And $ABC$ cannot be reduced. 
\end{enumerate}
\pagebreak
\section{Comments}
\begin{enumerate}
    \item I understand what Big $O, \Omega,\Theta$ notation is and how it is used but I have never applied it or showed how it works before. I hope that a simple rudimentary python script is enough. I wish I could have shown it graphically but I couldn't figure out a good way to do this for an exponential function
    \item Question 2 was alright I just needed to recognise that I'm trying to get back to the Hadamard gate to see how the rotations have changed it, which I hope I have. 
    \item Question 3. I think this was the question which I struggled with the most as I couldn't figure out a way to create a substantial argument for this. 
    \item Question 4 was long but not that difficult once youve found the correct identity to get rid of that double dot product and include the cross product that is in the answer. 
    \item Question 5 was alright once I understood the circuit notation and what the question was actually asking me. 
    \item Question 6 confused me a little bit as the way I understood it the answer was already in the text above the question. The difficult bit was the second part which I've not fully sure I understand why you can only reduce for one type of rotation. 
\end{enumerate}




\end{document}

