
\documentclass[12pt]{article}
\usepackage[finnish]{babel}
\usepackage[T1]{fontenc}
\usepackage[utf8]{inputenc}
\usepackage{delarray,amsmath,bbm,epsfig,slashed}
\newcommand{\pat}{\partial}
\newcommand{\be}{\begin{equation}}
\newcommand{\ee}{\end{equation}}
\newcommand{\bea}{\begin{eqnarray}}
\newcommand{\eea}{\end{eqnarray}}
\newcommand{\abf}{{\bf a}}
\newcommand{\Zmath}{\mathbf{Z}}
\newcommand{\Zcal}{{\cal Z}_{12}}
\newcommand{\zcal}{z_{12}}
\newcommand{\Acal}{{\cal A}}
\newcommand{\Fcal}{{\cal F}}
\newcommand{\Ucal}{{\cal U}}
\newcommand{\Vcal}{{\cal V}}
\newcommand{\Ocal}{{\cal O}}
\newcommand{\Rcal}{{\cal R}}
\newcommand{\Scal}{{\cal S}}
\newcommand{\Lcal}{{\cal L}}
\newcommand{\Hcal}{{\cal H}}
\newcommand{\hsf}{{\sf h}}
\newcommand{\half}{\frac{1}{2}}
\newcommand{\Xbar}{\bar{X}}
\newcommand{\xibar}{\bar{\xi }}
\newcommand{\barh}{\bar{h}}
\newcommand{\Ubar}{\bar{\cal U}}
\newcommand{\Vbar}{\bar{\cal V}}
\newcommand{\Fbar}{\bar{F}}
\newcommand{\zbar}{\bar{z}}
\newcommand{\wbar}{\bar{w}}
\newcommand{\zbarhat}{\hat{\bar{z}}}
\newcommand{\wbarhat}{\hat{\bar{w}}}
\newcommand{\wbartilde}{\tilde{\bar{w}}}
\newcommand{\barone}{\bar{1}}
\newcommand{\bartwo}{\bar{2}}
\newcommand{\nbyn}{N \times N}
\newcommand{\repres}{\leftrightarrow}
\newcommand{\Tr}{{\rm Tr}}
\newcommand{\tr}{{\rm tr}}
\newcommand{\ninfty}{N \rightarrow \infty}
\newcommand{\unitk}{{\bf 1}_k}
\newcommand{\unitm}{{\bf 1}}
\newcommand{\zerom}{{\bf 0}}
\newcommand{\unittwo}{{\bf 1}_2}
\newcommand{\holo}{{\cal U}}
%\newcommand{\bra}{\langle}
%\newcommand{\ket}{\rangle}
\newcommand{\muhat}{\hat{\mu}}
\newcommand{\nuhat}{\hat{\nu}}
\newcommand{\rhat}{\hat{r}}
\newcommand{\phat}{\hat{\phi}}
\newcommand{\that}{\hat{t}}
\newcommand{\shat}{\hat{s}}
\newcommand{\zhat}{\hat{z}}
\newcommand{\what}{\hat{w}}
\newcommand{\sgamma}{\sqrt{\gamma}}
\newcommand{\bfE}{{\bf E}}
\newcommand{\bfB}{{\bf B}}
\newcommand{\bfM}{{\bf M}}
\newcommand{\cl} {\cal l}
\newcommand{\ctilde}{\tilde{\chi}}
\newcommand{\ttilde}{\tilde{t}}
\newcommand{\ptilde}{\tilde{\phi}}
\newcommand{\utilde}{\tilde{u}}
\newcommand{\vtilde}{\tilde{v}}
\newcommand{\wtilde}{\tilde{w}}
\newcommand{\ztilde}{\tilde{z}}

% David Weir's macros


\newcommand{\nn}{\nonumber}
\newcommand{\com}[2]{\left[{#1},{#2}\right]}
\newcommand{\mrm}[1] {{\mathrm{#1}}}
\newcommand{\mbf}[1] {{\mathbf{#1}}}
\newcommand{\ave}[1]{\left\langle{#1}\right\rangle}
\newcommand{\halft}{{\textstyle \frac{1}{2}}}
\newcommand{\ie}{{\it i.e.\ }}
\newcommand{\eg}{{\it e.g.\ }}
\newcommand{\cf}{{\it cf.\ }}
\newcommand{\etal}{{\it et al.}}
\newcommand{\ket}[1]{\vert{#1}\rangle}
\newcommand{\bra}[1]{\langle{#1}\vert}
\newcommand{\bs}[1]{\boldsymbol{#1}}
\newcommand{\xv}{{\bs{x}}}
\newcommand{\yv}{{\bs{y}}}
\newcommand{\pv}{{\bs{p}}}
\newcommand{\kv}{{\bs{k}}}
\newcommand{\qv}{{\bs{q}}}
\newcommand{\bv}{{\bs{b}}}
\newcommand{\ev}{{\bs{e}}}
\newcommand{\gv}{\bs{\gamma}}
\newcommand{\lv}{{\bs{\ell}}}
\newcommand{\nabv}{{\bs{\nabla}}}
\newcommand{\sigv}{{\bs{\sigma}}}
\newcommand{\notvec}{\bs{0}_\perp}
\newcommand{\inv}[1]{\frac{1}{#1}}
%\newcommand{\xv}{{\bs{x}}}
%\newcommand{\yv}{{\bs{y}}}
\newcommand{\Av}{\bs{A}}
%\newcommand{\lv}{{\bs{\ell}}}

%\newcommand\bsigma{\vec{\sigma}}
\hoffset 0.5cm
\voffset -0.4cm
\evensidemargin -0.2in
\oddsidemargin -0.2in
\topmargin -0.2in
\textwidth 6.3in
\textheight 8.4in

\begin{document}

\normalsize

\baselineskip 14pt

\begin{center}
{\Large {\bf Quantum Information A \ \ Fall 2020 \ \  Answers to Problem Set 1}}\\
{\large { Jake Muff}}\\
{Student number: 015361763}\\
{04/09/2020}
\end{center}


%Problems 3 and 4 are from J. J. Sakurai: {\em Modern Quantum Mechanics}, numbers 3.2 and 3.8, respectively.

\begin{enumerate}

\item Suppose $V$ is a vector space with basis vectors $\ket{0}$ and $\ket{1}$, and $A$ is a linear operator from $V$ to $V$ such that 
$$
A\ket{0} = \ket{1} \ ; \ A\ket{1} = \ket{0} \ .
$$
Give a matrix representation for $A$, with respect to the input and output basis $\ket{0},\ket{1}$. Find input and output bases which give rise to a different 
matrix representation of $A$. (For example, you can use the states $\ket{\pm}$ appearing in problem 5.)

%Question 1 Answer here
\begin{enumerate}
    \item \underline{\textbf{\emph{Answer.}}}
    From
    $$
    A\ket{v_j} = \sum_{i} A_{ij} \ket{w_i}
    $$
    We get
    $$
    A\ket{0} = A_{11}\ket{0}+A_{21}\ket{1} = \ket{1}
    $$
    $$
    A\ket{1} = A_{12}\ket{0}+A_{22}\ket{1} = \ket{0}
    $$
    $$ 
    A_{11} = 0, A_{21}=1,A_{12}=1,A_{22}=0
    $$
    $$ 
    A=
    \begin{bmatrix}
        0 & 1\\
        1 & 0
    \end{bmatrix}
    $$
    Input $ (\ket{0},\ket{1})$ with output $ (\ket{1},\ket{0})$
    \item For a different matrix representation of A
    $$
    A\ket{0} = A_{11}\ket{1}+A_{21}\ket{0} = \ket{1}
    $$
    $$
    A\ket{1} = A_{12}\ket{1}+A_{22}\ket{0} = \ket{0}
    $$
    $$ 
    A_{11} = 1, A_{21}=0,A_{12}=0,A_{22}=1
    $$
    $$ 
    A=
    \begin{bmatrix}
        1 & 0\\
        0 & 1
    \end{bmatrix}
    $$
    Input $ (\ket{0},\ket{1})$ with output $ (\ket{0},\ket{1})$
\end{enumerate}

\item Let the basis of $V$ be $\{\ket{0},\ket{1}\}\equiv \{e_1,e_2\} $. The tensor product vector space $V\otimes V$ has the basis
\begin{eqnarray}
&&\{e_1\otimes e_1, e_1\otimes e_2, e_2\otimes e_1, e_2\otimes e_2\} = \{\ket{0}\ket{0}, \ket{0}\ket{1},\ket{1}\ket{0},\ket{1}\ket{1} \} \nonumber \\
&&\equiv \{\ket{00}, \ket{01},\ket{10},\ket{11} \} \ .
\end{eqnarray}
Let us use the two-component notation
\bea
\ket{0} = \left(\begin{array}{c} 1 \\ 0 \end{array}\right) \ ; \ \ket{1} = \left(\begin{array}{c} 0 \\ 1 \end{array}\right) \ .
\eea
In the component notation, the tensor product of vectors becomes the Kronecker product,
\bea
 \left(\begin{array}{c} a \\ b \end{array}\right)  \otimes \left(\begin{array}{c} c \\ d \end{array}\right) =  \left(\begin{array}{c} a  \left(\begin{array}{c} c \\ d \end{array}\right)  \\ b  \left(\begin{array}{c} c \\ d \end{array}\right)  \end{array}\right) =  \left(\begin{array}{c} ac \\ ad \\  bc \\ bd \end{array}\right) \ .
\eea
Show that 
\bea
\ket{00} =\left(\begin{array}{c} 1 \\ 0 \\  0 \\ 0 \end{array}\right)  \ ; \ \ket{01} =\left(\begin{array}{c} 0 \\ 1 \\  0 \\ 0 \end{array}\right)  \ ; \ \ket{10} =\left(\begin{array}{c} 0 \\ 0 \\  1 \\ 0 \end{array}\right)  \ ; \ \ket{11} =\left(\begin{array}{c} 0 \\ 0 \\  0 \\ 1 \end{array}\right)  \ .
\eea



%Question 2
\begin{enumerate}
    \item \underline{\textbf{\emph{Answer.}}}
    $$
    \ket{00} = \ket{0}\ket{0} 
    $$
    \bea
 \left(\begin{array}{c} 1 \\ 0 \end{array}\right)  \otimes \left(\begin{array}{c} 1 \\ 0 \end{array}\right) =  \left(\begin{array}{c} 1 \times 1 \\ 1 \times 0 \\  0 \times 0 \\ 0 \times 1 \end{array}\right) =  \left(\begin{array}{c} 1 \\ 0 \\ 0 \\ 0 \end{array}\right) \ .
    \eea
    $$
    \ket{01} = \ket{0}\ket{1} 
    $$
    \bea
 \left(\begin{array}{c} 1 \\ 0 \end{array}\right)  \otimes \left(\begin{array}{c} 0 \\ 1 \end{array}\right) =  \left(\begin{array}{c} 1 \times 0 \\ 1 \times 1 \\  0 \times 0 \\ 0 \times 1 \end{array}\right) =  \left(\begin{array}{c} 0 \\ 1 \\ 0 \\ 0 \end{array}\right) \ .
    \eea
    $$
    \ket{10} = \ket{1}\ket{0} 
    $$
    \bea
 \left(\begin{array}{c} 0 \\ 1 \end{array}\right)  \otimes \left(\begin{array}{c} 1 \\ 0 \end{array}\right) =  \left(\begin{array}{c} 0 \times 1 \\ 0 \times 0 \\  1 \times 1 \\ 1 \times 0 \end{array}\right) =  \left(\begin{array}{c} 0 \\ 0 \\ 1 \\ 0 \end{array}\right) \ .
    \eea
    $$
    \ket{11} = \ket{1}\ket{1} 
    $$
    \bea
 \left(\begin{array}{c} 0 \\ 1 \end{array}\right)  \otimes \left(\begin{array}{c} 0 \\ 1 \end{array}\right) =  \left(\begin{array}{c} 0 \times 0 \\ 0 \times 1 \\  1 \times 0 \\ 1 \times 1 \end{array}\right) =  \left(\begin{array}{c} 0 \\ 0 \\ 0 \\ 1 \end{array}\right) \ .
    \eea
\end{enumerate}



\item Using the above 4-component notation, show that the CNOT operator
\be
 U_{CN} = \left(\begin{array}{cccc} 1 & 0 & 0 & 0\\ 0 & 1 & 0 & 0 \\  0 & 0 & 0 & 1 \\ 0 & 0 & 1 & 0 \end{array}\right)
\ee
acts on the (computational) basis states as 
\bea
 U_{CN}\ket{00} = \ket{00} \ ; \   U_{CN}\ket{01} = \ket{01} \ ; \   U_{CN}\ket{10} = \ket{11} \ ; \   U_{CN}\ket{11} = \ket{10} \ . 
\eea

%Question 3
\begin{enumerate}
    \item \underline{\textbf{\emph{Answer.}}} Simply use matrix multiplication.
    $$
 U_{CN}\ket{00} = 
    $$
    
    \bea
U_{CN} \cdot \left(\begin{array}{c} 1 \\ 0 \\ 0 \\ 0 \end{array}\right) =  \left(\begin{array}{c} 1 \\ 0 \\ 0 \\ 0 \end{array}\right) = \ket{00}.
    \eea
    $$
 U_{CN}\ket{01} = 
    $$
    
    \bea
U_{CN} \cdot \left(\begin{array}{c} 0 \\ 1 \\ 0 \\ 0 \end{array}\right) =  \left(\begin{array}{c} 0 \\ 1 \\ 0 \\ 0 \end{array}\right) = \ket{01}.
    \eea

    $$
 U_{CN}\ket{10} = 
    $$
    
    \bea
U_{CN} \cdot \left(\begin{array}{c} 0 \\ 0 \\ 1 \\ 0 \end{array}\right) =  \left(\begin{array}{c} 0 \\ 0 \\ 0 \\ 1 \end{array}\right) = \ket{11}.
    \eea

    $$
 U_{CN}\ket{11} = 
    $$
    
    \bea
U_{CN} \cdot \left(\begin{array}{c} 0 \\ 0 \\ 0 \\ 1 \end{array}\right) =  \left(\begin{array}{c} 0 \\ 0 \\ 1 \\ 0 \end{array}\right) = \ket{10}.
    \eea

\end{enumerate}


\item Recall that (from Mathematical Methods of Physics IIIa or equivalent) the abelian group $\Zmath_2$ can be realized as the set $\{0,1\}$ with addition modulo two as the product.
The latter is the same as the XOR operation of bits, and we denote it with the symbol $\oplus$:
\be
0\oplus 0 = 1\oplus 1 = 0 \ ; \ 0\oplus 1 = 1\oplus 0 = 1 \ .
\ee
Note that $x\oplus x=0$. Recall also that we can define a product group 
\be
\Zmath^n_2 = \underbrace{\Zmath_2 \times \Zmath_2 \times \cdots \times \Zmath_2}_{n\ times} = \{0,1\}^n
\ee
with elements $x = (x_1,x_2,\ldots ,x_n)$ with each $x_i\in \Zmath_2$, with the product
\be
x\oplus y = (x_1\oplus y_1, x_2\oplus y_2,\ldots , x_n\oplus y_n) \ .
\ee
Next we shorten the notation and write
\be
x = x_1x_2\cdots x_n = (x_1,x_2,\ldots ,x_n) \ .
\ee
You notice that $x$ is a binary number with $n$ bits, e.g. $x= 011011$ for $n=6$. The above rule then extends the XOR operation for $n$-bit binary numbers. Converting
the binary numbers to decimal numbers and back gives fun results, e.g. $3\oplus 5 = 011\oplus 101 = 110 = 6$. Practise this by
calculating the results (in binary form) of
\begin{enumerate}
\item $01011\oplus 10001$
\item $10001  \oplus 01111$
\item $(01101\oplus 10101)\oplus 11100$
\end{enumerate}  

%Question 4
\begin{enumerate}
    \item \underline{\textbf{\emph{Answer.}}} Use the XOR operation on binary numbers and convert to decimal
    \emph{Example: $3\oplus 5 = 011\oplus 101 = 110 = 6$} $$ 
        \begin{tabular}{cccc}
            & 0 & 1 & 1 \\
            $\oplus$ &  1 & 0 & 1 \\
          \hline
            & 1 & 1 & 0 \\
          \end{tabular}
          = 6_{10}
        $$
    \emph{(a)$01011\oplus 10001$} $$ 
        \begin{tabular}{cccccc}
            & 0 & 1 & 0 & 1 & 1 \\
             $\oplus$ & 1 & 0 & 0 & 0 & 1 \\
        \hline
            & 1 & 1 & 0 & 1 & 0\\
        \end{tabular}
        = 26_{10} 
        $$
    \emph{(b)$10001\oplus 01111$} $$ 
        \begin{tabular}{cccccc}
            & 1 & 0 & 0 & 0 & 1 \\
            $\oplus$ & 0 & 1 & 1 & 1 & 1 \\
            \hline
            & 1 & 1 & 1 & 1 & 0\\
            \end{tabular}
            = 30_{10} 
        $$
    \emph{(b)$(01101\oplus 10101) \oplus 11100$} $$ 
    \begin{tabular}{cccccc}
        & 0 & 1 & 1 & 0 & 1 \\
        $\oplus$ & 1 & 0 & 1 & 0 & 1 \\
        \hline
        & 1 & 1 & 0 & 0 & 0\\
        \end{tabular}
        \implies 
        \begin{tabular}{cccccc}
            & 1 & 1 & 0 & 0 & 0 \\
            $\oplus$ & 1 & 1 & 1 & 0 & 0 \\
            \hline
            & 0 & 0 & 1 & 0 & 0\\
            \end{tabular}
            =4_{10}
        $$
\end{enumerate}

\item The Hadamard  gate is represented by
\be
 H = \frac{1}{\sqrt{2}}  \left(\begin{array}{cc} 1 & 1 \\ 1 & -1 \end{array}\right) \ .
\ee
Define the states
\be
 \ket{+} =  \frac{1}{\sqrt{2}} (\ket{0}+\ket{1}) = H\ket{0}\ ; \  \ket{+} =  \frac{1}{\sqrt{2}} (\ket{0}-\ket{1}) = H\ket{1}\ .
\ee
Let $H$ act on them, what are the resulting states $H\ket{+},H\ket{-}$?

%Question 5
\begin{enumerate}
    \item \underline{\textbf{\emph{Answer.}}} $H\ket{+}$
    $$ 
    H\ket{+} = H(\frac{1}{\sqrt{2}} (\ket{0}+\ket{1})) $$
    $$
    = \frac{1}{\sqrt{2}}  \left(\begin{array}{cc} 1 & 1 \\ 1 & -1 \end{array}\right) \ (\frac{1}{\sqrt{2}} (\ket{0}+\ket{1})$$
    $$
    = \frac{1}{2} \ket{0} + \frac{1}{2} \ket{1} + \frac{1}{2}\ket{0} - \frac{1}{2} \ket{1} $$
    $$
    = \frac{1}{2} (\ket{0}+\ket{1}) + \frac{1}{2}(\ket{0}-\ket{1}) = \ket{0} $$
    $$
    H\ket{+} = \ket{0} $$
    \item $ H\ket{-} $
    $$ 
    H\ket{-} = H(\frac{1}{\sqrt{2}} (\ket{0}-\ket{1})) $$
    $$
    = \frac{1}{\sqrt{2}}  \left(\begin{array}{cc} 1 & 1 \\ 1 & -1 \end{array}\right) \ (\frac{1}{\sqrt{2}} (\ket{0}-\ket{1})$$
    $$
    = \frac{1}{2} \ket{0} + \frac{1}{2} \ket{1} - \frac{1}{2}\ket{0} + \frac{1}{2} \ket{1} $$
    $$
    = \frac{1}{2} (\ket{0}+\ket{1}) + \frac{1}{2}(-\ket{0}+\ket{1}) = \ket{1} $$
    $$
    H\ket{-} = \ket{1} $$
\end{enumerate}

\end{enumerate}


\end{document}

