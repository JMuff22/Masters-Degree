
\documentclass[12pt]{article}
\usepackage[finnish]{babel}
\usepackage[T1]{fontenc}
\usepackage[utf8]{inputenc}
\usepackage{delarray,amsmath,bbm,epsfig,slashed,amssymb}
\newcommand{\pat}{\partial}
\newcommand{\be}{\begin{equation}}
\newcommand{\ee}{\end{equation}}
\newcommand{\bea}{\begin{eqnarray}}
\newcommand{\eea}{\end{eqnarray}}
\newcommand{\abf}{{\bf a}}
\newcommand{\Zmath}{\mathbf{Z}}
\newcommand{\Zcal}{{\cal Z}_{12}}
\newcommand{\zcal}{z_{12}}
\newcommand{\Acal}{{\cal A}}
\newcommand{\Fcal}{{\cal F}}
\newcommand{\Ucal}{{\cal U}}
\newcommand{\Vcal}{{\cal V}}
\newcommand{\Ocal}{{\cal O}}
\newcommand{\Rcal}{{\cal R}}
\newcommand{\Scal}{{\cal S}}
\newcommand{\Lcal}{{\cal L}}
\newcommand{\Hcal}{{\cal H}}
\newcommand{\hsf}{{\sf h}}
\newcommand{\half}{\frac{1}{2}}
\newcommand{\Xbar}{\bar{X}}
\newcommand{\xibar}{\bar{\xi }}
\newcommand{\barh}{\bar{h}}
\newcommand{\Ubar}{\bar{\cal U}}
\newcommand{\Vbar}{\bar{\cal V}}
\newcommand{\Fbar}{\bar{F}}
\newcommand{\zbar}{\bar{z}}
\newcommand{\wbar}{\bar{w}}
\newcommand{\zbarhat}{\hat{\bar{z}}}
\newcommand{\wbarhat}{\hat{\bar{w}}}
\newcommand{\wbartilde}{\tilde{\bar{w}}}
\newcommand{\barone}{\bar{1}}
\newcommand{\bartwo}{\bar{2}}
\newcommand{\nbyn}{N \times N}
\newcommand{\repres}{\leftrightarrow}
\newcommand{\Tr}{{\rm Tr}}
\newcommand{\tr}{{\rm tr}}
\newcommand{\ninfty}{N \rightarrow \infty}
\newcommand{\unitk}{{\bf 1}_k}
\newcommand{\unitm}{{\bf 1}}
\newcommand{\zerom}{{\bf 0}}
\newcommand{\unittwo}{{\bf 1}_2}
\newcommand{\holo}{{\cal U}}
%\newcommand{\bra}{\langle}
%\newcommand{\ket}{\rangle}
\newcommand{\muhat}{\hat{\mu}}
\newcommand{\nuhat}{\hat{\nu}}
\newcommand{\rhat}{\hat{r}}
\newcommand{\phat}{\hat{\phi}}
\newcommand{\that}{\hat{t}}
\newcommand{\shat}{\hat{s}}
\newcommand{\zhat}{\hat{z}}
\newcommand{\what}{\hat{w}}
\newcommand{\sgamma}{\sqrt{\gamma}}
\newcommand{\bfE}{{\bf E}}
\newcommand{\bfB}{{\bf B}}
\newcommand{\bfM}{{\bf M}}
\newcommand{\cl} {\cal l}
\newcommand{\ctilde}{\tilde{\chi}}
\newcommand{\ttilde}{\tilde{t}}
\newcommand{\ptilde}{\tilde{\phi}}
\newcommand{\utilde}{\tilde{u}}
\newcommand{\vtilde}{\tilde{v}}
\newcommand{\wtilde}{\tilde{w}}
\newcommand{\ztilde}{\tilde{z}}

% David Weir's macros


\newcommand{\nn}{\nonumber}
\newcommand{\com}[2]{\left[{#1},{#2}\right]}
\newcommand{\mrm}[1] {{\mathrm{#1}}}
\newcommand{\mbf}[1] {{\mathbf{#1}}}
\newcommand{\ave}[1]{\left\langle{#1}\right\rangle}
\newcommand{\halft}{{\textstyle \frac{1}{2}}}
\newcommand{\ie}{{\it i.e.\ }}
\newcommand{\eg}{{\it e.g.\ }}
\newcommand{\cf}{{\it cf.\ }}
\newcommand{\etal}{{\it et al.}}
\newcommand{\ket}[1]{\vert{#1}\rangle}
\newcommand{\bra}[1]{\langle{#1}\vert}
\newcommand{\bs}[1]{\boldsymbol{#1}}
\newcommand{\xv}{{\bs{x}}}
\newcommand{\yv}{{\bs{y}}}
\newcommand{\pv}{{\bs{p}}}
\newcommand{\kv}{{\bs{k}}}
\newcommand{\qv}{{\bs{q}}}
\newcommand{\bv}{{\bs{b}}}
\newcommand{\ev}{{\bs{e}}}
\newcommand{\gv}{\bs{\gamma}}
\newcommand{\lv}{{\bs{\ell}}}
\newcommand{\nabv}{{\bs{\nabla}}}
\newcommand{\sigv}{{\bs{\sigma}}}
\newcommand{\notvec}{\bs{0}_\perp}
\newcommand{\inv}[1]{\frac{1}{#1}}
%\newcommand{\xv}{{\bs{x}}}
%\newcommand{\yv}{{\bs{y}}}
\newcommand{\Av}{\bs{A}}
%\newcommand{\lv}{{\bs{\ell}}}

%\newcommand\bsigma{\vec{\sigma}}
\hoffset 0.5cm
\voffset -0.4cm
\evensidemargin -0.2in
\oddsidemargin -0.2in
\topmargin -0.2in
\textwidth 6.3in
\textheight 8.4in

\begin{document}

\normalsize

\baselineskip 14pt

\begin{center}
{\Large {\bf Open Quantum Systems \ \ Fall 2020 \ \  Answers to Exercise Set 1}}\\
{\large { Jake Muff}}\\
{Student number: 015361763}\\
{13/09/2020}
\end{center}



\begin{enumerate}

\item Exercise 1
%Question 1 Answer here
\begin{enumerate}
    \item Trace of an operator A is
     $$ Tr(A) = \sum_n u_n^\dagger A u_n = \sum_n \bra{u_n}A\ket{u_n} $$
     Proof:
     \\
     Choose another basis $w_n$. SInce this is an orthonormal basis we can write 
     $$ I= \sum_i \ket{w_i}\bra{w_i} $$
     $$\sum_n \bra{u_n}A\ket{u_n} = \sum_{i,j,n} \langle{u_n}|{w_i}\rangle\bra{w_i}A\ket{w_j}\langle w_j|u_n \rangle $$
     $$ = \sum_{i,j,n}\langle{w_j}|u_n\rangle\langle u_n|w_i\rangle\bra{w_i}A\ket{w_j} $$
     $$ = \sum_{i,j} \langle w_j|w_i\rangle\bra{w_i}A\ket{w_j} $$
     $$=\sum_{i,j} \delta_{i,j} \bra{w_i}A\ket{w_j} $$
     $$Tr(A)=\sum_i \bra{w_i}A\ket{w_i}$$
    
     \item $Tr(Avv^{\dagger})=v^\dagger A v$
     \\ Using dirac notation $\ket{v}^\dagger = \bra{v}$ and $(\ket{v}\bra{w})^\dagger = \ket{w}\bra{v}$
     \\ We can therefore see:
     $$ Tr(Avv^{\dagger}) = Tr(A\ket{v}\bra{v}) $$
     $$ = \sum_i \bra{i}A\ket{v}\langle v | i \rangle $$
     $$ = \bra{v}A\ket{v} $$
     $$ = v^\dagger A v$$
     For an orthonormal basis $\ket{i}$
\end{enumerate}

\item Exercise 2



%Question 2
\begin{enumerate}
    \item For a vector space $M_2 (\mathbb{C})$
$$
I = \left( \begin{array}{cc} 1 & 0 \\ 0 & 1 \end{array} \right)  \ ; \ \sigma_x = \left( \begin{array}{cc} 0 & 1 \\ 1 & 0 \end{array} \right) 
$$
$$
\sigma_y = \left( \begin{array}{cc} 0 & -i \\ i & 0 \end{array} \right)  \ ; \ \sigma_z = \left( \begin{array}{cc} 1 & 0 \\ 0 & -1 \end{array} \right) 
$$
\\
To show that the Pauli matrices denoted by $\{I,\sigma_i\}$ where $i = \{x,y,z\}$ form a basis for this vector space we have to show that $\{I,\sigma_i\}$ are linearly independent and that every matrices in $M_2(\mathbb{C})$ can be written as a linear combination of $\{I,\sigma_i\}$ and hence spans $M_2(\mathbb{C})$.
\\ For this first: Let $b_j \in \mathbb{C}$ such that: 
$$ 
b_0 I +b_1 \sigma_x + b_2 \sigma_y + b_3 \sigma_z = \left( \begin{array}{cc} 0 & 0 \\ 0 & 0 \end{array} \right)
$$
$$ 
 = \left( \begin{array}{cc} b_0+b_3 & b_1-ib_2 \\ b_1+ib_2 & b_0-b_3 \end{array} \right) = \left( \begin{array}{cc} 0 & 0 \\ 0 & 0 \end{array} \right)
$$
$$
b_0 = b_1 =b_2 =b_3=0 $$
Thus $\{I,\sigma_i\}$ are linearly independent.
\\
For the linear combination:
$$ 
\left( \begin{array}{cc} b_0+b_3 & b_1-ib_2 \\ b_1+ib_2 & b_0-b_3 \end{array} \right) = \left( \begin{array}{cc} Z_{11} & Z_{12} \\ Z_{21} & Z_{22} \end{array} \right)
$$
So
$$ b_0 +b_3 = Z_{11} $$
$$ b_1 -ib_2 = Z_{12} $$
$$ b_1 +ib_2 = Z_{21} $$
$$ b_0 - b_3 = Z_{22} $$
Hence:
$$ b_0 = \frac{1}{2} (Z_{11} + Z_{22}) $$
$$ b_1 = \frac{1}{2}(Z_{12} +Z_{21}) $$
$$ b_2 = \frac{1}{2}i(Z_{12}-Z_{21})$$
$$ b_3 = \frac{1}{2}(Z_{11}+Z_{22}) $$ 
With every matrix written as a linear combination of $\{I,\sigma_i\}$ and spans $M_2(\mathbb{C})$. Showing that the Pauli matrices form a basis for this vector space. 


\end{enumerate}



\item Exercise 3
%Question 3
\begin{enumerate}
    \item $(A,B) = Tr(A^\dagger B) $ %\emph{Note I use: $\overline{A} = A^*$}
    \begin{enumerate}
        \item $(A,B) = \overline{(B,A)}$ 
        $$ (A,B) = Tr(A^\dagger B) $$
        $$ = \sum_{i,j} \bra{i}A^\dagger \ket{j} \bra{j}B\ket{i} $$
        
        $$ = \sum_{i,j} \bra{i}B^\dagger \ket{j} ^* \bra{j}A\ket{i}^* $$
        $$ = \sum_i \bra{i} B^\dagger A \ket{i} $$
        $$ = \overline{Tr(B^\dagger A)} = \overline{(B,A)} $$

        \item $(A+C,B)=(A,B) + (C,B)$
        Useful Identities:
        $$ (A+C)^\dagger B = A^\dagger B + C^\dagger B $$
        $$ (A^\dagger + C^\dagger) B = A^\dagger B + C^\dagger B $$
        Using these we can take the trace to get the answer. 
        \\ From exercise (a) we have 
        $$ (A,B) = Tr(A^\dagger B) $$ 
        So:
        $$ Tr((A+C)^\dagger B) = Tr((A^\dagger + C^\dagger)B) $$
        $$ = Tr(A^\dagger B + C^\dagger B) $$ 
        $$ = \sum_i \bra{i}A^\dagger B + C^\dagger B \ket{i} $$
        $$ = \sum_i (\bra{i}A^\dagger B \ket{i} + \bra{i}C^\dagger B\ket{i}) $$
        $$ =Tr(A^\dagger B) + Tr(C^\dagger B) $$
        $$ = (A,B) + (C,B) $$
        \\ For the second part $(A,cB) = c(A,B) $ for $c \in \mathbb{C}$:
        $$ (A, \sum_i c_i B_i ) = Tr[A^\dagger (\sum_i c_i B_i)] $$
        $$ = Tr(A^\dagger c_1 B_1) + \ldots + Tr(A^\dagger c_n B_n) $$
        $$ = c_1 Tr(A^\dagger B_1) + \ldots c_n Tr(A^\dagger B_n) $$
        $$ =\sum_i c_i Tr(A^\dagger B_i) $$
        $$ =c(A,B) $$
        \item $(A,A) > 0$. 
        $$ (A,A) = Tr(A^\dagger A) $$
        $$ = \sum_i \bra{i}A^\dagger A \ket{i} $$
        \\ First prove that $A^\dagger A$ is positive:
        $$ \bra{\psi}A^\dagger A \ket{\psi} = || A \ket{\psi} || ^2 $$ 
        For all $\ket{\psi}$. Thus $A^\dagger A$ is positive. Proving that:  
        $$ \sum_i \bra{i}A^\dagger A \ket{i} \geq 0 $$ For all $\ket{i}$.
        \\ If $a_i$ is the i-th column of A and $\bra{i}A^\dagger A \ket{i} = 0$ then 
        $$ \bra{i}A^\dagger A \ket{i} = a_i^\dagger a_i = ||a_i || ^2 =0 $$ if and only if $a_i = \bf{0}$.
        Therefore for A not the zero-matrix $(A,A) > 0$. 
    \end{enumerate}
    \item Two vectors are said to be orthogonal if their inner product i.e $\langle u,v \rangle= 0 $. 
    \\ For any of the Pauli matrices 
    $$ \sigma_i \cdot \sigma_i = I  $$
    Where:
    $$ \sigma_0 = I $$
    $$ \sigma_x = \sigma_1 $$
    $$ \sigma_y = \sigma_2 $$
    $$ \sigma_z = \sigma_3 $$
    So for the Pauli matrices they are orthogonal with respect to the inner product if 
    $$ \langle \sigma_i, \sigma_j \rangle =0 $$
    Note that $\sigma_i \cdot I = \sigma_i \cdot \sigma_0 = \sigma_i $
    The trace of $\sigma_i^\dagger \sigma_j$ is always 0 as well. Listing all the different resulting matrices we can see that the trace is 0.
    $$ \sigma_x \cdot \sigma_y = i\sigma_z = \left( \begin{array}{cc} i & 0 \\ 0 & -i \end{array} \right) $$
    $$ \sigma_y \cdot \sigma_x = -i\sigma_z = \left( \begin{array}{cc} -i & 0 \\ 0 & i \end{array} \right) $$
    $$ \sigma_y \cdot \sigma_z = i\sigma_x = \left( \begin{array}{cc} 0 & i \\ i & 0 \end{array} \right) $$
    $$ \sigma_z \cdot \sigma_y = -i\sigma_x = \left( \begin{array}{cc} 0 & -i \\ -i & 0 \end{array} \right) $$
    $$ \sigma_z \cdot \sigma_x = i\sigma_y = \left( \begin{array}{cc} 0 & 1 \\ -1 & 0 \end{array} \right) $$
    $$ \sigma_x \cdot \sigma_z = -i\sigma_y = \left( \begin{array}{cc} 0 & -1 \\ 1 & 0 \end{array} \right) $$
    $$ \sigma_y^\dagger \cdot \sigma_x = \sigma_z = \left( \begin{array}{cc} 1 & 0 \\ 0 & -1 \end{array} \right) $$
    $$ \sigma_y^\dagger \cdot \sigma_z = -\sigma_x = \left( \begin{array}{cc} 0 & -1 \\ -1 & 0 \end{array} \right) $$

    Thus:
    $$ \langle \sigma_i | \sigma_j \rangle = Tr(\sigma_i^\dagger \sigma_j) =0$$

\end{enumerate}


\item Exercise 4

%Question 4
\begin{enumerate}
    \item $X \in M_n(\mathbb{C})$ 
    \\
    $a \rightarrow b$:
    \\
    If X is of the form $B^\dagger B$ then $\phi^\dagger X \phi$ can be proved to be positive for all $\phi$ in $\mathbb{C}^n$ as shown:
    $$ \phi^\dagger X \phi = \bra{\phi}B^\dagger B\ket{\phi} $$
    $$ = || B \ket{\phi} || ^2 \geq 0 $$ for all $\phi \in \mathbb{C}^n$ 

    \item $b \rightarrow c$ 
    \\
    Suppose (b) shows us that X is a positive operator $\bra{\phi}X\ket{\phi} \geq 0$ then X can be decomposed as follows:
    $$ X = \frac{X+X^\dagger}{2} +i \frac{X-X^\dagger}{2i} $$
    $$ C = \frac{X+X^\dagger}{2}, D=\frac{X-X^\dagger}{2i}$$
    Note that C and D are hermitian.
    $$ \bra{\phi}X\ket{\phi} = \bra{\phi}C+iD\ket{\phi} $$
    $$ = \bra{\phi}C\ket{\phi} + i\bra{\phi}D\ket{\phi} $$
    $$ = \alpha + i\beta $$
    Where 
    $$ \alpha = \bra{\phi}C\ket{\phi}, \beta = \bra{\phi}D\ket{\phi} $$
    Since C and D are hermitian, $\alpha,\beta \in \mathbb{R}$. From definition of a positive operator, (b), $\beta$ should vanish since $\bra{\phi}X\ket{\phi}$ is real. 
    $$ \beta = \bra{\phi}D\ket{\phi} = 0 \forall \ket{\phi} $$
    With $D=0$. 
    $$ \therefore X = X^\dagger $$
    If we also suppose that there is a projector $P$ onto the subspace X then it follows that every eigenvalue is non-negative as:
    $$ P^2 = P $$
    $$ P\ket{\lambda} = \lambda \ket{\lambda} $$
    $$ P\ket{\lambda} = P^2 \ket{\lambda} = \lambda P \ket{\lambda} = \lambda^2 \ket{\lambda} $$
    $$ \lambda = \lambda^2 $$
    $$ \lambda(\lambda-1)=0 \rightarrow \lambda = 0 \ \text{or} \ 1 $$
    And therefore non-negative. 

    \item $c \rightarrow a$ 
    \\
    To go from (c) to (a) we use spectral decomposition to show it is normal and of the form $B^\dagger B$ by proving that it is diagonalizable. 
    \\
    \emph{Note}: $P$ is the project onto $X$ and $Q$ is the orthogonal complement of $P$ such that $Q=I-P$ where $I$ is the identity matrix.
    \\
    $$ X = IXI $$
    $$ = (P+Q)X(P+Q) $$
    $$ = PXP + QXP +PXQ + QXQ $$ 
    Now: 
    $$ PXP=\lambda P $$
    $$ QXP =0 $$
    $$ PXQ = PX^\dagger Q =0 $$ 
    So we have: 
    $$ X = PXP + QXQ $$ 
    $$ QXQ(QXQ)^\dagger = QXQQX^\dagger Q $$
    $$ = QX^\dagger QQXQ $$
    $$= (QX^\dagger Q) QXQ $$
    By induction $QXQ$ is diagonal. $PXP$ is already diagonal due to the nature of $P$. It follows therefore that $X$ is diagonal as well. $X$ being diagonal means it is of the form $B^\dagger B$ and we have $c \rightarrow a$. 


\end{enumerate}

\item Exercise 5

%Question 5
\begin{enumerate}
    \item For $A,B \in M_n(\mathbb{C})$, show that the trace is cyclic. 
    \\ Proof: 
    $$ Tr(AB) = \sum_i \bra{i}AB\ket{i} $$
    $$ = \sum_i \bra{i} AIB\ket{i} $$
    $$ = \sum_{i,j} \bra{i}A\ket{j}\bra{j}B\ket{i} $$
    $$ = \sum_{i,j} /bra{j}B\ket{i}\bra{i}A\ket{j} $$
    $$ =\sum_j \bra{j} BA \ket{j} $$
    $$ = Tr(BA) $$
    \item Show that $Tr(XY) \geq 0$ 
    $X$ and $Y$ are $n\times n$ square matrices and are both positive semidefinite ($\geq 0$). This means that there must exist a matrix $Z$ such that $Y=ZZ^\dagger$, therefore we have:
    $$ Tr(XY) = Tr(XZZ^{\dagger}) = Tr(Z^{\dagger} X Z) = \sum_{i=1}^{n} z_i^{\dagger} X z_i $$
    $  \sum_{i=1}^n z_i^\dagger X z_i \geq 0 $ as $X$ is positive semidefinite. 
    \\ 
    
    \item Is $XY$ a positive matrix?
    $XY$ is always a positive as both $X$ and $Y$ are hermitian matrices, therefore they will always have real and positive eigenvalues, meaning that their product will always be positive. 

\end{enumerate}


\item Exercise 6
%Question 6
\begin{enumerate}
    \item Show that $\exp(A) = \sum_i \exp(a_i)v_i v_i^\dagger $
    \\ First we note that $A^\dagger = (A^*)^T$ and that if A is fully diagonalizable then A is of the form $A = PDP^{-1}$ where $P$ is an invertible matrix and $D$ is a diagonal matrix. 
    We therefore have: 
    $$ \exp(PDP^{-1}) = \sum_{n=0}^{\infty} \frac{1}{n!} (PDP^{-1})^n$$ 
    $$ = \sum_{n=0}^{\infty} \frac{1}{n!} PD^nP^{-1} $$
    $$ = P(\sum_{n=0}^{\infty} \frac{1}{n!} D^n) P^{-1} $$
    $$ P \exp(D) P^{-1} $$
    Using this: 
    $$ \exp(\sum_i a_i v_i v_i^\dagger) = \sum_{n=0}^{\infty} \frac{1}{n!}(\sum_i a_i v_i v_i^\dagger)^n $$
    $$ = v_i (\sum_{n=0}^{\infty} \frac{1}{n!} (\sum_i a_i)^n ) v_i^\dagger $$
    $$ = \sum_i v_i \exp(a_i) v_i^\dagger $$
    $$ = \sum_i \exp(a_i) v_i v_i^\dagger $$

    \item Show that $\exp(i\sigma_i t) = cos(t)\mathbb{I} + isin(t)\sigma_i  $
    $$
    \sigma_1 = \left( \begin{array}{cc} 0 & 1 \\ 1 & 0 \end{array} \right)  \ ; \ \sigma_2 = \left( \begin{array}{cc} 0 & -i \\ i & 0 \end{array} \right) 
    $$
    $$
    \sigma_3 = \left( \begin{array}{cc} 1 & 0 \\ 0 & -1 \end{array} \right)  \ 
    $$
    Define a vector dot product such that 
    $$ \vec{x} \cdot \vec{\sigma} = \sum_i^3 x_i \cdot \sigma_i$$
    So:
    $$ x_1 \left( \begin{array}{cc} 0 & 1 \\ 1 & 0 \end{array} \right)  + x_2 \left( \begin{array}{cc} 0 & -i \\ i & 0 \end{array} \right) + x_3 \left( \begin{array}{cc} 1 & 0 \\ 0 & -1 \end{array} \right)$$
    $$ = \left( \begin{array}{cc} x_3 & x_1-ix_2 \\ x_1+ix_2 & -x_3 \end{array} \right) $$
    Lets find the eigenvalues: 
    $$ det(\vec{x} \cdot \vec{\sigma} - \lambda I) = |\left( \begin{array}{cc} x_3-\lambda & x_1 -ix_2 \\ x_1 +i x_2 & -x_3 -\lambda \end{array} \right)| $$
    $$ = (x_3^2 +\lambda^2) - (x_1^2 +x_2^2) $$
    $$ = \lambda^2 - (x_1^2 +x_2^2 +x_3^2) $$
    $$ \lambda^2 -1 =0 $$ 
    $$ \lambda = \pm 1 $$
    With $ \ket{\lambda_{\pm{1}}} $ eigenvectors. 
    \\ 
    $\vec{x} \cdot \vec{\sigma} $ is hermitian and therefore diagonalizable so 
    $$ \vec{x} \cdot \vec{\sigma} = \ket{\lambda_1}\bra{\lambda_1} - \ket{\lambda_{-1}}\bra{\lambda_{-1}} $$
    Therefore taking the exponential like in (a): 
    $$ \exp(it\vec{x}\cdot\vec{\sigma}) = \exp(i\sigma_i t) $$
    $$ \exp(i\sigma_i t) = \exp(it) \ket{\lambda_1}\bra{\lambda_1} - \exp(it) \ket{\lambda_{-1}}\bra{\lambda_{-1}} $$
    $$ = \exp(it) \ket{\lambda_1}\bra{\lambda_1} + \exp(-it) \ket{\lambda_{-1}}\bra{\lambda_{-1}} $$
    $$ = cos(t) + isin(t) \ket{\lambda_1}\bra{\lambda_1} + cos(t) -isin(t) \ket{\lambda_{-1}}\bra{\lambda_{-1}} $$
    $$ =cos(t)\Big[\ket{\lambda_1}\bra{\lambda_1} + \ket{\lambda_{-1}}\bra{\lambda_{-1}}\Big] + isin(t)\Big[\ket{\lambda_1}\bra{\lambda_1} - \ket{\lambda_{-1}}\bra{\lambda_{-1}} \Big] $$ 
    $$ = cos(t)\mathbb{I} + isin(t)\sigma_i $$
    $\ket{\lambda1}$ and $\ket{\lambda_{-1}}$ are orthogonal so $\ket{\lambda_1}\bra{\lambda_1} + \ket{\lambda_{-1}}\bra{\lambda_{-1}} = \mathbb{I} $

    \item Show that $\frac{d}{dt} \exp(At) = A\exp(At)$
    $$ \frac{d}{dt} \exp(At) = \frac{d}{dt} \bigg[I+tA+\frac{1}{2}t^2A^2 + \frac{1}{3!}t^3A^3 +\ldots\bigg]$$
    $$ = A + tA^2 + \frac{1}{2} t^2 A^3 + \ldots $$ 
    $$ = A\exp(At) $$
    Since $t \in \mathbb{R} $

\end{enumerate}


\item Exercise 7
%Question 7
\begin{enumerate}
    \item BBGKY Hierarchy.
    \\ First of all the boundary condition for the n-particle density functional 
    $$ f_n(q_1 \ldots q_n,p_1 \ldots p_n ) = \int f_N (q_1 \ldots q_N,p_1 \ldots p_N) dq_{n+1} \ldots dq_N dp_{n+1} \ldots dp_N $$ 
    Can be written as 
    $$ f_n(q_1 \ldots q_n,p_1 \ldots p_n ) = \frac{N!}{(N-n!)} \int f_N(q,p) \prod_{i=1}^{N-n} dq_i dp_i$$
    There are $\frac{N!}{(N-n!)}$ ways to choose $n$ particles out of $N$ total particles if the particles are identical. 
    \\ Now integrate the Liouville equation:
    $$ \frac{\partial f_N}{\partial t} + \sum_{i=1}^N \frac{p_i}{m} \cdot \frac{\partial f_N}{\partial q_i} + \sum_{i=1}^N \vec{F_i} \cdot \frac{\partial f_N}{\partial p_i} =0 $$
    $$ \frac{\partial f_N}{\partial t} + \sum_{i=1}^N \frac{p_i}{m} \cdot \frac{\partial f_N}{\partial q_i} + \sum_{i=1}^N \vec{F_i^{ext}} \cdot \frac{\partial f_N}{\partial p_i} + \sum_{i=1}^N \sum_{j=1}^N \vec{F_{ij}} \cdot \frac{\partial f_N}{\partial p_i} = 0 $$
    Where $\vec{F_i^{ext}} $ is the external force and $\vec{F_{ij}}$ is the pairwise interaction between particles. 
    \\ Integrating w.r.t $n+1 \rightarrow N$ 
    $$ \int \frac{\partial f_N}{\partial t} + \sum_{i=1}^N \frac{p_i}{m} \cdot \frac{\partial f_N}{\partial q_i} + \sum_{i=1}^N \vec{F_i^{ext}} \cdot \frac{\partial f_N}{\partial p_i} \prod_{i=n}^{N-n} dq_i dp_i $$
    $$ + \int \sum_{i=1}^N \sum_{j=1}^N \vec{F_{ij}} \cdot \frac{\partial f_N}{\partial p_i} \prod_{i=n}^{N-n} dq_i dp_i =0 $$
    Split the integral up and move the second part to the other side of the equals sign. 
    $$ \int \frac{\partial f_N}{\partial t} + \sum_{i=1}^N \frac{p_i}{m} \cdot \frac{\partial f_N}{\partial q_i} \sum_{i=1}^N \vec{F_i^{ext}} \cdot \frac{\partial f_N}{\partial p_i} \prod_{i=n}^{N-n} dq_i dp_i  $$
    $$ = -  \int \sum_{i=1}^N \sum_{j=1}^N \vec{F_{ij}} \cdot \frac{\partial f_N}{\partial p_i} \prod_{i=n}^{N-n} dq_i dp_i $$
    Recognising that inside the integral every term where $i > n$ will have a probability density of 0. This comes from the fact that we are integrating over differentiable objects as shown by:
    $$ \int \frac{\partial f_N}{\partial p_i} dp_{N-n} = f_N \Big|_{-\infty}^{+\infty} $$
    This means that there is zero probability density outside of the boundary of the 'box' which the particles are contained, this comes from the derivation of the Liouville equation, meaning that the term is exactly 0. 
    \\
    For the LHS (Left hand side) of the equation, this is easy, as we only have index $i$ in the summation and therefore every DoF (degree of freedom) we integrate over vanishes. 
    \\ 
    The RHS has index $j$ in the sums as well as $i$ which makes it difficult to simplify. For now we keep it in:
    $$ \frac{\partial f_n}{\partial t} + \sum_{i=1}^n \frac{p_i}{m} \cdot \frac{\partial f_n}{\partial q_i} + \sum_{i=1}^n \vec{F_{i}^{ext}} \cdot \frac{\partial f_n}{\partial p_i} = - \sum_{i=1}^n \sum_{j=1}^N \int \vec{F_{ij}} \cdot \frac{\partial f_N}{\partial p_i} \prod_{i=n}^{N-n} dq_i dp_i $$
    Replaced $f_N \rightarrow f_n$ on the LHS where necessary as well as some of the summations for the above reasons (vanishes). For the RHS we can now split into two parts:
    $$ \frac{\partial f_n}{\partial t} + \sum_{i=1}^n \frac{p_i}{m} \cdot \frac{\partial f_n}{\partial q_i} + \sum_{i=1}^n \vec{F_{i}^{ext}} \cdot \frac{\partial f_n}{\partial p_i} $$
    $$ = - \Big[\sum_{i=1}^n \sum_{j=1}^n \int \vec{F_{ij}} \cdot \frac{\partial f_N}{\partial p_i} \prod_{i=n}^{N-n} dq_i dp_i + \sum_{i=1}^n \sum_{j=n+1}^N \int \vec{F_{ij}} \frac{\partial f_N}{\partial p_i} \prod_{i=n}^{N-n} dq_i dp_i \Big] $$
    The first $\vec{F_{ij}}$ term can be taken out of the integral and moved to the LHS preserving the summation:
    $$ \frac{\partial f_n}{\partial t} + \sum_{i=1}^n \frac{p_i}{m} \cdot \frac{\partial f_n}{\partial q_i} + \sum_{i=1}^n \Big(\vec{F_i^{ext}} + \sum_{j=1}^n \vec{F_{ij}}\Big) \cdot \frac{\partial f_n}{\partial p_i} $$
    $$ = - \sum_{i=1}^n \sum_{j=n+1}^N \int \vec{F_{ij}} \cdot \frac{\partial f_N}{\partial p_i} \prod_{i=n}^{N-n} dq_i dp_i $$
    Now recognising that the $\sum_{j=n+1}^N$ can be simplified to $(N-n)$ as permutation of the particles makes no difference as they're identical. 
    $$ \frac{\partial f_n}{\partial t} + \sum_{i=1}^n \frac{p_i}{m} \cdot \frac{\partial f_n}{\partial q_i} + \sum_{i=1}^n \Big(\vec{F_i^{ext}} + \sum_{j=1}^n \vec{F_{ij}}\Big) \cdot \frac{\partial f_n}{\partial p_i} $$ 
    $$ = - (N-n) \sum_{i=1}^n \int \vec{F_{i,n+1}} \cdot \frac{\partial f_{n+1}}{\partial p_i} \prod_{i=n}^{N-n} dq_i dp_i $$
    Substitution of 
    $$\vec{F_{i}^{ext}} = - \frac{\partial \Phi^{ext}}{\partial q_i} $$
    $$ \vec{F_{ij}} = - \frac{\partial \Phi_{ij}}{\partial q_i} $$
    Gives
    $$ \frac{\partial f_n}{\partial t} + \sum_{i=1}^n \frac{p_i}{m} \cdot \frac{\partial f_n}{\partial q_i} - \sum_{i=1}^n \Big(\frac{\partial \Phi^{ext}}{\partial q_i}+ \sum_{j=1}^n \frac{\partial \Phi_{ij}}{\partial q_i} \Big) \cdot \frac{\partial f_n}{\partial p_i} $$
    $$ = (N-n) \sum_{i=1}^n \int \frac{\partial \Phi_{i,n+1}}{\partial q_i}\frac{\partial f_{n+1}}{\partial p_i} \prod_{i=n}^{N-n} dq_i dp_i $$
    Which simplifies to
    $$ \frac{\partial f_n}{\partial t} + \sum_{i=1}^n \frac{p_i}{m} \cdot \frac{\partial f_n}{\partial q_i} - \sum_{i=1}^n \Big( \sum_{j=1\neq i}^n \frac{\partial \Phi_{ij}}{\partial q_i} + \frac{\partial \Phi_i^{ext}}{\partial q_i} \Big) \cdot \frac{\partial f_n}{\partial p_i} $$
    $$ = (N-n) \sum_{i=1}^n \int \frac{\partial \Phi_{i,n+1}}{\partial q_i}\frac{\partial f_{n+1}}{\partial p_i} dq_{n+1} dp_{n+1} $$
    This gives the $n$ particle density function.
    \\ For the 1 particle density function we simply substitute $n=1$ and simplify
    $$ \frac{\partial f_1}{\partial t} + \sum_{i=1}^1 \frac{p_i}{m} \cdot \frac{\partial f_1}{\partial q_i} - \sum_{i=1}^1 \Big( \sum_{j=1\neq i}^1 \frac{\partial \Phi_{ij}}{\partial q_i} + \frac{\partial \Phi_i^{ext}}{\partial q_i} \Big) \cdot \frac{\partial f_1}{\partial p_i} $$
    $$ = (N-1) \sum_{i=1}^1 \int \frac{\partial \Phi_{i,2}}{\partial q_i}\frac{\partial f_{2}}{\partial p_i} dq_{2} dp_{2} $$
    Giving 
    $$  \frac{\partial f_1}{\partial t} + \frac{p_1}{m} \cdot \frac{\partial f_1}{\partial q_1} -  \frac{\partial \Phi^{ext}}{\partial q_1} \Big) \cdot \frac{\partial f_1}{\partial p_1} $$
    $$  = (N-1) \int \frac{\partial \Phi_{1,2}}{\partial q_1}\frac{\partial f_{2}}{\partial p_1} dq_{2} dp_{2} $$

\end{enumerate}


\end{enumerate}


\end{document}

