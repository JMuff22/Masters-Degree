
\documentclass[12pt]{article}
\usepackage[finnish]{babel}
\usepackage[T1]{fontenc}
\usepackage[utf8]{inputenc}
\usepackage{delarray,amsmath,bbm,epsfig,slashed}
\newcommand{\pat}{\partial}
\newcommand{\be}{\begin{equation}}
\newcommand{\ee}{\end{equation}}
\newcommand{\bea}{\begin{eqnarray}}
\newcommand{\eea}{\end{eqnarray}}
\newcommand{\abf}{{\bf a}}
\newcommand{\Zmath}{\mathbf{Z}}
\newcommand{\Zcal}{{\cal Z}_{12}}
\newcommand{\zcal}{z_{12}}
\newcommand{\Acal}{{\cal A}}
\newcommand{\Fcal}{{\cal F}}
\newcommand{\Ucal}{{\cal U}}
\newcommand{\Vcal}{{\cal V}}
\newcommand{\Ocal}{{\cal O}}
\newcommand{\Rcal}{{\cal R}}
\newcommand{\Scal}{{\cal S}}
\newcommand{\Lcal}{{\cal L}}
\newcommand{\Hcal}{{\cal H}}
\newcommand{\hsf}{{\sf h}}
\newcommand{\half}{\frac{1}{2}}
\newcommand{\Xbar}{\bar{X}}
\newcommand{\xibar}{\bar{\xi }}
\newcommand{\barh}{\bar{h}}
\newcommand{\Ubar}{\bar{\cal U}}
\newcommand{\Vbar}{\bar{\cal V}}
\newcommand{\Fbar}{\bar{F}}
\newcommand{\zbar}{\bar{z}}
\newcommand{\wbar}{\bar{w}}
\newcommand{\zbarhat}{\hat{\bar{z}}}
\newcommand{\wbarhat}{\hat{\bar{w}}}
\newcommand{\wbartilde}{\tilde{\bar{w}}}
\newcommand{\barone}{\bar{1}}
\newcommand{\bartwo}{\bar{2}}
\newcommand{\nbyn}{N \times N}
\newcommand{\repres}{\leftrightarrow}
\newcommand{\Tr}{{\rm Tr}}
\newcommand{\tr}{{\rm tr}}
\newcommand{\ninfty}{N \rightarrow \infty}
\newcommand{\unitk}{{\bf 1}_k}
\newcommand{\unitm}{{\bf 1}}
\newcommand{\zerom}{{\bf 0}}
\newcommand{\unittwo}{{\bf 1}_2}
\newcommand{\holo}{{\cal U}}
%\newcommand{\bra}{\langle}
%\newcommand{\ket}{\rangle}
\newcommand{\muhat}{\hat{\mu}}
\newcommand{\nuhat}{\hat{\nu}}
\newcommand{\rhat}{\hat{r}}
\newcommand{\phat}{\hat{\phi}}
\newcommand{\that}{\hat{t}}
\newcommand{\shat}{\hat{s}}
\newcommand{\zhat}{\hat{z}}
\newcommand{\what}{\hat{w}}
\newcommand{\sgamma}{\sqrt{\gamma}}
\newcommand{\bfE}{{\bf E}}
\newcommand{\bfB}{{\bf B}}
\newcommand{\bfM}{{\bf M}}
\newcommand{\cl} {\cal l}
\newcommand{\ctilde}{\tilde{\chi}}
\newcommand{\ttilde}{\tilde{t}}
\newcommand{\ptilde}{\tilde{\phi}}
\newcommand{\utilde}{\tilde{u}}
\newcommand{\vtilde}{\tilde{v}}
\newcommand{\wtilde}{\tilde{w}}
\newcommand{\ztilde}{\tilde{z}}

\newtheorem{theorem}{Theorem}

% David Weir's macros


\newcommand{\nn}{\nonumber}
\newcommand{\com}[2]{\left[{#1},{#2}\right]}
\newcommand{\mrm}[1] {{\mathrm{#1}}}
\newcommand{\mbf}[1] {{\mathbf{#1}}}
\newcommand{\ave}[1]{\left\langle{#1}\right\rangle}
\newcommand{\halft}{{\textstyle \frac{1}{2}}}
\newcommand{\ie}{{\it i.e.\ }}
\newcommand{\eg}{{\it e.g.\ }}
\newcommand{\cf}{{\it cf.\ }}
\newcommand{\etal}{{\it et al.}}
\newcommand{\ket}[1]{\vert{#1}\rangle}
\newcommand{\bra}[1]{\langle{#1}\vert}
\newcommand{\bs}[1]{\boldsymbol{#1}}
\newcommand{\xv}{{\bs{x}}}
\newcommand{\yv}{{\bs{y}}}
\newcommand{\pv}{{\bs{p}}}
\newcommand{\kv}{{\bs{k}}}
\newcommand{\qv}{{\bs{q}}}
\newcommand{\bv}{{\bs{b}}}
\newcommand{\ev}{{\bs{e}}}
\newcommand{\gv}{\bs{\gamma}}
\newcommand{\lv}{{\bs{\ell}}}
\newcommand{\nabv}{{\bs{\nabla}}}
\newcommand{\sigv}{{\bs{\sigma}}}
\newcommand{\notvec}{\bs{0}_\perp}
\newcommand{\inv}[1]{\frac{1}{#1}}
%\newcommand{\xv}{{\bs{x}}}
%\newcommand{\yv}{{\bs{y}}}
\newcommand{\Av}{\bs{A}}
%\newcommand{\lv}{{\bs{\ell}}}

%\newcommand\bsigma{\vec{\sigma}}
\hoffset 0.5cm
\voffset -0.4cm
\evensidemargin -0.2in
\oddsidemargin -0.2in
\topmargin -0.2in
\textwidth 6.3in
\textheight 8.4in

\begin{document}

\normalsize

\baselineskip 14pt

\begin{center}
{\Large {\bf Open Quantum Systems \ \ Fall 2020 \ \  Answers to Exercise Set 5}}\\
{\large { Jake Muff}}\\
{Student number: 015361763}\\
{5/11/2020}
\end{center}



\section{Random Phases}
$$ \psi = a \phi_1 + b \phi_2 $$
$a$ and $b$ have the condition that they must $|a|^2 = |b|^2 =1$. 
$$ \psi (t) = a e^{i \theta_1} \phi_1 + b e^{i \theta_2} \phi_2 $$
With probability 
$$ P(\theta_1, \theta_2) = \frac{1}{\sqrt{2 \pi \lambda_1 t}} \frac{1}{\sqrt{2 \pi \lambda_2 t}} e^{- \frac{\theta_1^2}{2 \lambda_1 t}}e^{- \frac{\theta_2^2}{2 \lambda_2 t}} $$
The density matrix can be written as an integral from a state vector $\psi (t)$ from the statistical description of the density matrix 
$$ \rho (t) = \sum_i p_i \ket{\psi_i}\bra{\psi_i} $$
$$ \rho (t) = \sum_i p_i \psi (t) \psi^{\dagger} (t) $$
$$ = \int_{- \infty}^{\infty} P(\theta_1, \theta_2) \psi (t) \psi^{\dagger} (t) $$
So we have 
$$ \psi (t) \psi^{\dagger} (t) = (ae^{i \theta_1} \phi_1 + be^{i \theta_2} \phi_2) \cdot (a^* e^{-i \theta_1} \phi_1^{\dagger} + b^* e^{-i \theta_2} \phi_2^{\dagger}  ) $$
$$ = |a|^2 \phi_1 \phi_1^{\dagger} + ab^* e^{i \theta_1 - i\theta_2} \phi_1 \phi_2^{\dagger} + ba^* e^{i \theta_2 - i \theta_1} \phi_2 \phi_1^{\dagger} + |b|^2 \phi_2 \phi_2^{\dagger} $$
\begin{enumerate}
    \item The density matrix at time $t$ is 
    $$ \rho (t) = \int_{- \infty}^{\infty} \int_{- \infty}^{\infty}  P(\theta_1, \theta_2) \psi (t) \psi^{\dagger} (t) d \theta_1 d \theta_2 $$
    $$ = \int_{- \infty}^{\infty} \int_{- \infty}^{\infty}  \frac{1}{\sqrt{2 \pi \lambda_1 t}} \frac{1}{\sqrt{2 \pi \lambda_2 t}} e^{- \frac{\theta_1^2}{2 \lambda_1 t}}e^{- \frac{\theta_2^2}{2 \lambda_2 t}} \ldots $$
    $$ \cdot \Big[ |a|^2 \phi_1 \phi_1^{\dagger} + ab^* e^{i \theta_1 - i\theta_2} \phi_1 \phi_2^{\dagger} + ba^* e^{i \theta_2 - i \theta_1} \phi_2 \phi_1^{\dagger} + |b|^2 \phi_2 \phi_2^{\dagger}  \Big] d \theta_1 d \theta_2 $$

    To make this look easier I introduce the substitutions $c = 2 \lambda_1 t$ and $d = 2 \lambda_2 t$ so that the equation looks like 
    $$ = \int_{- \infty}^{\infty} \int_{- \infty}^{\infty}  \frac{1}{\sqrt{ \pi c}} \frac{1}{\sqrt{ \pi d}} e^{- \frac{\theta_1^2}{ c}}e^{- \frac{\theta_2^2}{d}} \psi (t) \psi^{\dagger} (t) d \theta_1 d \theta_2 $$
    The gaussian part  of the integral is easily seen now such that the first part of the integral evaluates at 1. Also notice that the parts of $\psi (t) \psi^{\dagger} (t)$ which contribute to $\theta_1, \theta_2$ can be split up as such
    $$ a b^* (e^{i \theta_1 - i\theta_2}) = a b^* (e^{i \theta_1 }e^{- i\theta_2}) $$
    $$  ba^* (e^{i \theta_2 - i \theta_1}) = ba^* (e^{i \theta_2} e^{ - i \theta_1}) $$
    And that split up and evaluated through the integral also $=1$. So the solution is 
    $$ \rho (t) = \int_{- \infty}^{\infty} \int_{- \infty}^{\infty} P(\theta_1, \theta_2) \psi (t) \psi^{\dagger} (t) d \theta_1 d \theta_2 $$
    $$ = |a|^2 \phi_1 \phi_1^{\dagger} + |b|^2 \phi_2 \phi_2^{\dagger} \ldots $$

    \textbf{N.B} Not sure how the value of $e^{-\frac{1}{2} t(\lambda_1 + \lambda_2)}$ comes into the solution to the integral. 

    \item Show that $\rho (t)$ satisfies the master equation 
    $$ \rho (t) = |a|^2 \phi_1 \phi_1^{\dagger} + |b|^2 \phi_2 \phi_2^{\dagger} + e^{-\frac{1}{2} t(\lambda_1 + \lambda_2)}(ab^* \phi_1 \phi_2^{\dagger}+ba^*\phi_2 \phi_1^{\dagger}) $$
    Taking the derivative of this w.r.t $t$ 
    $$ \frac{d}{dt} \rho (t) = - \frac{(\lambda_1 + \lambda_2)(ba^* \phi_2 \phi_1^{\dagger} + ab^* \phi_1 \phi_2^{\dagger})e^{-\frac{1}{2} t(\lambda_1 + \lambda_2)}}{2} $$
    $$ = - \frac{1}{2} (\lambda_1 + \lambda_2) \Big[ \rho (t) - (\phi_1 \phi_1^{\dagger} \rho (t) \phi_1 \phi_1^{\dagger} + \phi_2 \phi_2^{\dagger} \rho (t) \phi_2 \phi_2^{\dagger})\Big] $$
    $$ = - \frac{1}{2} (\lambda_1 + \lambda_2) \Big[ \rho (t) - \sum_i \phi_i \phi_i^{\dagger} \rho (t) \phi_i \phi_i^{\dagger} \Big] $$
    Expanding and splitting $- \frac{1}{2} (\lambda_1 + \lambda_2)$ so that 
    $$ \Rightarrow \sqrt{\frac{1}{2} (\lambda_1 + \lambda_2)ß} \phi_i \phi_i^{\dagger} \cdot -\sqrt{\frac{1}{2} (\lambda_1 + \lambda_2)} \phi_i \phi_i^{\dagger} \equiv - \frac{1}{2} (\lambda_1 + \lambda_2)$$
    So the first term can be written as 
    $$ \sqrt{\frac{1}{2} (\lambda_1 + \lambda_2)} \phi_i \phi_i^{\dagger} \ \rho (t) \ (\sqrt{\frac{1}{2} (\lambda_1 + \lambda_2)} \phi_i \phi_i^{\dagger})^{\dagger} $$
    Where 
    $$ (\sqrt{\frac{1}{2} (\lambda_1 + \lambda_2)} \phi_i \phi_i^{\dagger})^{\dagger} = -\sqrt{\frac{1}{2} (\lambda_1 + \lambda_2)} \phi_i^{\dagger} \phi_i $$
    The second term can also be written but it has 4 terms in (from $i=1$ to $2$) so we have 
    $$ \frac{1}{2} (\lambda_1 + \lambda_2) \rho (t) + \rho (t) \frac{1}{2} (\lambda_1 + \lambda_2) $$ 
    Which is the anti commutation relation with an extra value of 2 included from which the prefactor $\frac{1}{2}$ outside of the commutation brackets to be able to use this simplification. So we have 
    $$ \frac{d}{dt} \rho (t) = \sqrt{\frac{1}{2} (\lambda_1 + \lambda_2)} \phi_i \phi_i^{\dagger} \rho (t) \cdot -\sqrt{\frac{1}{2} (\lambda_1 + \lambda_2)} \phi_i^{\dagger}\phi_i $$
    $$ - \frac{1}{2}(\frac{1}{2} (\lambda_1 + \lambda_2) \rho(t) + \rho (t) \frac{1}{2} (\lambda_1 + \lambda_2)) $$
    $$ = \sum_i \Big[ \sqrt{\frac{1}{2} (\lambda_1 + \lambda_2)} \phi_i \phi_i^{\dagger} \rho (t) \cdot -\sqrt{\frac{1}{2} (\lambda_1 + \lambda_2)} \phi_i^{\dagger}\phi_i $$
    $$ - \frac{1}{2}\{-\sqrt{\frac{1}{2} (\lambda_1 + \lambda_2)} \phi_i^{\dagger} \phi_i \sqrt{\frac{1}{2} (\lambda_1 + \lambda_2)} \phi_i \phi_i^{\dagger}, \rho (t) \} \Big]$$
    Substituting 
    $$ L_i = \sqrt{\frac{1}{2} (\lambda_1 + \lambda_2)} \phi_i  \phi_i^{\dagger} $$
    So 
    $$ L_i^{\dagger} =  -\sqrt{\frac{1}{2} (\lambda_1 + \lambda_2)} \phi_i^{\dagger} \phi_i $$
    Gives 
    $$ \frac{d}{dt} \rho (t) = \sum_i \Big[ L_i \rho (t) L_i^{\dagger} - \frac{1}{2} \{L_i^{\dagger}L_i, \rho (t)\}\Big] $$
    \\
    \textbf{Note}. $ \sum_i^2 \phi_i^{\dagger} \phi_i \phi_i \phi_i^{\dagger} = 1$



\end{enumerate}
\section{Unitary Jump}
\begin{enumerate}
    \item Density operator at $t +dt$. $\psi (t)$ has probability $P= \lambda dt$ to jump to $e^{-i G} \psi (t) $. So it has $1-\lambda dt$ of staying $\psi (t)$.
    For normal time evolution we have 
    $$ \rho (t) = P e^{-i Ht} \rho(0) e^{iHt} $$
    Where $\rho (0)$ is the initial density matrix. For this the initial density matrix is $\rho (t)$ and we need to add on the probability that the state is unchanged to satisfy the total probability.
    $$ \rho (t +dt) = \mathrm{Probability \ to \ remain \ unchanged} \cdot \rho (t) + \mathrm{Probability\ to \ change} \cdot e^{-iG} \rho (0=t) e^{iG} $$
    $$ \rho (t +dt) = (1-\lambda dt) \rho (t) + \lambda dt e^{-iG} \rho (t) e^{iG} $$

    \item Show that $\rho (t)$ satisfies the differential equation 
    $$ \frac{d}{dt} \rho (t) = \frac{d}{dt} \Big[ (1-\lambda dt) \rho (t) + \lambda t e^{-iG} \rho (t) e^{iG} \Big] $$
    $$ = - \lambda \rho (t) + \lambda e^{-iG} \rho (t) e^{iG} $$ 
    $$ = - \lambda \Big[ \rho (t) - e^{-iG} \rho (t) e^{iG} \Big] $$
    To solve the time evolution we use the equation for $\rho (t + dt)$ and set $dt = 0$.
    
    \item Finding $L$ in the master equation form given.
    $$ \frac{d}{dt} \rho (t) = L \rho (t) L^{\dagger} - \frac{1}{2} \{L^{\dagger}L, \rho (t) \} $$
    Setting this equal to the equation in the previous equation 
    $$  L \rho (t) L^{\dagger} - \frac{1}{2} \{L^{\dagger}L, \rho (t) \} = - \lambda \Big[ \rho (t) - e^{-iG} \rho (t) e^{iG} \Big] $$
    By directly comparing :
    $$ L \rho (t) L^{\dagger} \rightarrow \lambda e^{-iG} \rho (t) e^{iG} $$
    So that 
    $$ L = \sqrt{2 \lambda} \cdot e^{-iG} $$
    Because 
    $$ L^{\dagger} = (\sqrt{2 \lambda} \cdot e^{-iG} )^{\dagger} = \sqrt{2 \lambda} e^{iG} $$
    With the 2 being there so that when expanded you get $2 \lambda$ which multiplied by $\frac{1}{2}$ gives just $\lambda$ which we were looking for. 
    \item Showing that the off diagonal elements satisfy the differential equation. We have 
    $$ E_1 = g_1 \ , \ E_2 = g_2 $$
    Where $E_i$ are eigenvalues so $\ket{E_i}$ are eigenvectors. 
    $$ \ket{E_1} = \ket{g_1} \ , \ \ket{E_2} = \ket{g_2} $$
    If the diagonals are $g_1$ and $g_2$ so we have 
    $$ \rho_{ii} (t) = \ket{g_i}^{\dagger} \rho (t) \ket{g_i} $$
    $$ \frac{d}{dt} \rho_{ii} (t) = 0 $$
    This is because $\ket{g_i}^{\dagger} \rho (t) \ket{g_i} $ will always equal a constant and the differential of a constant is 0. 
    For $\rho_{12} (t)$ 
    $$ \rho_{12} (t) = \ket{g_1}^{\dagger} \rho(t) \ket{g_2} $$
    From part (b) 
    $$ \frac{d}{dt} \rho_{12} (t) = -\lambda [ \rho_{12} (t) - e^{-i g_1} \rho_{12} (t) e^{ig_2} ] $$
    $$ = -\lambda \rho_{12} (t) + \lambda \rho_{12} (t) e^{i(g_2-g_1)} $$
    $$ = -\lambda \rho_{12} (t) [ 1- e^{i(g2-g1)}] $$




\end{enumerate}

\section{Random Unitary transformation} 
In a time dt 
$$ \psi (t +dt ) = e^{-iG \theta} \psi (t) $$
With probability 
$$ P(\theta) = \frac{1}{\sqrt{2 \pi \lambda dt}} e^{-\frac{\theta^2}{2 \lambda dt}} $$
\begin{enumerate}
    \item Find density matrix at time $t +dt $. First need to show that order $\theta^3$ and higher can be neglected. We can Taylor expand 
    $$ e^{-iG \theta} \psi(t) $$ 
    Around $\theta$. So we get 
    $$ e^{-iG \theta} - iG \theta e^{-i G \theta} - \frac{1}{2} G^2 \theta^2 e^{-iG \theta} + \frac{1}{6} iG^3 \theta^3 e^{-iG \theta} \ldots $$
    Lets also expand $e^{iG \theta}$ around $\theta$ 
    $$ e^{iG \theta} + iG \theta e^{iG \theta} - \frac{1}{2} G^2 \theta^2 e^{iG \theta} - \frac{1}{6} i G^3 \theta^3 e^{iG \theta} \ldots $$
    So $e^{-iG \theta} \rho (t) e^{iG \theta} $ expanded in $\theta$ will have terms of order $\theta^3$ and higher cancel out as the signs will be different. Can also say that that for $t$ we have $e^{-i G \theta}$ and for $dt$ we have $ e^{-iG \theta} \rho (t) e^{iG \theta}$ so naturally $dt$ will always have 1 order higher of $\theta$ past $\theta^3$. 
    \\
    For $\rho (t+dt)$ it is then a simpler version of Part 1.
    $$ \rho (t + dt ) = \int_{-\infty}^{\infty} P(\theta) e^{-iG \theta} \rho (t) e^{-iG \theta} $$ 
    Using my expansions I used mathematica to evaluate $e^{-iG \theta} \rho (t) e^{-iG \theta}$
    $$ e^{-iG \theta} \rho (t) e^{-iG \theta} = \frac{4 \rho(t) + G^4 \theta^4 \rho(t) -8 G^2 \theta^2 \rho(t)}{4} + iG^3 \theta^3 \rho (t) - 2iG \theta \rho (t) $$
    Obviously this ignored the commuatative effect so it needed to be split up as 
    $$ G \rho (t) G \neq G^2 \rho (t) $$
    $$ \frac{1}{2} G^2 \theta^2 \rho (t) \neq \rho (t) \frac{1}{2} G^2 \theta^2 $$
    And ignoring terms $\theta^3$ or higher 
    $$ = \rho (t) - \frac{1}{2} G^2 \theta^2 \rho (t) - \frac{1}{2}G^2 \theta^2 + G \rho (t) G \theta^2 $$
    So we have 
    $$ \rho (t + dt) = \int_{-\infty}^{\infty} P(\theta) \Big[ \rho (t) - \frac{1}{2} G^2 \theta^2 \rho (t) - \rho (t) \frac{1}{2} G^2 \theta^2 + G \rho (t) G \theta^2] $$
    Like in question 1 we have a guassian integral from $P(\theta)$ so the solution is 
    $$ \rho (t + dt) = \rho (t) - \frac{\theta^3}{2} \Big[ G^2 \rho(t) + \rho (t) G^2 - 2G \rho (t) G \Big] $$
    And substituting $\theta^3 = \lambda dt$ 
    $$ \rho (t +dt ) = \rho (t) - \frac{\lambda dt}{2} \Big[ G^2 \rho (t) + \rho (t) G^2 - 2G \rho (t) G \Big] $$

    \item Finding $L$ so that the derivative satisfies the master equation. Evaluating the derivative like in question 2 we can find the derivative from the equation above with $dt =t$
    $$ \frac{d}{dt} = \frac{d}{dt} \Big[\rho (0) - \frac{\lambda t}{2} \Big[ G^2 \rho (t) + \rho (t) G^2 - 2G\rho (t) G \Big] \Big] $$
    $$ = 0 - \frac{\lambda}{2} \Big[ G^2 \rho (t) + \rho (t) G^2 - 2G \rho (t) G \Big] $$
    Which can be simplified to 
    $$ = -\frac{\lambda}{2} G \{G, \rho (t) \} + \lambda G \rho (t) G $$
    As we can taker out a factor of $G$ outside the brackets then inside the brackets is the commutation relation + an extra factor to bring it back to the original. Therefore 
    $$ L = \sqrt{\lambda} G $$

    \item Showing that the components of density operator in basis of eigenvectors of $G$ satify the differential equation. Lie before in question 2d, for the off diagonal elements they have values of $g_1$ and $g_2$ ($g_i, g_j$). Substituting $g_i$ and $g_j$ into the differential equation above ( $g_i$ is left side multiplier of $G$ and $g_j$ is right side multiplier of $G$)
    $$ \frac{d}{dt} \rho_{ij} (t) = - \frac{\lambda}{2} \Big[ g_i^2 \rho_{ij} (t) + \rho_{ij} (t) g_j^2 - 2g_i \rho_{ij} (t) g_j  \Big] $$
    $$ = - \frac{\lambda}{2} \Big[ g_i^2 + g_j^2 - 2g_i g_j \Big] \rho{ij} (t) $$
    $$ = - \frac{\lambda}{2} (g_i - g_j)^2 \rho{ij} (t) $$


\end{enumerate}

\section{State Exchange} 
Two orthonormal vectors such that 
$$ \psi (t) = a(t) \phi_1 + b(t) \phi_2 $$
In time $dt$ with probability $\lambda dt$, $\psi (t)$ undergoes 
$$ \psi (t) \rightarrow a (t) \phi_2 b(t) \phi_1 $$
\begin{enumerate}
    \item Showing that the state operator satisfies the master equation in the canonical pauli $x$ basis. To begin with ( from Unitary Jumpy question) we have 
    $$ \rho (t + dt) = ( 1- \lambda dt) \rho (t) + \lambda dt \sigma_x \rho (t) \sigma_x $$
    So the derivative of this is 
    $$ \frac{d}{dt} \rho (t) = - \lambda \Big[ \rho (t) - \sigma_x \rho (t) \sigma_x \Big] $$
    $$ = - \lambda \rho (t) + \lambda \sigma_x \rho (t) \sigma_x $$
    However, this implies that $L = \sqrt{2 \lambda} \sigma_x$, so for $L = \sqrt{ \lambda} \sigma_x$ the differential equation would be 
    $$ \frac{d}{dt} \rho (t) = - \frac{\lambda}{2} \Big[ \rho (t) - \sigma_x \rho (t) \sigma_x \Big] $$
    Whiich satisfies the master equation as the master equation expanded gives 
    $$ \sqrt{\lambda} \sigma_x \rho (t) (\sqrt{\lambda}\sigma_x)^{\dagger} - \frac{1}{2} ( \sqrt{\lambda} \sigma_x (\sqrt{\lambda} \sigma_x )^{\dagger} \rho (t) + \rho(t) \sqrt{\lambda} \sigma_x ( \sqrt{\lambda} \sigma_x)^{\dagger} ) $$
    And $\sigma_x$ is in the basis $\phi_{1,2}$ such that it is in the same form as the previous lindblad equations.

    \item The Pauli matrices have eigenvalues of $+1$ and $-1$ which implies for that opposite diagonals have different signs. 
    
\end{enumerate}

\section{The Lindblad Equation}
\begin{enumerate}
    \item Showing that the Lindblad equation is trace preserving. 
    $$ \frac{d}{dt} \Tr \rho (t) = -i \Tr([H, \rho(t)]) + \sum_i \Big[ \Tr(L_i \rho (t) L_i^{\dagger} ) - \frac{1}{2} \Tr ( \{ L_i^{\dagger} L_i, \rho (t) \}) \Big] $$
    $$ = -i \Tr ([H, \rho (t)]) + \sum_i \Big[ \Tr(L_i \rho (t) L_i^{\dagger} ) - \frac{1}{2} \Tr ( L_i^{\dagger} L_i \rho (t) ) - \frac{1}{2} \Tr ( \rho (t) L_i^{\dagger} L_i) \Big] $$
    Using the cyclic property of traces where 
    $$ \Tr (ABC) = \Tr(ACB) = \Tr(BAC) $$ 
    The second part of the equation becomes 
    $$ \ldots + \sum_i \Big[ \Tr (ABC) - \frac{1}{2} \Tr (ACB) - \frac{1}{2} \Tr (BAC) \Big] $$
    $$ \Rightarrow 0 $$
    And because $H=0$ 
    $$ -i \Tr ( [H, \rho (t) ]) = -i \Tr(H, \rho (t) , \rho (t) H) = -i \Tr (0) = 0 $$
    And we have 
    $$ \frac{d}{dt} \Tr \rho (t) = 0 $$

    \item Show that the state operator is valid at all times. 
    \\
    $\rho (t) $ is hermitian (self adjoint) due to 
    $$ \rho (t) ^{\dagger} = \Big( \sum_i M_i (t) \rho_0 M_i^{\dagger} (t) \Big)^{\dagger} $$
    $$ = \sum_i M_i \rho_0^{\dagger} M_i^{\dagger} = \rho (t) $$
    For a longer (in my opinion more rigorous method) prove the hermicity from the full Lindblad equation 
    $$ \rho (t + dt )^{\dagger} = \rho (t) ^{\dagger} + dt(\frac{d}{dt} \rho (t)) $$
    $$ = \rho (t) + dt \Big[  -i [H, \rho(t)] + \sum_i \Big[ L_i \rho (t) L_i^{\dagger}  - \frac{1}{2}   \{ L_i^{\dagger} L_i, \rho (t) \} \Big]  \Big]^{\dagger} $$
    $$ = \rho (t) + dt \Big[  -i [\rho(t), H] + \sum_i \Big[ (L_i \rho (t) L_i^{\dagger})^{\dagger}  - \frac{1}{2} (L_i^{\dagger} L_i \rho(t))^{\dagger}  - \frac{1}{2} (\rho (t) L_i^{\dagger} L_i)^{\dagger}  \Big] $$
    $$ = \rho (t) + dt \Big[  -i [H, \rho(t)] + \sum_i \Big[ L_i \rho (t) L_i^{\dagger}  - \frac{1}{2}   \{ L_i^{\dagger} L_i, \rho (t) \} \Big]  \Big]^{\dagger} $$
    $$ = \rho (t +dt ) $$
    \\
    For trace of 1 we have 
    $$ \Tr \rho (t) = \Tr \Big( \sum_i M_i \rho_0 M_i^{\dagger} \Big) $$
    $$ = \Tr \Big( \rho_0 \sum_i M_i M_i^{\dagger} \Big) = \Tr(\rho_0) = 1$$
    \\
    For semi-positive definite 
    $$ \rho (t) = \sum_i M_i (t) \rho_0 M_i^{\dagger} (t) $$
    $$ = \sum_i \rho_0 \langle \psi | M_i \rangle \langle M_i | \psi \rangle $$
    $$ = \sum_i \rho_0 | \langle \psi | M_i \rangle |^2 \geq 0 $$
    For an arbitrary vector $\ket{\psi}$ in the state space, using the fact that $\Tr (\rho_0) =1$ 

\end{enumerate}


\end{document}

