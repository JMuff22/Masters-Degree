
\documentclass[12pt]{article}
\usepackage[english]{babel}
\usepackage[T1]{fontenc}
\usepackage[utf8]{inputenc}
\usepackage{delarray,amsmath,bbm,epsfig,slashed}
\newcommand{\pat}{\partial}
\newcommand{\be}{\begin{equation}}
\newcommand{\ee}{\end{equation}}
\newcommand{\bea}{\begin{eqnarray}}
\newcommand{\eea}{\end{eqnarray}}
\newcommand{\abf}{{\bf a}}
\newcommand{\Zmath}{\mathbf{Z}}
\newcommand{\Zcal}{{\cal Z}_{12}}
\newcommand{\zcal}{z_{12}}
\newcommand{\Acal}{{\cal A}}
\newcommand{\Fcal}{{\cal F}}
\newcommand{\Ucal}{{\cal U}}
\newcommand{\Vcal}{{\cal V}}
\newcommand{\Ocal}{{\cal O}}
\newcommand{\Rcal}{{\cal R}}
\newcommand{\Scal}{{\cal S}}
\newcommand{\Lcal}{{\cal L}}
\newcommand{\Hcal}{{\cal H}}
\newcommand{\hsf}{{\sf h}}
\newcommand{\half}{\frac{1}{2}}
\newcommand{\Xbar}{\bar{X}}
\newcommand{\xibar}{\bar{\xi }}
\newcommand{\barh}{\bar{h}}
\newcommand{\Ubar}{\bar{\cal U}}
\newcommand{\Vbar}{\bar{\cal V}}
\newcommand{\Fbar}{\bar{F}}
\newcommand{\zbar}{\bar{z}}
\newcommand{\wbar}{\bar{w}}
\newcommand{\zbarhat}{\hat{\bar{z}}}
\newcommand{\wbarhat}{\hat{\bar{w}}}
\newcommand{\wbartilde}{\tilde{\bar{w}}}
\newcommand{\barone}{\bar{1}}
\newcommand{\bartwo}{\bar{2}}
\newcommand{\nbyn}{N \times N}
\newcommand{\repres}{\leftrightarrow}
\newcommand{\Tr}{{\rm Tr}}
\newcommand{\tr}{{\rm tr}}
\newcommand{\ninfty}{N \rightarrow \infty}
\newcommand{\unitk}{{\bf 1}_k}
\newcommand{\unitm}{{\bf 1}}
\newcommand{\zerom}{{\bf 0}}
\newcommand{\unittwo}{{\bf 1}_2}
\newcommand{\holo}{{\cal U}}
%\newcommand{\bra}{\langle}
%\newcommand{\ket}{\rangle}
\newcommand{\muhat}{\hat{\mu}}
\newcommand{\nuhat}{\hat{\nu}}
\newcommand{\rhat}{\hat{r}}
\newcommand{\phat}{\hat{\phi}}
\newcommand{\that}{\hat{t}}
\newcommand{\shat}{\hat{s}}
\newcommand{\zhat}{\hat{z}}
\newcommand{\what}{\hat{w}}
\newcommand{\sgamma}{\sqrt{\gamma}}
\newcommand{\bfE}{{\bf E}}
\newcommand{\bfB}{{\bf B}}
\newcommand{\bfM}{{\bf M}}
\newcommand{\cl} {\cal l}
\newcommand{\ctilde}{\tilde{\chi}}
\newcommand{\ttilde}{\tilde{t}}
\newcommand{\ptilde}{\tilde{\phi}}
\newcommand{\utilde}{\tilde{u}}
\newcommand{\vtilde}{\tilde{v}}
\newcommand{\wtilde}{\tilde{w}}
\newcommand{\ztilde}{\tilde{z}}

% David Weir's macros


\newcommand{\nn}{\nonumber}
\newcommand{\com}[2]{\left[{#1},{#2}\right]}
\newcommand{\mrm}[1] {{\mathrm{#1}}}
\newcommand{\mbf}[1] {{\mathbf{#1}}}
\newcommand{\ave}[1]{\left\langle{#1}\right\rangle}
\newcommand{\halft}{{\textstyle \frac{1}{2}}}
\newcommand{\ie}{{\it i.e.\ }}
\newcommand{\eg}{{\it e.g.\ }}
\newcommand{\cf}{{\it cf.\ }}
\newcommand{\etal}{{\it et al.}}
\newcommand{\ket}[1]{\vert{#1}\rangle}
\newcommand{\bra}[1]{\langle{#1}\vert}
\newcommand{\bs}[1]{\boldsymbol{#1}}
\newcommand{\xv}{{\bs{x}}}
\newcommand{\yv}{{\bs{y}}}
\newcommand{\pv}{{\bs{p}}}
\newcommand{\kv}{{\bs{k}}}
\newcommand{\qv}{{\bs{q}}}
\newcommand{\bv}{{\bs{b}}}
\newcommand{\ev}{{\bs{e}}}
\newcommand{\gv}{\bs{\gamma}}
\newcommand{\lv}{{\bs{\ell}}}
\newcommand{\nabv}{{\bs{\nabla}}}
\newcommand{\sigv}{{\bs{\sigma}}}
\newcommand{\notvec}{\bs{0}_\perp}
\newcommand{\inv}[1]{\frac{1}{#1}}
%\newcommand{\xv}{{\bs{x}}}
%\newcommand{\yv}{{\bs{y}}}
\newcommand{\Av}{\bs{A}}
%\newcommand{\lv}{{\bs{\ell}}}

%\newcommand\bsigma{\vec{\sigma}}
\hoffset 0.5cm
\voffset -0.4cm
\evensidemargin -0.2in
\oddsidemargin -0.2in
\topmargin -0.2in
\textwidth 6.3in
\textheight 8.4in

\begin{document}

\normalsize

\baselineskip 14pt

\begin{center}
{\Large {\bf Open Quantum Systems \ \  Answers to Exercise Set 1 - JPBK }}\\
{\large { Jake Muff}}\\
jake.muff@helsinki.fi \\
{Student number: 015361763}\\
{25/09/2020}
\end{center}



\begin{enumerate}

\item Problem 1.1
%Question 1 Answer here
\begin{enumerate}
    \item Obtain 
    $$ I = \frac{1}{eR_T} \int d \epsilon \ n_L (\epsilon) n_R (\epsilon + eV)[f_l (\epsilon) - f_R (\epsilon + eV)]$$
    To do this we subsititute in the forward and backward tunneling rates into 
    \begin{equation} \label{19}
        I = e(\Gamma_f - \Gamma_b) 
    \end{equation}
    Noticing that the "transparancy" value is equal\cite{1} to 
    $$ |\tau | ^2 = \frac{1}{e^2 R_T} $$
    $$ I = e(|\tau|^2 \int d \epsilon \ n_L (\epsilon) f_L (\epsilon) n_R(\epsilon +eV)[1-f_R (\epsilon +eV)] $$
    $$ - |\tau |^2 \int d \epsilon \ n_R (\epsilon +eV) f_R (\epsilon +eV) n_L (\epsilon) [1-f_L (\epsilon)] ) $$ 
    $$ I = e |\tau|^2 \int n_L (\epsilon) f_L(\epsilon) n_R(\epsilon + eV) - n_L(\epsilon) n_R (\epsilon + eV) f_R(\epsilon + eV) d \epsilon $$
    $$ = e |\tau | ^2 \int d \epsilon \ n_L(\epsilon) n_R(\epsilon +eV)[f_L (\epsilon) - f_R (\epsilon + eV)] $$
    $$ = \frac{1}{eR_T} \int d \epsilon \ n_L(\epsilon) n_R(\epsilon +eV)[f_L (\epsilon) - f_R (\epsilon + eV)] $$

    

    
\end{enumerate}

\item Problem 1.2 \\
The left and right electrode temperatures have occuptation of energy levels as describes by the fermi distribution 
\begin{equation}
    f_{L,R} (\epsilon) = \frac{1}{1+ \exp ((\epsilon - \epsilon_{f,L(R)} ) / k_B T_{L,R}) } 
\end{equation}   

    So that we define 
    $$ \epsilon_{F,L(R)} = 0 \ \text{and} \ \epsilon_{f,R} = \epsilon - eV$$ 
    Applying electron hole symmetry we get equation 18 in the lecture notes. The density of states is approximately constant close to the fermi energy so we have $n_{L,R} (\epsilon) = 1$. \\
    Using the fermi distribution above and equations 18 from the lecture notes for the forward and backward tunneling rates we get 
    $$ \Gamma_{f,b} (V) = \frac{1}{e^2 R_T} \frac{\pm eV}{1-\exp (\mp eV / k_B T)} $$
    Which is the same as 
    $$ \Gamma = \frac{1}{e^2 R_T} \frac{ eV}{1-\exp (- eV \beta)} $$
    Assuming of course that $T_L = T_R = T = \frac{1}{k_B \beta}$ \\
    Subsituting the above equation with these assumptions into (\ref{19}) gives a linear relationship such that $I(V) = \frac{V}{R_T} $ showing that the junction is ohmic.





\item Problem 1.3
For problem 1.3 we are looking at a NIS junction at low temperatures where $eV, k_B T << \Delta $. For this we can use the Bardeen-Cooper-Schrieffer theory of superconductors where electrons form a condensate of cooper pairs that occupy the ground state below the critical temperature. Single electron excitations in the superconductor are seperated from the ground state by a gap labelled $\Delta$ such that 
\[ n_s (\epsilon) = n_s (0)
\begin{cases}
\frac{|\epsilon|}{\sqrt{\epsilon^2 - \Delta^2}}&\text{for $|\epsilon| > \Delta $}\\
0&\text{$|\epsilon| < \Delta$}\\
\end{cases}
\]
Where $n_s(0)$ is constant and equivalent to $n_R (\epsilon)$ in the notes. \\
To answer this lets change some notation and introduce \cite{2} a new factor $G_{NIN} $ to represent the conductance for the NIN junction that is independent of V. 
$$ I_{NIN} = \frac{1}{eR_T} \int d \epsilon \ n_L(\epsilon) n_R(\epsilon +eV)[f_L (\epsilon) - f_R (\epsilon + eV)] $$
$$ I_{NIN} = G_{NIN} V $$
We can make this relation as it is ohmic. Such that we can now write the tunnelign current for NIS as 
$$ I_{NIS} = \frac{G_{NIN}}{e} \int d \epsilon \ n_s (\epsilon) [f_s (\epsilon) - f_N (\epsilon + eV)] $$
And our conductance $G_{NIS}$ for the NIS junction is 
$$ G_{NIS} = \frac{d I_{NIS}}{d V} = G_{NIN} \int n_s (\epsilon) \Big[ -\frac{\partial f_N ( \epsilon + eV)}{\partial (eV)}\Big] d \epsilon $$
The conductance at $V=0$ is related to the width of the gap $\Delta$ through evaluating the derivatives over each other. We can now introduce our low temperature conditions. \\
For $k_B T << \Delta $ 
$$ \frac{G_{NIS}}{G_{NIN}} \Big|_{V=0} = \sqrt{\frac{2 \pi \Delta}{k_B T}} e^{- \Delta / k_b T} $$
For voltages $eV << \Delta$ this changes to 
$$ \sqrt{2 \pi \Delta k_B T } e^{-(\Delta - eV)/ k_B T} $$
Where $I_0 = \sqrt{2 \pi \Delta k_B T }$\\
We could take this further if we introduced the Dynes density of states, however I have kept to using the BCS DoS 




\end{enumerate}

\begin{thebibliography}{9}
    \bibitem{1}
    Anna Feshchenko
    \textit{Electron Themometry, refrigeration and heat transport in nanostructures at sub-kelvin temperatures}
    Doctoral Thesis, Department of Applied Physics, Low Temperature Laboratory, Aalto University
    \bibitem{2}
    M. Tinkham 
    Introduction to Superconductivity
    Dover Publications, 2004

\end{thebibliography}
\end{document}

