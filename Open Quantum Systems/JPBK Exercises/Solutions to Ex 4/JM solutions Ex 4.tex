
\documentclass[12pt]{article}
\usepackage[english]{babel}
\usepackage[T1]{fontenc}
\usepackage[utf8]{inputenc}
\usepackage{hyperref}
\usepackage{delarray,amsmath,bbm,epsfig,slashed}
\newcommand{\pat}{\partial}
\newcommand{\be}{\begin{equation}}
\newcommand{\ee}{\end{equation}}
\newcommand{\bea}{\begin{eqnarray}}
\newcommand{\eea}{\end{eqnarray}}
\newcommand{\abf}{{\bf a}}
\newcommand{\Zmath}{\mathbf{Z}}
\newcommand{\Zcal}{{\cal Z}_{12}}
\newcommand{\zcal}{z_{12}}
\newcommand{\Acal}{{\cal A}}
\newcommand{\Fcal}{{\cal F}}
\newcommand{\Ucal}{{\cal U}}
\newcommand{\Vcal}{{\cal V}}
\newcommand{\Ocal}{{\cal O}}
\newcommand{\Rcal}{{\cal R}}
\newcommand{\Scal}{{\cal S}}
\newcommand{\Lcal}{{\cal L}}
\newcommand{\Hcal}{{\cal H}}
\newcommand{\hsf}{{\sf h}}
\newcommand{\half}{\frac{1}{2}}
\newcommand{\Xbar}{\bar{X}}
\newcommand{\xibar}{\bar{\xi }}
\newcommand{\barh}{\bar{h}}
\newcommand{\Ubar}{\bar{\cal U}}
\newcommand{\Vbar}{\bar{\cal V}}
\newcommand{\Fbar}{\bar{F}}
\newcommand{\zbar}{\bar{z}}
\newcommand{\wbar}{\bar{w}}
\newcommand{\zbarhat}{\hat{\bar{z}}}
\newcommand{\wbarhat}{\hat{\bar{w}}}
\newcommand{\wbartilde}{\tilde{\bar{w}}}
\newcommand{\barone}{\bar{1}}
\newcommand{\bartwo}{\bar{2}}
\newcommand{\nbyn}{N \times N}
\newcommand{\repres}{\leftrightarrow}
\newcommand{\Tr}{{\rm Tr}}
\newcommand{\tr}{{\rm tr}}
\newcommand{\ninfty}{N \rightarrow \infty}
\newcommand{\unitk}{{\bf 1}_k}
\newcommand{\unitm}{{\bf 1}}
\newcommand{\zerom}{{\bf 0}}
\newcommand{\unittwo}{{\bf 1}_2}
\newcommand{\holo}{{\cal U}}
%\newcommand{\bra}{\langle}
%\newcommand{\ket}{\rangle}
\newcommand{\muhat}{\hat{\mu}}
\newcommand{\nuhat}{\hat{\nu}}
\newcommand{\rhat}{\hat{r}}
\newcommand{\phat}{\hat{\phi}}
\newcommand{\that}{\hat{t}}
\newcommand{\shat}{\hat{s}}
\newcommand{\zhat}{\hat{z}}
\newcommand{\what}{\hat{w}}
\newcommand{\sgamma}{\sqrt{\gamma}}
\newcommand{\bfE}{{\bf E}}
\newcommand{\bfB}{{\bf B}}
\newcommand{\bfM}{{\bf M}}
\newcommand{\cl} {\cal l}
\newcommand{\ctilde}{\tilde{\chi}}
\newcommand{\ttilde}{\tilde{t}}
\newcommand{\ptilde}{\tilde{\phi}}
\newcommand{\utilde}{\tilde{u}}
\newcommand{\vtilde}{\tilde{v}}
\newcommand{\wtilde}{\tilde{w}}
\newcommand{\ztilde}{\tilde{z}}

% David Weir's macros


\newcommand{\nn}{\nonumber}
\newcommand{\com}[2]{\left[{#1},{#2}\right]}
\newcommand{\mrm}[1] {{\mathrm{#1}}}
\newcommand{\mbf}[1] {{\mathbf{#1}}}
\newcommand{\ave}[1]{\left\langle{#1}\right\rangle}
\newcommand{\halft}{{\textstyle \frac{1}{2}}}
\newcommand{\ie}{{\it i.e.\ }}
\newcommand{\eg}{{\it e.g.\ }}
\newcommand{\cf}{{\it cf.\ }}
\newcommand{\etal}{{\it et al.}}
\newcommand{\ket}[1]{\vert{#1}\rangle}
\newcommand{\bra}[1]{\langle{#1}\vert}
\newcommand{\bs}[1]{\boldsymbol{#1}}
\newcommand{\xv}{{\bs{x}}}
\newcommand{\yv}{{\bs{y}}}
\newcommand{\pv}{{\bs{p}}}
\newcommand{\kv}{{\bs{k}}}
\newcommand{\qv}{{\bs{q}}}
\newcommand{\bv}{{\bs{b}}}
\newcommand{\ev}{{\bs{e}}}
\newcommand{\gv}{\bs{\gamma}}
\newcommand{\lv}{{\bs{\ell}}}
\newcommand{\nabv}{{\bs{\nabla}}}
\newcommand{\sigv}{{\bs{\sigma}}}
\newcommand{\notvec}{\bs{0}_\perp}
\newcommand{\inv}[1]{\frac{1}{#1}}
%\newcommand{\xv}{{\bs{x}}}
%\newcommand{\yv}{{\bs{y}}}
\newcommand{\Av}{\bs{A}}
%\newcommand{\lv}{{\bs{\ell}}}

%\newcommand\bsigma{\vec{\sigma}}
\hoffset 0.5cm
\voffset -0.4cm
\evensidemargin -0.2in
\oddsidemargin -0.2in
\topmargin -0.2in
\textwidth 6.3in
\textheight 8.4in

\begin{document}

\normalsize

\baselineskip 14pt

\begin{center}
{\Large {\bf Open Quantum Systems \ \  Answers to Problems 4.1 \& 4.2 - JPBK }}\\
{\large { Jake Muff}}\\
jake.muff@helsinki.fi \\
{Student number: 015361763}\\
{21/10/2020}
\end{center}


\section{Answers}
\begin{enumerate}

    \item Problem 4.1. Derive the backflow of heat due to S being at non-zero temperature. \\
    \\
    Because S is now included in the derivation we have 
    $$ \dot{Q}_{NIS}^N = \frac{1}{e^2 R_T} \int_{-\infty}^{\infty} d \epsilon \ (\epsilon - eV) \ n_s (\epsilon) \{f_N (\epsilon - eV) - f_S (\epsilon)\} $$
    $$ \dot{Q}^S_{NIS} = \frac{1}{e^2 R_T} \int_{-\infty}^{\infty} d \epsilon  \ \epsilon  \ n_s (\epsilon) \{f_N (\epsilon - eV) - f_S (\epsilon)\} $$
    So we consider $\dot{Q}^S_{NIS}$ as $\dot{Q}_{NIS}^N$ has already been derived. We also notice that this can be simplified using 
    $$ f_{N,S} (-\epsilon) = 1- f_{N,S} (\epsilon) \ ; \ n_s (-\epsilon) = \epsilon $$
    So we can write 
    $$ \dot{Q}^S_{NIS} = \frac{2}{e^2 R_T} \int_{\Delta}^{\infty} d \epsilon  \ \epsilon \ n_s (\epsilon) \{f_N (\epsilon - eV) - f_S (\epsilon)\} $$
    The fermi distribution $f_{N,S} (\epsilon)$ gives 
    $$ f_{N,S} (\epsilon) = \frac{1}{1+e^{\beta( \epsilon - \mu)}} $$
    And the density of states 
    $$ n_s (\epsilon) = \frac{\epsilon}{\sqrt{\epsilon^2 - \Delta^2}} $$
    Where $\beta = \frac{1}{k_B T}$. So we have 
    $$ \dot{Q}^S_{NIS} =  \frac{2}{e^2 R_T} \int_{\Delta}^{\infty} d \epsilon  \ \frac{\epsilon^2}{\sqrt{\epsilon^2 - \Delta^2}} \Big[ e^{-\beta_N \epsilon} - e^{-\beta_S \epsilon}\Big] $$
    We can approximate $\frac{\epsilon^2}{\sqrt{\epsilon^2 - \Delta^2}}$ for low temperatures where $\epsilon \approx \delta$
    $$ \frac{\epsilon^2}{\sqrt{\epsilon^2 - \Delta^2}} \approx \frac{\Delta^2}{\sqrt{2 \Delta( \epsilon - \Delta)}} = \sqrt{\frac{\Delta^3}{2(\epsilon - \Delta)}} $$
    So now 
    $$ \dot{Q}^S_{NIS} = \frac{2}{e^2 R_T} \int_{\Delta}^{\infty} d \epsilon \sqrt{\frac{\Delta^3}{2(\epsilon - \Delta)}} \Big[ e^{-\beta_N \epsilon} - e^{-\beta_S \epsilon}\Big] $$
    $$ = \frac{\Delta \sqrt{2}}{e^2 R_T} \int_{\Delta}^{\infty} d \epsilon \sqrt{\frac{\Delta}{2(\epsilon - \Delta)}} \Big[ e^{-\beta_N \epsilon} - e^{-\beta_S \epsilon}\Big] $$
    Now introduce a change of variables where 
    $$ E = \frac{\epsilon}{\Delta} \ ; \ dE = \frac{d \epsilon}{\Delta} $$
    $$ u = \sqrt{E-1} \ ; \ dE = 2u \ du $$
    Which transformes the equation into the easily solved integral 
    $$ \dot{Q}^S_{NIS} = \frac{2 \sqrt{2} \Delta^2}{e^2 R_T} \Big[e^{-\beta_N \Delta}\int_0^{\infty} e^{-\beta_N \Delta u^2} du - e^{\beta_S \Delta} \int_0^{\infty} e^{-\beta_S \Delta u^2} du \Big] $$
    Easily solved due to 
    $$ \int_0^{\infty} e^{-\beta \Delta u^2} du = \frac{\sqrt{\pi}}{2 \sqrt{\Delta} \sqrt{\beta}} = \frac{1}{2} \sqrt{\frac{\pi}{\Delta \beta}} $$
    $$ \dot{Q}^S_{NIS} = \frac{2 \sqrt{2} \Delta^2}{e^2 R_T} \Big[e^{-\beta_N \Delta}\cdot \frac{1}{2} \sqrt{\frac{\pi}{\Delta \beta}} - e^{\beta_S \Delta} \cdot \frac{1}{2} \sqrt{\frac{\pi}{\Delta \beta}} \Big] $$
    Which simplifies to 
    $$ \dot{Q}^S_{NIS} = \frac{\sqrt{2 \pi \beta_S \beta_n} \Delta^{\frac{3}{2}}}{e^2 R_T} \Big[ e^{-\beta_N \Delta} - e^{-\beta_S \Delta}\Big] $$
    Isolating the $T_S$ part gives us the backflow of heat which is 
    $$ \dot{Q}^S_{NIS} = \frac{\Delta^2}{e^2 R_T} \Big[ \sqrt{\frac{2 \pi k_B T_S}{\Delta}} e^{-\frac{\Delta}{k_b T_S}}\Big]$$
    So totally for NIS cooling where $S$ is at non-zero temperature we have 
    $$ \dot{Q}_{NIS} \approx \frac{\Delta^2}{e^2 R_T} \Big[ 0.63\Big(\frac{k_B T_N}{\Delta}\Big)^{\frac{3}{2}} + \sqrt{\frac{2 \pi k_B T_S}{\Delta}} e^{-\frac{\Delta}{k_b T_S}}\Big] $$

    \item Problem 4.2. Derive the coefficient of performance. \\
    \\
    The coefficient of performance as stated is the cooling power divided by the power from the source at the optimum cooling point of the NIS cooler. 
    $$ \eta = \frac{\dot{Q}_{NIS}}{P_{in}} $$
    The cooling power is 
    $$ \dot{Q}_{NIS} \approx \gamma \frac{\Delta^2}{e^2 R_T}\Big(\frac{k_B T_N}{\Delta}\Big)^{\frac{3}{2}} $$
    And the power from source at optimum cooling point is 
    $$ P_{in} = I(V_{opt}) V = \alpha \frac{\Delta^2}{e^2 R_T} \sqrt{\frac{k_B T}{\Delta}} $$
    Ignoring the prefactors $\alpha$ and $\gamma$ we have 
    $$ \eta = \frac{\dot{Q}_{NIS}}{P_{in}} = \frac{\frac{\Delta^2}{e^2 R_T}\Big(\frac{k_B T}{\Delta}\Big)^{\frac{3}{2}}}{\frac{\Delta^2}{e^2 R_T} \sqrt{\frac{k_B T}{\Delta}}} $$
    $$ \eta = \frac{k_B T}{\Delta} $$


\end{enumerate}
\begin{thebibliography}{9}
    \bibitem{1}
    Anna Feshchenko. \\
    \textit{Electron Themometry, refrigeration and heat transport in nanostructures at sub-kelvin temperatures}
    Doctoral Thesis, Department of Applied Physics, Low Temperature Laboratory, Aalto University
    \bibitem{2}
    \textit{Micrometer-scale refrigerators}. Muhonen et al.,
    Reports on Progress in Physics 75, 046501 (2012)
    \url{ http://arxiv.org/abs/1203.5100}
    \bibitem{3}
    \textit{Superconducting tunnel junctions
    and nanorefrigeration using InAs
    nanowires} Jaakko Mastomäki. Master’s thesis, June 14, 2017
    \bibitem{4} 
    \textit{Tunnel-Junction Thermometry Down to Millikelvin Temperatures} A. V. Feshchenko et al 2015.
    \url{https://arxiv.org/pdf/1504.03841.pdf}


\end{thebibliography}
\end{document}

