
\documentclass[12pt]{article}
\usepackage[english]{babel}
\usepackage[T1]{fontenc}
\usepackage[utf8]{inputenc}
\usepackage{hyperref}
\usepackage{delarray,amsmath,bbm,epsfig,slashed}
\usepackage{graphicx}
\newcommand{\pat}{\partial}
\newcommand{\be}{\begin{equation}}
\newcommand{\ee}{\end{equation}}
\newcommand{\bea}{\begin{eqnarray}}
\newcommand{\eea}{\end{eqnarray}}
\newcommand{\abf}{{\bf a}}
\newcommand{\Zmath}{\mathbf{Z}}
\newcommand{\Zcal}{{\cal Z}_{12}}
\newcommand{\zcal}{z_{12}}
\newcommand{\Acal}{{\cal A}}
\newcommand{\Fcal}{{\cal F}}
\newcommand{\Ucal}{{\cal U}}
\newcommand{\Vcal}{{\cal V}}
\newcommand{\Ocal}{{\cal O}}
\newcommand{\Rcal}{{\cal R}}
\newcommand{\Scal}{{\cal S}}
\newcommand{\Lcal}{{\cal L}}
\newcommand{\Hcal}{{\cal H}}
\newcommand{\hsf}{{\sf h}}
\newcommand{\half}{\frac{1}{2}}
\newcommand{\Xbar}{\bar{X}}
\newcommand{\xibar}{\bar{\xi }}
\newcommand{\barh}{\bar{h}}
\newcommand{\Ubar}{\bar{\cal U}}
\newcommand{\Vbar}{\bar{\cal V}}
\newcommand{\Fbar}{\bar{F}}
\newcommand{\zbar}{\bar{z}}
\newcommand{\wbar}{\bar{w}}
\newcommand{\zbarhat}{\hat{\bar{z}}}
\newcommand{\wbarhat}{\hat{\bar{w}}}
\newcommand{\wbartilde}{\tilde{\bar{w}}}
\newcommand{\barone}{\bar{1}}
\newcommand{\bartwo}{\bar{2}}
\newcommand{\nbyn}{N \times N}
\newcommand{\repres}{\leftrightarrow}
\newcommand{\Tr}{{\rm Tr}}
\newcommand{\tr}{{\rm tr}}
\newcommand{\ninfty}{N \rightarrow \infty}
\newcommand{\unitk}{{\bf 1}_k}
\newcommand{\unitm}{{\bf 1}}
\newcommand{\zerom}{{\bf 0}}
\newcommand{\unittwo}{{\bf 1}_2}
\newcommand{\holo}{{\cal U}}
%\newcommand{\bra}{\langle}
%\newcommand{\ket}{\rangle}
\newcommand{\muhat}{\hat{\mu}}
\newcommand{\nuhat}{\hat{\nu}}
\newcommand{\rhat}{\hat{r}}
\newcommand{\phat}{\hat{\phi}}
\newcommand{\that}{\hat{t}}
\newcommand{\shat}{\hat{s}}
\newcommand{\zhat}{\hat{z}}
\newcommand{\what}{\hat{w}}
\newcommand{\sgamma}{\sqrt{\gamma}}
\newcommand{\bfE}{{\bf E}}
\newcommand{\bfB}{{\bf B}}
\newcommand{\bfM}{{\bf M}}
\newcommand{\cl} {\cal l}
\newcommand{\ctilde}{\tilde{\chi}}
\newcommand{\ttilde}{\tilde{t}}
\newcommand{\ptilde}{\tilde{\phi}}
\newcommand{\utilde}{\tilde{u}}
\newcommand{\vtilde}{\tilde{v}}
\newcommand{\wtilde}{\tilde{w}}
\newcommand{\ztilde}{\tilde{z}}

% David Weir's macros


\newcommand{\nn}{\nonumber}
\newcommand{\com}[2]{\left[{#1},{#2}\right]}
\newcommand{\mrm}[1] {{\mathrm{#1}}}
\newcommand{\mbf}[1] {{\mathbf{#1}}}
\newcommand{\ave}[1]{\left\langle{#1}\right\rangle}
\newcommand{\halft}{{\textstyle \frac{1}{2}}}
\newcommand{\ie}{{\it i.e.\ }}
\newcommand{\eg}{{\it e.g.\ }}
\newcommand{\cf}{{\it cf.\ }}
\newcommand{\etal}{{\it et al.}}
\newcommand{\ket}[1]{\vert{#1}\rangle}
\newcommand{\bra}[1]{\langle{#1}\vert}
\newcommand{\bs}[1]{\boldsymbol{#1}}
\newcommand{\xv}{{\bs{x}}}
\newcommand{\yv}{{\bs{y}}}
\newcommand{\pv}{{\bs{p}}}
\newcommand{\kv}{{\bs{k}}}
\newcommand{\qv}{{\bs{q}}}
\newcommand{\bv}{{\bs{b}}}
\newcommand{\ev}{{\bs{e}}}
\newcommand{\gv}{\bs{\gamma}}
\newcommand{\lv}{{\bs{\ell}}}
\newcommand{\nabv}{{\bs{\nabla}}}
\newcommand{\sigv}{{\bs{\sigma}}}
\newcommand{\notvec}{\bs{0}_\perp}
\newcommand{\inv}[1]{\frac{1}{#1}}
%\newcommand{\xv}{{\bs{x}}}
%\newcommand{\yv}{{\bs{y}}}
\newcommand{\Av}{\bs{A}}
%\newcommand{\lv}{{\bs{\ell}}}

%\newcommand\bsigma{\vec{\sigma}}
\hoffset 0.5cm
\voffset -0.4cm
\evensidemargin -0.2in
\oddsidemargin -0.2in
\topmargin -0.2in
\textwidth 6.3in
\textheight 8.4in

\begin{document}

\normalsize

\baselineskip 14pt

\begin{center}
{\Large {\bf Open Quantum Systems \ \  Answers to Problems 5.1 \& 5.2 - JPBK }}\\
{\large { Jake Muff}}\\
jake.muff@helsinki.fi \\
{Student number: 015361763}\\
{15/11/2020}
\end{center}


\section{Answers}
\begin{enumerate}
    \item Problem 5.1. If the microscopic energy conservation is assumed then 
    $$ H_0 = H_e + H_p = 0 $$
    Therefore 
    $$ \dot{H}_e = - \dot{H}_p $$
    So we can see 
    $$ ( \varepsilon_k - \varepsilon_{k-q}) = \hbar \omega_q $$
    Because if substituted into $\dot{H}_e$ we get $\dot{H}_p$ as
    $$ \omega_q^{1/2} \times \omega_q = \omega_q^{3/2} $$
    And the extra $\hbar$ factor is canceled through the prefactor before the sum 
    $$ \frac{-i \gamma}{\hbar} $$
    So $\dot{H}_e$ is 
    $$ \dot{H}_e = \frac{-i \gamma}{\hbar} \sum_{k,q} \omega_q^{1/2} ( \varepsilon_k - \varepsilon_{k-q} ) (a^{\dagger}_k a_{k-q} c_q - a^{\dagger}_{k-q} a_k c_q^{\dagger} ) $$
    \textbf{N.B} Is equation 4 in the lecture notes correct? In the reference in the lecture notes it gives a different equation for $\dot{H}_p$. 


    \item Problem 5.2. The classical Jospheson junction current is given by 
    $$ F_J = E_J (1- cos(\phi)) $$
    With 
    $$ E_J = \frac{I_c}{2e} $$
    Apply a bias current which is just $E_J$ at a non critical current so we get 
    $$ F_J = E_J (1 - \cos \phi) - \frac{\hbar I}{2e} \phi $$
    This forms a washboard potential 

    \begin{figure}[h]
        \includegraphics[width=15cm]{washboard.png}
        \centering
        \caption{Plot of the Potential as a function of $\phi$. The angle of tilt is given by $\frac{I}{I_c}$}
    \end{figure}
    $$ U(\phi) = E_J (1-cos \phi) - \phi \frac{I}{I_c} $$
    Where $ \frac{I}{I_c}$ is the normalized bias current. The first Josphenson relation describes the AC Josphenson effect 
    $$ \hbar \dot{\phi} = 2eV $$
    $$ \hbar \frac{d \phi}{dt} = 2eV $$
    From this we see that the critical current of the junction is when this is maximised so we can rearrange $E_J$ so that 
    $$ I_c = \frac{E_J 2e}{\hbar} = \frac{2e}{\hbar} E_J $$



\end{enumerate}
\end{document}

