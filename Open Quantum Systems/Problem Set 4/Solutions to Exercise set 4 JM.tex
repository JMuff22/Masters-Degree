
\documentclass[12pt]{article}
\usepackage[finnish]{babel}
\usepackage[T1]{fontenc}
\usepackage[utf8]{inputenc}
\usepackage{delarray,amsmath,bbm,epsfig,slashed}
\newcommand{\pat}{\partial}
\newcommand{\be}{\begin{equation}}
\newcommand{\ee}{\end{equation}}
\newcommand{\bea}{\begin{eqnarray}}
\newcommand{\eea}{\end{eqnarray}}
\newcommand{\abf}{{\bf a}}
\newcommand{\Zmath}{\mathbf{Z}}
\newcommand{\Zcal}{{\cal Z}_{12}}
\newcommand{\zcal}{z_{12}}
\newcommand{\Acal}{{\cal A}}
\newcommand{\Fcal}{{\cal F}}
\newcommand{\Ucal}{{\cal U}}
\newcommand{\Vcal}{{\cal V}}
\newcommand{\Ocal}{{\cal O}}
\newcommand{\Rcal}{{\cal R}}
\newcommand{\Scal}{{\cal S}}
\newcommand{\Lcal}{{\cal L}}
\newcommand{\Hcal}{{\cal H}}
\newcommand{\hsf}{{\sf h}}
\newcommand{\half}{\frac{1}{2}}
\newcommand{\Xbar}{\bar{X}}
\newcommand{\xibar}{\bar{\xi }}
\newcommand{\barh}{\bar{h}}
\newcommand{\Ubar}{\bar{\cal U}}
\newcommand{\Vbar}{\bar{\cal V}}
\newcommand{\Fbar}{\bar{F}}
\newcommand{\zbar}{\bar{z}}
\newcommand{\wbar}{\bar{w}}
\newcommand{\zbarhat}{\hat{\bar{z}}}
\newcommand{\wbarhat}{\hat{\bar{w}}}
\newcommand{\wbartilde}{\tilde{\bar{w}}}
\newcommand{\barone}{\bar{1}}
\newcommand{\bartwo}{\bar{2}}
\newcommand{\nbyn}{N \times N}
\newcommand{\repres}{\leftrightarrow}
\newcommand{\Tr}{{\rm Tr}}
\newcommand{\tr}{{\rm tr}}
\newcommand{\ninfty}{N \rightarrow \infty}
\newcommand{\unitk}{{\bf 1}_k}
\newcommand{\unitm}{{\bf 1}}
\newcommand{\zerom}{{\bf 0}}
\newcommand{\unittwo}{{\bf 1}_2}
\newcommand{\holo}{{\cal U}}
%\newcommand{\bra}{\langle}
%\newcommand{\ket}{\rangle}
\newcommand{\muhat}{\hat{\mu}}
\newcommand{\nuhat}{\hat{\nu}}
\newcommand{\rhat}{\hat{r}}
\newcommand{\phat}{\hat{\phi}}
\newcommand{\that}{\hat{t}}
\newcommand{\shat}{\hat{s}}
\newcommand{\zhat}{\hat{z}}
\newcommand{\what}{\hat{w}}
\newcommand{\sgamma}{\sqrt{\gamma}}
\newcommand{\bfE}{{\bf E}}
\newcommand{\bfB}{{\bf B}}
\newcommand{\bfM}{{\bf M}}
\newcommand{\cl} {\cal l}
\newcommand{\ctilde}{\tilde{\chi}}
\newcommand{\ttilde}{\tilde{t}}
\newcommand{\ptilde}{\tilde{\phi}}
\newcommand{\utilde}{\tilde{u}}
\newcommand{\vtilde}{\tilde{v}}
\newcommand{\wtilde}{\tilde{w}}
\newcommand{\ztilde}{\tilde{z}}

\newtheorem{theorem}{Theorem}

% David Weir's macros


\newcommand{\nn}{\nonumber}
\newcommand{\com}[2]{\left[{#1},{#2}\right]}
\newcommand{\mrm}[1] {{\mathrm{#1}}}
\newcommand{\mbf}[1] {{\mathbf{#1}}}
\newcommand{\ave}[1]{\left\langle{#1}\right\rangle}
\newcommand{\halft}{{\textstyle \frac{1}{2}}}
\newcommand{\ie}{{\it i.e.\ }}
\newcommand{\eg}{{\it e.g.\ }}
\newcommand{\cf}{{\it cf.\ }}
\newcommand{\etal}{{\it et al.}}
\newcommand{\ket}[1]{\vert{#1}\rangle}
\newcommand{\bra}[1]{\langle{#1}\vert}
\newcommand{\bs}[1]{\boldsymbol{#1}}
\newcommand{\xv}{{\bs{x}}}
\newcommand{\yv}{{\bs{y}}}
\newcommand{\pv}{{\bs{p}}}
\newcommand{\kv}{{\bs{k}}}
\newcommand{\qv}{{\bs{q}}}
\newcommand{\bv}{{\bs{b}}}
\newcommand{\ev}{{\bs{e}}}
\newcommand{\gv}{\bs{\gamma}}
\newcommand{\lv}{{\bs{\ell}}}
\newcommand{\nabv}{{\bs{\nabla}}}
\newcommand{\sigv}{{\bs{\sigma}}}
\newcommand{\notvec}{\bs{0}_\perp}
\newcommand{\inv}[1]{\frac{1}{#1}}
%\newcommand{\xv}{{\bs{x}}}
%\newcommand{\yv}{{\bs{y}}}
\newcommand{\Av}{\bs{A}}
%\newcommand{\lv}{{\bs{\ell}}}

%\newcommand\bsigma{\vec{\sigma}}
\hoffset 0.5cm
\voffset -0.4cm
\evensidemargin -0.2in
\oddsidemargin -0.2in
\topmargin -0.2in
\textwidth 6.3in
\textheight 8.4in

\begin{document}

\normalsize

\baselineskip 14pt

\begin{center}
{\Large {\bf Open Quantum Systems \ \ Fall 2020 \ \  Answers to Exercise Set 4}}\\
{\large { Jake Muff}}\\
{Student number: 015361763}\\
{15/10/2020}
\end{center}


\section{Answers}
\begin{enumerate}

\item \textbf{Exercise 1}
\begin{enumerate}
    \item Nielsen and Chaung Exercise 2.57\\
    $$ \{ L_l \} \rightarrow \{ M_m \} = \{ N_{lm} \} $$
    Suppose we have a state $\ket{\phi}$, the sttate of the system after measurement is given by 
    $$ \ket{\phi} = \frac{L_l \ket{\psi}}{\sqrt{\langle \psi|L_l^\dagger L_l | \psi \rangle}} $$
    We then have another set of measurements such that 
    $$ \langle \phi | M_m^\dagger M_m | \phi \rangle = \frac{\langle \psi | L_l^\dagger M_m^\dagger M_m L_l | \psi \rangle}{\langle \psi | L_l^\dagger L_l | \psi \rangle }$$
    So totally we have 
    $$ \frac{M_m \ket{\phi}}{\sqrt{\langle \phi | M_m^\dagger M_m | \phi \rangle}} = \frac{M_m L_l \ket{\psi}}{\sqrt{\langle \psi | L_l^\dagger L_l | \psi \rangle}} \cdot \frac{\sqrt{\langle \psi | L_l^\dagger L_l | \psi \rangle}}{\sqrt{\langle \psi | L_l^\dagger M_m^\dagger M_m L_l | \psi \rangle}} $$
    $$ = \frac{M_m L_l \ket{\psi}}{\sqrt{\langle \psi | L_l^\dagger M_m^\dagger M_m L_l | \psi \rangle}} $$
    $$ = \frac{N_{lm} \ket{\psi}}{\langle \psi | N_{lm}^\dagger N_{lm}  | \psi \rangle } $$

    \item State after the second measurement is performed \\
    \item Last measurement unrecorded 
\end{enumerate}

\item \textbf{Exercise 2}
\begin{enumerate}
    \item Equations of motion for the components $\psi_{\pm}(t)$ \\
    
    \item Equations of motion for $\xi_{-}(t)$
    
    \item Probability $P(t)$ that it is $- \hbar /2$. When is $P(t)$ a maximum? 

\end{enumerate}

\item \textbf{Exercise 3} 

\begin{enumerate}
    \item Rotating $n$ times with angle $\theta$ is the same as rotating with an angle $n \theta$\\
    $$U(\theta)^n = \left(\begin{array}{cc} cos(\theta) & sin(\theta) \\ -sin(\theta) & cos(\theta) \end{array}\right)^n $$
    We can trivially show that for $n=2$ we have 
    $$ U(\theta)^2 = \left(\begin{array}{cc} cos^2(\theta)-sin^2(\theta) & 2sin(\theta)cos(\theta) \\ -2sin(\theta)cos(\theta) & cos^2(\theta)-sin^2 (\theta) \end{array}\right) $$
    Which, using trig identities simplifies down to 
    $$ U(\theta)^2 = \left(\begin{array}{cc} cos(2\theta) & sin(2\theta) \\ -sin(2\theta) & cos(2\theta) \end{array}\right) $$
    And this applies for integer values of $n$. This follows from De Moivre's theorem such that 
    $$ (cos(x) + isin(x))^n = cos(nx) + isin(nx) $$
    Which is related to matrix representation in Group Theory 
    $$ z = x + iy \in \mathbbm{C} \rightarrow \left(\begin{array}{cc} x & -y \\ y & x \end{array}\right) \in GL(2)  $$
    And $(cos(x) + isin(x))^n$ is an isomorphism from $\mathbbm{C}$ to $GL(2)$. Notice how the $y's$ is the above matrix are swapped. In this case the unitary transformation representation a clockwise rotation or a $-\theta$ rotation as opposed to the convention. Rotations will have this same property rotating clockwise or anticlockwise. \\
    In summary this property $U(\theta)^n = U(n \theta)$ is a direct consequence of the fact that for matrices of this type are isomorphic to the space of complex number $\mathbbm{C}$ which I why De Moivre's theorem can be used here. 
    
    \item Projector on a photon state with polarisation rotated $\pi /4$ from the horizontal? \\
    The horizontal polarisation is $\ket{h}$ and the vertical polarisation is $\ket{v}$. The projector onto $\ket{h}$ and $\ket{v}$ is 
    $$ P_h = \ket{h} \bra{h} $$
    $$ P_v = \ket{v} \bra{v} $$
    When the polarisation is rotated by $\pi /4$ which si the same as 45 degrees the projector onto the photon state will be 
    $$ P_{\frac{\pi}{4}} = \frac{\ket{h + v} \bra{h+v}}{2} $$
    Additionally the state after going through polarisation is 
    $$ \frac{\ket{h+v}}{\sqrt{2}} $$ 
    With probability equal to $\frac{1}{2}$ 
    
    \item Probability of finding the photon in the horizontal state after rotation from horizontal by $\theta$ \\
    After a rotation of $\theta$ from the horizontal, the projector will be 
    $$ P_{\theta} = (cos(\theta) \ket{h} + sin(\theta)\ket{v})(cos(\theta) \bra{h} + sin(\theta) \bra{v}) $$
    With a probability 
    $$ p(\theta) = |(cos(\theta) \bra{h} + sin(\theta) \bra{v}) \ket{h} |^2 $$
    $$ p(\theta) = cos^2(\theta) $$
    
\end{enumerate}

\item \textbf{Exercise 4} 
\begin{enumerate}
    \item Calculate $\ket{\phi} = M_n \ket{h}$\\
    After $n$ rotations the photon is rotated 
    $$ U(\theta)^n \ket{h} = U(n \theta) \ket{h} = cos(n \theta) \ket{h} + sin(n \theta) \ket{v} $$
    Such that 
    $$\ket{\phi}_n = cos(n \theta) \ket{h} + sin(n \theta) \ket{v}$$
   
    
    \item Probability of the probability for the photon to have horizontal polarisation after applying $M_n$ to $\ket{h}$\\
    Probability to get horizontal polarisation is simply 
    $$ p_n = cos^{2n} (\theta) $$
    
    \item What happens to probability for small $\theta$? \\
    For small $\theta$ the probability tends to 1 with state $\ket{h}$
\end{enumerate}

\item \textbf{Exercise 4} 
\begin{enumerate}
    \item Show that 
    $$ e^A e^B = e^{A+B} $$
    For commuting matrices \\
    From definition 
    $$ e^A = \sum_{k=0}^\infty \frac{A^k}{k!} $$
    So we have 
    $$ e^A e^B = \Big(\sum_{k=0}^\infty \frac{A^k}{k!} \Big) \Big(\sum_{k=0}^\infty \frac{B^k}{k!} \Big) $$
    $$ = \sum_{k=0}^\infty \sum_{j=0}^\infty \frac{A^k B^j}{k! j!} = \sum_{l=0}^\infty \sum_{k=0}^l \frac{A^k B^{l-k}}{k! (l-k)!} $$
    $$ = \sum_{l=0}^\infty \frac{1}{l!} \sum_{k=0}^l \frac{l!}{k!(l-k)!}A^k B^{l-k} $$
    $$ = \sum_{l=0}^\infty \frac{(A+B)^l}{l!} = e^{A+B} $$
    Which only works if we can use the fact that $AB = BA$ and they commute. \\
    \item 
    As a counter example for non-commuting matrices we have the Pauli matrices which have the property 
    $$ [ \sigma_j, \sigma_k ] = 2i \sum_{l=1}^3 \epsilon_{jkl} \sigma_l $$
    Lets use the $\sigma_1$ and $\sigma_2$ pauli matrices and calculate $e^A e^B$ and $e^{A+B}$ 
    $$ e^{\sigma_1} e^{\sigma_2} = \Big(\sum_{k=0}^\infty \frac{1}{k!} \sigma_1^k \Big) \Big(\sum_{k=0}^\infty \frac{1}{k!} \sigma_2^k \Big) $$
    $$ = \left(\begin{array}{cc} \frac{e^2 +1}{2e} & \frac{e^2 -1 }{2e} \\ \frac{e^2 -1}{2e} & \frac{e^2 +1}{2e} \end{array}\right) \cdot \left(\begin{array}{cc} e & 0 \\ 0 & \frac{1}{e} \end{array}\right) $$
    $$ = \left(\begin{array}{cc} \frac{e^2 +1}{2} & \frac{e^2 -1 }{2e^2} \\ \frac{e^2 -1}{2} & \frac{e^2 +1}{2e^2} \end{array}\right) $$

    On the other side we have 
    $$ e^{\sigma_1 + \sigma_2} = \exp \left(\begin{array}{cc} 1 & 1\\ 1 & -1 \end{array}\right) $$
    $$ = \left(\begin{array}{cc} \frac{e^{2 \sqrt{2}} + \sqrt{2} e^{2 \sqrt{2}} -1 + \sqrt{2}}{2 \sqrt{2} e^{\sqrt{2}}} & \frac{e^{2 \sqrt{2}}-1}{2 \sqrt{2} e^{\sqrt{2}}} \\ \frac{e^{2 \sqrt{2}} -1}{2 \sqrt{2} e^{\sqrt{2}}} & \frac{-e^{2 \sqrt{2}}(- \sqrt{2}+1) +1 + \sqrt{2}}{2 \sqrt{2} e^{\sqrt{2}}} \end{array}\right) $$
    Clearly 
    $$ e^{\sigma_1} e^{\sigma_2} \neq e^{\sigma_1 + \sigma_2} $$


\end{enumerate}

\end{enumerate}
\section{Appendix}
\begin{enumerate}
    \item I didn't fully understand what the notation with $M_n$ meant in exercise 4 as I computed this and my answer didn't make sense. The answer i've given makes sense conceptually to me so thats why I gave it. Here is what I got using the notation in the exercise \\
    \\
    We know that 
    $$ P_h = \ket{h} \bra{h} = \left(\begin{array}{cc} 0 \\ 1\end{array}\right) \left(\begin{array}{cc} 1 &0\end{array}\right) = \left(\begin{array}{cc} 0 & 0 \\ 1 & 0\end{array}\right)$$
    Such that 
    $$ P_h U^n (\theta) P_h = P_h U(n \theta) P_h $$
    $$ = \left(\begin{array}{cc} 0 & 0 \\ sin(n \theta) & 0\end{array}\right) $$
    And 
    $$ U^n (\theta) P_h U^n (\theta) = \left(\begin{array}{cc} sin(n \theta)cos(n \theta) & sin^2(n \theta) \\ cos^2 (n \theta) & cos(n \theta)sin(n \theta)\end{array}\right)$$
    So that 
    $$ M_n = \left(\begin{array}{cc} 0 & 0 \\ sin^2 (n \theta)cos(n \theta) & sin^3(n \theta) \end{array}\right)$$
    And 
    $$ \ket{\phi } = \left(\begin{array}{cc} 0 & 0 \\ sin^3 (n \theta) & 0 \end{array}\right)$$
    \\
    This didn't make sense as the probably given using this answer wouldn't tend to 1 as I know it should as I've studied the Quantum Zeno effect previously. 
\end{enumerate}

\end{document}

