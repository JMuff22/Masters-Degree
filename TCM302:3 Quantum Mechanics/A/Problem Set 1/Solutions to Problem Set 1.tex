
\documentclass[12pt]{article}
%\usepackage[finnish]{babel}
\usepackage[T1]{fontenc}
\usepackage[utf8]{inputenc}
\usepackage{amssymb}
\usepackage{amsmath}
\usepackage{graphicx}
\usepackage{hyperref}
\newcommand{\pat}{\partial}
\newcommand{\be}{\begin{equation}}
\newcommand{\ee}{\end{equation}}
\newcommand{\bea}{\begin{eqnarray}}
\newcommand{\eea}{\end{eqnarray}}
\newcommand{\abf}{{\bf a}}
\newcommand{\Zcal}{{\cal Z}_{12}}
\newcommand{\zcal}{z_{12}}
\newcommand{\Acal}{{\cal A}}
\newcommand{\Fcal}{{\cal F}}
\newcommand{\Ucal}{{\cal U}}
\newcommand{\Vcal}{{\cal V}}
\newcommand{\Ocal}{{\cal O}}
\newcommand{\Rcal}{{\cal R}}
\newcommand{\Scal}{{\cal S}}
\newcommand{\Lcal}{{\cal L}}
\newcommand{\Hcal}{{\cal H}}
\newcommand{\hsf}{{\sf h}}
\newcommand{\half}{\frac{1}{2}}
\newcommand{\Xbar}{\bar{X}}
\newcommand{\xibar}{\bar{\xi }}
\newcommand{\barh}{\bar{h}}
\newcommand{\Ubar}{\bar{\cal U}}
\newcommand{\Vbar}{\bar{\cal V}}
\newcommand{\Fbar}{\bar{F}}
\newcommand{\zbar}{\bar{z}}
\newcommand{\wbar}{\bar{w}}
\newcommand{\zbarhat}{\hat{\bar{z}}}
\newcommand{\wbarhat}{\hat{\bar{w}}}
\newcommand{\wbartilde}{\tilde{\bar{w}}}
\newcommand{\barone}{\bar{1}}
\newcommand{\bartwo}{\bar{2}}
\newcommand{\nbyn}{N \times N}
\newcommand{\repres}{\leftrightarrow}
\newcommand{\Tr}{{\rm Tr}}
\newcommand{\tr}{{\rm tr}}
\newcommand{\ninfty}{N \rightarrow \infty}
\newcommand{\unitk}{{\bf 1}_k}
\newcommand{\unitm}{{\bf 1}}
\newcommand{\zerom}{{\bf 0}}
\newcommand{\unittwo}{{\bf 1}_2}
\newcommand{\holo}{{\cal U}}
%\newcommand{\bra}{\langle}
%\newcommand{\ket}{\rangle}
\newcommand{\muhat}{\hat{\mu}}
\newcommand{\nuhat}{\hat{\nu}}
\newcommand{\rhat}{\hat{r}}
\newcommand{\phat}{\hat{\phi}}
\newcommand{\that}{\hat{t}}
\newcommand{\shat}{\hat{s}}
\newcommand{\zhat}{\hat{z}}
\newcommand{\what}{\hat{w}}
\newcommand{\sgamma}{\sqrt{\gamma}}
\newcommand{\bfE}{{\bf E}}
\newcommand{\bfB}{{\bf B}}
\newcommand{\bfM}{{\bf M}}
\newcommand{\cl} {\cal l}
\newcommand{\ctilde}{\tilde{\chi}}
\newcommand{\ttilde}{\tilde{t}}
\newcommand{\ptilde}{\tilde{\phi}}
\newcommand{\utilde}{\tilde{u}}
\newcommand{\vtilde}{\tilde{v}}
\newcommand{\wtilde}{\tilde{w}}
\newcommand{\ztilde}{\tilde{z}}
\newcommand{\ket}[1]{\vert{#1}\rangle}
\newcommand{\bra}[1]{\langle{#1}\vert}


\hoffset 0.5cm
\voffset -0.4cm
\evensidemargin -0.2in
\oddsidemargin -0.2in
\topmargin -0.2in
\textwidth 6.3in
\textheight 8.4in

\begin{document}

\normalsize

\baselineskip 14pt

\begin{center}
{\Large {\bf Quantum Mechanics IIa 2021 \ \ \\ Solutions to Problem Set 1}} \\
Jake Muff
22/01/21
\end{center}
\section*{Question 1}
Simple harmonic oscillator subjected to a perturbation 
$$ \lambda V = bx $$
\begin{enumerate}
    \item Energy shift of ground state to lowest non-vanishing order: \\
    
    The energy shift for this pertubation is 
    $$ \Delta_n^{(0)} \equiv E_n - E_n^{(0)} $$
    $$ = \lambda V_{nn} + \lambda^2 \sum_{k \neq n} \frac{|V_{nk}|^2}{E_n^{(0)} - E_k^{(0)} } + \ldots $$
    $$ = b \langle n^{(0)} | x | n^{(0)} \rangle + b^2 \sum_{k \neq n} \frac{| \langle n^{(0)} | x| k^{(0)} \rangle |^2}{E_n^{(0)} - E_k^{(0)} } + \ldots $$
    The first order correction is: 
    $$ \Delta_n^{(1)} = b \langle n^{(0)} | x | n^{(0)} \rangle $$
    $$ = b \langle 0 | x | 0 \rangle $$
    Using the hint we get 
    $$ = b \sqrt{\frac{\hbar}{2 m \omega} } ( \sqrt{0+1} \delta_{0,0+1} + \sqrt{0} \delta_{0, 0-1} ) $$
    $$ = 0 $$
    This didn't work so we go to second order: 
    $$ \Delta_n^{(2)} = \sum_{k \neq n} \frac{|V_{nk}|^2}{E_n^{(0)} - E_k^{(0)}} $$
    $$ = b^2 \sum_{k \neq n} \frac{| \langle 0 | x | 1 \rangle |^2 }{E_n^{(0)} - E_k^{(0)} } $$
    Here we have $k^{(0)} = 1$ as $k \neq n$ so the sum $\rightarrow \sum_{k=1}^{\infty} $. The ground state energy of a simple harmonic oscillator is well known 
    $$ E_n^{(0)} = (n + \frac{1}{2}) \hbar \omega $$
    Therefore, 
    $$ E_0 = \frac{1}{2} \hbar \omega $$
    $$ E_1 = \frac{3}{2} \hbar \omega $$
    Using this we can evaluate the second order correction 
    $$ \Delta_n^{(2)} = b^2 \sum_{k \neq n} \frac{| \langle 0 | x | 1 \rangle |^2}{\frac{1}{2} \hbar \omega - \frac{3}{2} \hbar \omega} = b^2 \sum_{k \neq n} \frac{| \langle 0 | x| 1 \rangle |^2 }{- \hbar \omega} $$
    From the hint the numerator can be evaluated 
    $$ \langle 0 | x | 1 \rangle = \sqrt{\frac{\hbar}{2 m \omega}} ( \sqrt{1+1} \delta_{0,1+1} + \sqrt{1}\delta_{0, 1-1} ) $$
    $$ = \sqrt{\frac{\hbar}{2 m \omega}} $$
    So we get 
    $$ \Delta_n^{(2)} = b^2 \frac{|\sqrt{\frac{\hbar}{2 m \omega}}|^2}{- \hbar \omega} = - \frac{b^2}{2 m \omega} $$

    \item To solve this problem exactly we need the perturbed hamiltonian for a simple harmonic oscillator 
    $$ H = \frac{p^2}{2m} + \frac{1}{2} m \omega^2 x^2 + bx $$
    By completing the square on this we see that this a simple harmonic oscillator which is shifted 
    $$ H = \frac{p^2}{2m} + \frac{1}{2} m \omega^2 ( x^2 + \frac{2bx}{m \omega^2} ) $$
    $$ = \frac{p^2}{2m} + \frac{m \omega^2}{2} ( x+ \frac{b}{m \omega^2})^2 - \frac{b^2}{2 m \omega^2} $$
    So, clearly, this has the same energy eigenvalues before just with an energy shift of 
    $$ - \frac{b^2}{2 m \omega^2} $$
    This is the same as above, so in this case doing pertubation theory up to second order gives the same answer as in the exact case.
\end{enumerate}

\section*{Question 2}
From QM1 studies of the Orbital Angular momentum we know that we can write 
$$ \langle n', l',m' |\hat{z} | n,l,m \rangle = \int \int \int d^3 \vec{x} \psi_{n' l' m'}^* (r, \theta, \phi ) r \cos (\theta) \psi_{nlm} (r, \theta, \phi) $$
So in our case we have 
$$ \langle 1 0 0  |\hat{z} | n,l,m \rangle = \int \int \int d^3 \vec{x} \psi_{100}^* (r, \theta, \phi ) r \cos (\theta) \psi_{nlm} (r, \theta, \phi) $$
Due to spherical symmetry i.e $ \psi_{100} (\vec{x} ) = \psi_{100} (r)$, and $n' l' m' = 100$ , when $l=0$ the matrix elements vanish. This means that the matrix elements must vanish unless $l=1$ which is the same as saying $l \neq 1$. 
The $\cos (\theta)$ introduced here by the $\hat{z}$ changes the known selection rule to be $l' = l+1$ which shows us what we want. 


\section*{Question 3}
For eigenstates 
$$ \ket{2, \pm} = \frac{1}{\sqrt{2}} (\ket{200} \pm \ket{210} ) $$
The energy shifts are
$$ \Delta_{\pm}^{(1)} = \pm 3 e a_0 | \vec{E} | $$
To which the energy shift corresponds to the dipole number $\vec{d_{\pm} } $
$$ - \vec{d} \cdot \vec{E} = \pm 3 e a_0 | \vec{E} | $$
Taking the dot product 
$$ - |\vec{d} | \cdot | \vec{E} | \cos(\theta) = \pm 3 e a_0 | \vec{E} | $$
$$ | \vec{d}| = \mp 3 e a_0 / \cos (\theta) $$
The dipole moments are orientated parallel or antiparallel to the external field, thus the $\cos$ cancels out and we're left with 
$$ \mp 3 e a_0 $$
\section*{Question 4}
This is a 2D infinite square well, of which the solutions  to the 1D case are well known from QM1, it is easy to extend this to the 2D case. 
\\
For the Ground state (n=1), the wavefunction is 
$$ \psi_1 (x,y) = \frac{2}{L} \sin (\frac{ \pi x}{L}) \sin (\frac{\pi y}{L}) $$
For the first excited state (n=2), which has 2 fold degeneracy, the wavefunctions are 
$$ \psi_{2a} (x,y) = \frac{2}{L} \sin (\frac{ 2 \pi x}{L}) \sin (\frac{\pi y }{L}) $$
$$ \psi_{2b} (x,y) = \frac{2}{L} \sin (\frac{\pi x }{L}) \sin ( \frac{ 2 \pi y}{L} ) $$
Ground state energy eigenfunctions: The zeroth order energy eigenfunction is simply 
$$ \psi_1^{(0)}  (x,y) = \frac{2}{L} \sin (\frac{ \pi x}{L}) \sin (\frac{\pi y}{L}) $$
The first order energy shift when the perturbation $\lambda V_1 = \lambda xy$ is applied can be calculated from 
$$ \Delta_n^{(1)} = \langle n^{(0)} | V_1 | n^{(0)} \rangle $$
$$ = \lambda \langle 1^{(0)} | xy | 1^{(0)} \rangle $$
$$ = \lambda \int_0^L \int_0^L |\psi_1^0 (x,y) |^2 xy dx dy $$
$$ = \frac{4 \lambda}{L^2} \int_0^L \int_0^L xy \sin^2 (\frac{\pi x}{L} ) dx dy $$
$$ = \frac{1}{4} \lambda L^2 $$
For the first excited state: 
\\
The 1st order energy shift is 
$$ \Delta_2^{(1)} = \langle 2^{(0)} | V_1 | 2^{(0)} \rangle $$
$$ = \lambda \langle 2^{(0)} | xy | 2^{(0)} \rangle $$
We have 2 fold degeneracy and can be represented through a 2x2 matrix for the different permutations 
$$ \Delta_2^{(1)} = \lambda \begin{bmatrix}
    \langle \psi_{2a} | xy | \psi_{2a} \rangle & \langle \psi_{2a} | xy | \psi_{2b} \rangle \\
    \langle \psi_{2b} | xy | \psi_{2a} \rangle & \langle \psi_{2b} | xy | \psi_{2a} \rangle 
\end{bmatrix} $$
We now evaluate the matrix elements 
$$ \langle \psi_{2a} | xy | \psi_{2a} \rangle = \frac{4}{L^2} \int_0^{L} \int_0^L \sin^2 (\frac{2 \pi x}{L} ) \sin^2 (\frac{ \pi y}{L}) xy dx dy $$
$$ = \frac{L^2}{4} $$
Also
$$ \langle \psi_{2b} | xy | \psi_{2b} \rangle = \frac{L^2}{4} $$
Due to symmetry. 
$$ \langle \psi_{2a} | xy | \psi_{2b} \rangle = \frac{4}{L^2} \int_0^L \int_0^L \sin (\frac{2 \pi x}{L}) \sin (\frac{\pi x}{L} ) \sin (\frac{2 \pi y}{L} ) \sin (\frac{\pi y}{L}) xy dx dy $$
$$ = \frac{256 L^2 }{\pi^4 81} $$
And 
$$ \langle \psi_{2b} | xy | \psi_{2a} \rangle = \frac{256 L^2 }{\pi^4 81} $$
$$ \Delta_2^{(1)} = \lambda \begin{bmatrix}
    \frac{L^2}{4} & \frac{256 L^2 }{\pi^4 81} \\
    \frac{256 L^2 }{\pi^4 81} & \frac{L^2}{4} 
\end{bmatrix} $$
Finding the eigenvalues  (E) of this matrix gives 
$$ E = \frac{81 \pi^4 \pm 1024}{324 \pi^4} \lambda L^2 $$
So we have 
$$ \Delta_{2a}^{(1)} = 0.282 \lambda L^2 $$
$$ \Delta_{2b}^{(1)} = 0.218 \lambda L^2 $$
The normalize eigenvectors are then 
$$ \psi_{2a}^{(0)} = \frac{1}{\sqrt{2}} \begin{bmatrix}
    1 \\ 1 
\end{bmatrix} $$
$$ \psi_{2b}^{(0)} = \frac{1}{\sqrt{2}} \begin{bmatrix}
    -1 \\ 1 
\end{bmatrix} $$
Thus, the zeroth order energy eigenfunctions are 
$$ \psi_{2a}^{(0)} = \frac{\sqrt{2}}{L} ( \sin (\frac{ 2 \pi x}{L } ) \sin (\frac{\pi y}{L}) + \sin (\frac{\pi x}{L} ) \sin (\frac{2 \pi y }{L}) ) $$
$$ \psi_{2a}^{(0)} = \frac{\sqrt{2}}{L} ( \sin (\frac{ 2 \pi x}{L } ) \sin (\frac{\pi y}{L}) - \sin (\frac{\pi x}{L} ) \sin (\frac{2 \pi y }{L}) ) $$


To evaluate the integrals and find the eigenvalues I used Maple. I have attached the maple worksheet along with this. 



\section*{Question 5}
$$H_0 = \frac{P_x^2 + P_y^2}{2m} + \frac{1}{2} m \omega_0^2 (x^2 + y^2) $$
$$ = \hbar \omega_0 (a^{\dagger} a + b^{\dagger} b + 1) $$
$$ x= \frac{1}{\sqrt{2 \beta} } (a + a^{\dagger}) $$
$$ x= \frac{1}{\sqrt{2 \beta} } (b + b^{\dagger}) $$
$$ \beta \equiv \frac{m \omega_0}{\hbar} $$
\begin{enumerate}
    \item In the level number representation $\ket{nm}$, where $n$ relates to $a(x)$ and $m$ relates to $b (y)$ then the energy eigenvalues are 
    $$ E = \hbar \omega_0 (n+m +1) $$
    The degeneracies are determined bu counting the number of different ways that the two integers can be added to give $N$ where $N = n +m$. In this case the degneracy is $N+1$ for each energy level $N$. 
    So for the lowest energy eigenstate where $\ket{n=0, m=0} = \ket{0,0}, E=\hbar \omega_0$ and the 'degeneracy' is 1. 
    For the next level we have two combinations for $\ket{nm}$; $\ket{0,1}$ and $\ket{1,0}$ which have energies $E= 2 \hbar \omega_0$. The degeneracy matches the rule $N+1 = 1 +1 = 2$. This is the first excited state.
    
    \item If turn on the pertubation 
    $$ \lambda V = \lambda xy$$ 
    We get 
    $$ H = H_0 + \lambda xy $$ 
    Let's represent the first excited energy level with 2 degenerate energy eigenvalues as $\psi_{10}$ and $\psi_{01}$ for the $\ket{n,m} = \ket{1,0} = \ket{0,1} $. 
    This can be represented by the 2x2 matrix 
    $$ \begin{pmatrix}
        \langle \psi_{10} | xy | \psi_{10} \rangle & \langle \psi_{10} | xy | \psi_{01} \rangle 
        \\
        \langle \psi_{01} | xy| \psi_{10} \rangle & \langle \psi_{01} | xy | \psi_{01} \rangle 
    \end{pmatrix} $$
    From the question we can represent $xy$ using the creation/annihilation operators 
    $$ xy = \Big( \frac{1}{\sqrt{2 \beta} } \Big)^2 (a + a^{\dagger} ) (b + b^{\dagger} ) $$
    Which we can use to evaluate the matrix elements 
    $$  \langle \psi_{10} | xy | \psi_{10} \rangle =  \Big( \frac{1}{\sqrt{2 \beta} } \Big)^2 \langle \psi_{1} (x) | a+a^{\dagger} | \psi_{1} (x) \rangle \langle \psi_{0} (y) | b+b^{\dagger} | \psi_{0} (y) \rangle = 0 $$
    $$  \langle \psi_{10} | xy | \psi_{01} \rangle =  \Big( \frac{1}{\sqrt{2 \beta} } \Big)^2 \langle \psi_{1} (x) | a+a^{\dagger} | \psi_{0} (x) \rangle \langle \psi_{0} (y) | b+b^{\dagger} | \psi_{1} (y) \rangle = \Big( \frac{1}{\sqrt{2 \beta} } \Big)^2 $$
    $$  \langle \psi_{01} | xy | \psi_{01} \rangle =  \Big( \frac{1}{\sqrt{2 \beta} } \Big)^2 \langle \psi_{0} (x) | a+a^{\dagger} | \psi_{0} (x) \rangle \langle \psi_{1} (y) | b+b^{\dagger} | \psi_{1} (y) \rangle = 0 $$
    $$  \langle \psi_{01} | xy | \psi_{10} \rangle =  \Big( \frac{1}{\sqrt{2 \beta} } \Big)^2 \langle \psi_{0} (x) | a+a^{\dagger} | \psi_{1} (x) \rangle \langle \psi_{1} (y) | b+b^{\dagger} | \psi_{0} (y) \rangle = \Big( \frac{1}{\sqrt{2 \beta} } \Big)^2 $$

    We can diagonalize this or solve the eigenvalue problem for this case. Either way we find eigenvalues 
    $$ \mathcal{E} = \pm \Big( \frac{1}{\sqrt{2 \beta} } \Big)^2 $$
    Which corresponds to corrections up to first order of the energies of 
    $$ \pm \frac{\lambda \hbar}{2 m \omega_0} $$
\end{enumerate}

\end{document}


