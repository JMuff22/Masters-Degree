
\documentclass[12pt]{article}
%\usepackage[finnish]{babel}
\usepackage[T1]{fontenc}
\usepackage[utf8]{inputenc}
\usepackage{amssymb}
\usepackage{amsmath}
\usepackage{graphicx}
\usepackage{hyperref}
\newcommand{\pat}{\partial}
\newcommand{\be}{\begin{equation}}
\newcommand{\ee}{\end{equation}}
\newcommand{\bea}{\begin{eqnarray}}
\newcommand{\eea}{\end{eqnarray}}
\newcommand{\abf}{{\bf a}}
\newcommand{\Zcal}{{\cal Z}_{12}}
\newcommand{\zcal}{z_{12}}
\newcommand{\Acal}{{\cal A}}
\newcommand{\Fcal}{{\cal F}}
\newcommand{\Ucal}{{\cal U}}
\newcommand{\Vcal}{{\cal V}}
\newcommand{\Ocal}{{\cal O}}
\newcommand{\Rcal}{{\cal R}}
\newcommand{\Scal}{{\cal S}}
\newcommand{\Lcal}{{\cal L}}
\newcommand{\Hcal}{{\cal H}}
\newcommand{\hsf}{{\sf h}}
\newcommand{\half}{\frac{1}{2}}
\newcommand{\Xbar}{\bar{X}}
\newcommand{\xibar}{\bar{\xi }}
\newcommand{\barh}{\bar{h}}
\newcommand{\Ubar}{\bar{\cal U}}
\newcommand{\Vbar}{\bar{\cal V}}
\newcommand{\Fbar}{\bar{F}}
\newcommand{\zbar}{\bar{z}}
\newcommand{\wbar}{\bar{w}}
\newcommand{\zbarhat}{\hat{\bar{z}}}
\newcommand{\wbarhat}{\hat{\bar{w}}}
\newcommand{\wbartilde}{\tilde{\bar{w}}}
\newcommand{\barone}{\bar{1}}
\newcommand{\bartwo}{\bar{2}}
\newcommand{\nbyn}{N \times N}
\newcommand{\repres}{\leftrightarrow}
\newcommand{\Tr}{{\rm Tr}}
\newcommand{\tr}{{\rm tr}}
\newcommand{\ninfty}{N \rightarrow \infty}
\newcommand{\unitk}{{\bf 1}_k}
\newcommand{\unitm}{{\bf 1}}
\newcommand{\zerom}{{\bf 0}}
\newcommand{\unittwo}{{\bf 1}_2}
\newcommand{\holo}{{\cal U}}
%\newcommand{\bra}{\langle}
%\newcommand{\ket}{\rangle}
\newcommand{\muhat}{\hat{\mu}}
\newcommand{\nuhat}{\hat{\nu}}
\newcommand{\rhat}{\hat{r}}
\newcommand{\phat}{\hat{\phi}}
\newcommand{\that}{\hat{t}}
\newcommand{\shat}{\hat{s}}
\newcommand{\zhat}{\hat{z}}
\newcommand{\what}{\hat{w}}
\newcommand{\sgamma}{\sqrt{\gamma}}
\newcommand{\bfE}{{\bf E}}
\newcommand{\bfB}{{\bf B}}
\newcommand{\bfM}{{\bf M}}
\newcommand{\cl} {\cal l}
\newcommand{\ctilde}{\tilde{\chi}}
\newcommand{\ttilde}{\tilde{t}}
\newcommand{\ptilde}{\tilde{\phi}}
\newcommand{\utilde}{\tilde{u}}
\newcommand{\vtilde}{\tilde{v}}
\newcommand{\wtilde}{\tilde{w}}
\newcommand{\ztilde}{\tilde{z}}
\newcommand{\ket}[1]{\vert{#1}\rangle}
\newcommand{\bra}[1]{\langle{#1}\vert}


\hoffset 0.5cm
\voffset -0.4cm
\evensidemargin -0.2in
\oddsidemargin -0.2in
\topmargin -0.2in
\textwidth 6.3in
\textheight 8.4in

\begin{document}

\normalsize

\baselineskip 14pt

\begin{center}
{\Large {\bf Quantum Mechanics IIa 2021 \ \ \\ Solutions to Problem Set 2}} \\
Jake Muff
2/02/21
\end{center}
\section*{Question 1}
First order correction to third eigenenergy $E_3^{(0)}$ for a 1D box with infite walls at $x=0$ and $x=L$.
\begin{enumerate}
    \item $V= 10^{-3} E_1 x/L$. For a 1D box like this we know the energy eigenvalues 
    $$ E_n = n^2 \frac{ \pi^2 \hbar^2}{2mL^2} $$
    With eigenfunctions 
    $$ \psi (x) = \sqrt{\frac{2}{L}} \sin ( \frac{n \pi}{L} x) $$
    The first order correction to the nth eigenenergy 
    $$ E_n^{(1)} = \langle \psi_n^0 | V | \psi_n^0 \rangle $$
    $$ = \frac{2}{L} \int_0^L \sin^2 (\frac{n \pi x}{L} ) \cdot ( 10^{-3} \cdot E_1 \cdot \frac{x}{L} ) dx $$
    Evaluate at $n=3$ using Maple 
    $$ E_3^{(1)} = \frac{E_1}{2000} $$
    \item Using the same technique as before just with different $V$ we get 
    $$ E_3^{(1)} = \frac{(6 \pi^2 - 1) E_1}{18000 \pi^2} $$
    \item $$ E_3^{(1)} = \frac{ - 9\pi^2 E_1 (-1+ \cos(1)) }{250(36 \pi^2 -1)} $$
\end{enumerate}

\section*{Problem 2} 
For this problem we have a particle in a box with a bump between $-a/2$ and $a/2$
\begin{enumerate}
    \item $$ \Delta_2^{(1)} = \langle n^0 | V | n^0 \rangle = \int_{- \frac{a}{2}}^{\frac{a}{2}} \frac{2}{L} \sin ( \frac{2 \pi ( x + L/2) }{L} ) V_0 \sin ( \frac{2 \pi ( x + L/2) }{L} ) dx $$
    $$ = \frac{2}{L} V_0 \int_{- \frac{a}{2}}^{\frac{a}{2}} \sin^2 ( \frac{ 2 \pi x}{L} ) dx $$
    $$ = - \frac{V_0 ( \cos( \frac{\pi a}{L} ) \sin ( \frac{ \pi a}{L} ) L - \pi a ) }{L \pi} $$
    For the eigenfunction we use 
    $$ \ket{n} = \ket{n_0} + \lambda \sum_{k \neq n} \ket{k_0} \frac{ V_{kn} }{E_n^{(0)} - E_k^{(0)}}$$ 
    We want the last part 
    $$ \sum_{k \neq n} \ket{k_0} \frac{ V_{kn} }{E_n^{(0)} - E_k^{(0)}}$$ 
    %$$ \psi_2^1 = \sqrt{\frac{2}{L}} \sin(\frac{2 \pi x}{L} ) \frac{\int_{- \frac{a}{2}}^{\frac{a}{2}} \frac{2}{L} \sin( \frac{\pi x}{L}) V_0 \sin(\frac{2 \pi x}{L} )    }{E_2^{(0)} - E_1^{(0)}} dx $$
    %$$ = \sqrt
    Evaluating $V_{kn}$ for $n=2$ we get 
    $$ V_{k2} =  \int_{- \frac{a}{2}}^{\frac{a}{2}} \frac{2}{L} \sin(\frac{k \pi(x + \frac{L}{2})}{L} ) V_0 \sin( \frac{2 \pi (x + \frac{L}{2} ) }{L} ) dx $$
    $$ = 4 V_0 \cos (\frac{ \pi k}{2} ) \frac{k \sin (\frac{\pi a}{L} ) \cos(\frac{\pi a k}{2L} ) - 2 \cos (\frac{\pi a}{L} ) \sin(\frac{\pi a k }{2L} )}{\pi (k^2 -4)} $$
    The $\cos (\frac{ \pi k}{2} )$ shows that only the even terms are non-zero. So we can simplify by introducing a new variable $k=2s$ 
    $$ V_{s} = V_0 \cos (\frac{ \pi k}{2} )  \frac{2s \sin (\frac{\pi a}{L} ) \cos(\frac{\pi a s}{L} ) - 2 \cos (\frac{\pi a}{L} ) \sin(\frac{\pi a s }{L} )}{\pi (s^2 -1)} $$
    So we have 
    $$ \ket{n=2} = \ket{n_0 =2} + \lambda \sum_{k=2} \ket{2k} V_0 \cos (\frac{ \pi k}{2} )  \frac{2s \sin (\frac{\pi a}{L} ) \cos(\frac{\pi a s}{L} ) - 2 \cos (\frac{\pi a}{L} ) \sin(\frac{\pi a s }{L} )}{\pi (s^2 -1)} $$


    \item The $\frac{a}{L}$ is the dimensionless ratio that should be much smaller than 1, because, otherwise the approximation could be higher than the exact eigenfunction. 
    
    \item The first order correction to the nth state is in this case 
    $$ \int_{- \frac{a}{2}}^{\frac{a}{2}} \frac{2}{L} V_0 \sin^2 (\frac{n \pi x}{L} ) dx $$
    $$ = \frac{2}{L} V_0 \frac{1}{2} ( a - \frac{L \sin ( \frac{\pi a n }{L} ) }{\pi n} ) $$ 
    And thus even eigenstates are greater than odd because of the $sin$ dependency
\end{enumerate}


\section*{Problem 3} 
Periodically driven two state system. The Schrodinger equation is 
$$ i \hbar \frac{d}{dt} \ket{\psi (t) } = H(t) \ket{\psi (t)} $$
With time dependent hamiltonian 
$$ H(t) = H_0 + V(t) $$
\begin{enumerate}
    \item Lets use the interaction picture/representation to solve this
    $$ \ket{\psi (t) } _I = e^{i H_0 t / \hbar} \ket{\psi (t) } _S $$ 
    Where subscript I meaning interaction, subscript S meaning schrodinger picture/representation. Important to note that we have 
    $$ \ket{\psi (0) }_I = \ket{ \psi (0)}_S $$
    Because we are in the interaction picture we know that the wavefunction pbeys the equation of motion 
    $$ i \hbar \frac{ \partial}{\partial t} \ket{\psi (t)}_I = i \hbar \frac{\partial}{\partial t} ( e^{i H_0 t \hbar} \ket{\psi (t)}_S ) $$ 
    $$ = - H_0 e^{i H_0 t/ \hbar} \ket{ \psi (t) }_S + i \hbar \frac{\partial}{\partial t} e^{i H_0 t / \hbar} \ket{\psi (t)}_S $$
    Here we have $\frac{\partial}{\partial t} H_0 = 0$ due to time independence. 
    $$ = H e^{i H_0 t / \hbar} \ket{\psi (t)}_S - H_0 e^{i H_0 t / \hbar} \ket{ \psi (t)}_S $$
    $$ = e^{i H_0 t / \hbar} V (t) e^{-i H_0 t / \hbar} \ket{\psi (t) }_I $$
    Setting 
    $$ V_I (t) = e^{i H_0 t / \hbar} V (t) e^{-i H_0 t / \hbar} $$
    We have 
    $$ i \hbar \frac{\partial}{\partial t} \ket{\psi (t)}_I = V_I (t) \ket{\psi (t)}_I $$
    We expand out $\ket{\psi (t)}_I = \sum_n c_n (t) \ket{n} $ for $H_0 \ket{n} = E_n \ket{n}$ 
    $$ i \hbar \frac{ \partial}{\partial t} \sum_n c_n (t) \ket{n} = e^{i H_0 t / \hbar} V(t) e^{-i H_0 t / \hbar} \sum_n c_n (t) \ket{n} $$
    $$ i \hbar \sum_n \dot{c_n} (t) \ket{n} = \sum_n c_n (t) e^{i H_0 t / \hbar} V(t) e^{-i H_0 t / \hbar} \ket{n} $$
    We can now swap $e^{-i H_0 t / \hbar} \ket{n} \equiv e^{-i E_n t / \hbar} \ket{n} $. Say we interaction with a general state $\ket{m}$ we have 
    $$ \sum_n \dot{c_n} (t) \langle m | n \rangle = \sum_n c_n (t) \langle m | e^{i H_0 t / \hbar} V (t) e^{-i E_n t / \hbar} | n \rangle $$
    Where $\langle m | e^{i H_0 t / \hbar} \equiv \langle m | e^{i E_m t / \hbar}$
    So now we have 
    $$ i \hbar \dot{c_m} (t) = \sum_n \langle m | V(t) | n \rangle e^{i (E_m - E_n) t / \hbar} c_n (t) $$
    And introduce the variables given in the question to get the final answer; $\omega_{mn} = \frac{E_m - E_n}{2}, V_{mn} (t) = \langle m | V(t) | n \rangle$ 
    $$ i \hbar \dot{c_m} (t) = \sum_n e^{i \omega_{mn} t} V_{mn} (t) c_n (t) $$

    \item From the above we have two differential equations 
    $$ \dot{c_1} (t) = \frac{-i}{\hbar} V_mn (t) e^{i \omega_{mn} t} c_2 (t) $$
    $$ \dot{c_2} (t) = \frac{-i}{\hbar} V_mn (t) e^{i \omega_{mn} t} c_1 (t) $$
    At the zeroth order, due to the intitial condition we have
    $$ c_1^{(0)} = \delta_{mi}; c_2^{(0)} = \delta_{mi} $$
    This also means that we are 'preparing' an intitial state $\ket{i}$ at time $t = t_0$, as opposed to before where we have states $\ket{m}, \ket{n}$, hence the change in the subscripts later on.
    To first order we would then have 
    $$ \dot{c_2}^{(1)} (t) = \frac{ -i}{\hbar} V_{mn} e^{i \omega_{mn} t} \delta_{mi} $$
    $$ c_2^{(1)} (t) = \frac{-i}{\hbar} \int_{t_0}^t V_{mi} (t') e^{i \omega_{mi} t'} dt' $$
    The same applies to $c_1$, thus we have (in general).  
    $$ c_m (t) = \underbrace{\delta_{mi}}_{c_m^{(0)}}  - \underbrace{\frac{i}{\hbar} \int_{t_0}^t V_{mi} (t') e^{i \omega_{mi} t'} dt' }_{c_m^{(1)}} $$
    
    \item $$ c_m (t) = \delta_{mi} - \frac{i}{\hbar} \int_0^t dt' e^{i \omega_{mi}t'} V_{mi}(t') $$
    $$ c_m^{(0)} = \delta_{m1} $$
    $$ c_m (t) = \delta_{m1} - \frac{i}{\hbar} \int_0^t dt' e^{i \omega_{m1}t'} V_{m1}(t') $$
    $$ c_1 (t) = \delta_{11} - \frac{i}{\hbar} \int_0^t dt' e^{i \omega_{11}t'} \underbrace{V_{11}(t')}_{=0} $$
    $$ \delta_{11} -0 = 1 $$
    $$ c_2 (t) = \delta_{21} - \frac{i}{\hbar} \int_0^t dt' e^{i \omega_{21}t'} V_{21}(t') $$
    Via Wolfram ALpha ($ \omega_{21} = \omega \pm \omega_0 $)
    $$ c_2 (t) = \frac{-\lambda + \lambda e^{i(w + \omega_0)t }}{\omega + \omega_0} + \frac{\lambda - \lambda e^{-i(w - \omega_0)t }}{\omega - \omega_0}  $$
    $$ c_2 (t) =  \lambda \Big[ \frac{-1 + e^{i(w + \omega_0)t }}{\omega + \omega_0} + \frac{1-  e^{-i(w - \omega_0)t }}{\omega - \omega_0}  \Big] $$

    \item $$ H_0 = \begin{pmatrix}
        \hbar \omega_0 /2 & 0 \\ 0 & - \hbar \omega_0 /2 
    \end{pmatrix} $$
    $$ V(t) = \begin{pmatrix}
        0 & -2 \hbar \lambda \cos( \omega t) \\ -2 \hbar \lambda \cos( \omega t) & 0 
    \end{pmatrix} $$
    $$ H(t) = \begin{pmatrix}
        \hbar \omega_0 /2 & -2 \hbar \lambda \cos( \omega t) \\ -2 \hbar \lambda \cos( \omega t) & - \hbar \omega_0 /2  

    \end{pmatrix} $$
    Substituting energy eigenvalues and interaction potiential into equation (2) on the PS gets 
    $$ i \hbar \dot{c_1} (t) = \sum_n e^{i \omega_{12} t} V_{12} (t) c_2 (t) $$
    $$ i \hbar \dot{c_2} (t) = \sum_n e^{i \omega_{21} t} V_{21} (t) c_1 (t) $$
    So we get the two coupled differential equations 
    $$ \frac{ d c_1}{dt} = i \lambda ( e^{i ( \omega - \omega_0) t} +e^{-i ( \omega + \omega_0) t})c_2 (t) $$
    $$ \frac{ d c_2}{dt} = i \lambda ( e^{-i ( \omega - \omega_0) t} +e^{i ( \omega + \omega_0) t})c_1 (t) $$
    Setting $\delta \equiv \omega - \omega_0$ and applying the rotating wave approximation 
    $$ \frac{d c_1}{dt} = i \lambda e^{i \delta t} c_2 $$
    $$ \frac{d c_2}{dt} = i \lambda e^{-i \delta t} c_1 $$

    \item We have initial conditions $c_m^{(0)} (0) = \delta_{mi}$ meaning $c_1 (0) = 1, c_2 (0) = 0. $
    Now we can solve these because we have a linear second order differential equation with constant coefficients of which the general solutions are well knwon for the complex case, thereby getting 
    $$ c_1 (t) = \cos ( \frac{1}{2} \sqrt{\delta^2 + 4 \lambda^2} t) e^{i \delta t /2} - i \frac{\delta}{\sqrt{\delta^2 + 4 \lambda^2} } \sin ( \frac{1}{2} \sqrt{\delta^2 + 4 \lambda^2} t)e^{i \delta t /2} $$
    $$ = e^{i \delta t/2} [ \cos(\frac{1}{2} \Omega t) - i \frac{\delta}{\Omega} \sin ( \frac{1}{2} \Omega t) ] $$
    $$ c_2 (t) = e^{-i \delta t/2} \frac{2i \lambda}{\sqrt{\delta^2 + 4 \lambda^2} } \sin ( \frac{1}{2} \sqrt{ \delta + 4 \lambda^2} t ) $$
    $$ = e^{-i \delta t/2} \frac{ 2 i \lambda}{\Omega} \sin (\frac{1}{2} \Omega t) $$

    \item Not sure how to get that approximation as applying the RWA to (4) gives 
    $$ |c_2 (t) |^2 = | \lambda \Big( - \frac{1}{\omega + \omega_0} + \frac{1}{\omega - \omega_0}\Big) |^2 $$
    However, for the second part 
    $$ |c_2 (t) |^2 = | e^{-i \delta t/2} \frac{2i \lambda}{\Omega} \sin(\frac{\Omega t}{2}) |^2 $$
    $$ = e^{-i \delta t} (\frac{2 i \lambda}{\Omega} )^2 \sin^2 ( \frac{\Omega t}{2} ) $$
    Applying small angle approximation 
    $$ = -e^{-i \delta t} \frac{4 \lambda^2}{\Omega^2} \frac{\Omega^2 t^2}{4} $$
    $$ = -e^{-i \delta t} \lambda^2 t^2 $$
    $$ \approx \lambda^2 t^2 \approx (\lambda t)^2 $$
\end{enumerate}


\end{document}


