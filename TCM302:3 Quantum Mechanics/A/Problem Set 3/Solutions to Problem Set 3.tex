
\documentclass[12pt]{article}
%\usepackage[finnish]{babel}
\usepackage[T1]{fontenc}
\usepackage[utf8]{inputenc}
\usepackage{amssymb}
\usepackage{amsmath}
\usepackage{graphicx}
\usepackage{hyperref}
\newcommand{\pat}{\partial}
\newcommand{\be}{\begin{equation}}
\newcommand{\ee}{\end{equation}}
\newcommand{\bea}{\begin{eqnarray}}
\newcommand{\eea}{\end{eqnarray}}
\newcommand{\abf}{{\bf a}}
\newcommand{\Zcal}{{\cal Z}_{12}}
\newcommand{\zcal}{z_{12}}
\newcommand{\Acal}{{\cal A}}
\newcommand{\Fcal}{{\cal F}}
\newcommand{\Ucal}{{\cal U}}
\newcommand{\Vcal}{{\cal V}}
\newcommand{\Ocal}{{\cal O}}
\newcommand{\Rcal}{{\cal R}}
\newcommand{\Scal}{{\cal S}}
\newcommand{\Lcal}{{\cal L}}
\newcommand{\Hcal}{{\cal H}}
\newcommand{\hsf}{{\sf h}}
\newcommand{\half}{\frac{1}{2}}
\newcommand{\Xbar}{\bar{X}}
\newcommand{\xibar}{\bar{\xi }}
\newcommand{\barh}{\bar{h}}
\newcommand{\Ubar}{\bar{\cal U}}
\newcommand{\Vbar}{\bar{\cal V}}
\newcommand{\Fbar}{\bar{F}}
\newcommand{\zbar}{\bar{z}}
\newcommand{\wbar}{\bar{w}}
\newcommand{\zbarhat}{\hat{\bar{z}}}
\newcommand{\wbarhat}{\hat{\bar{w}}}
\newcommand{\wbartilde}{\tilde{\bar{w}}}
\newcommand{\barone}{\bar{1}}
\newcommand{\bartwo}{\bar{2}}
\newcommand{\nbyn}{N \times N}
\newcommand{\repres}{\leftrightarrow}
\newcommand{\Tr}{{\rm Tr}}
\newcommand{\tr}{{\rm tr}}
\newcommand{\ninfty}{N \rightarrow \infty}
\newcommand{\unitk}{{\bf 1}_k}
\newcommand{\unitm}{{\bf 1}}
\newcommand{\zerom}{{\bf 0}}
\newcommand{\unittwo}{{\bf 1}_2}
\newcommand{\holo}{{\cal U}}
%\newcommand{\bra}{\langle}
%\newcommand{\ket}{\rangle}
\newcommand{\muhat}{\hat{\mu}}
\newcommand{\nuhat}{\hat{\nu}}
\newcommand{\rhat}{\hat{r}}
\newcommand{\phat}{\hat{\phi}}
\newcommand{\that}{\hat{t}}
\newcommand{\shat}{\hat{s}}
\newcommand{\zhat}{\hat{z}}
\newcommand{\what}{\hat{w}}
\newcommand{\sgamma}{\sqrt{\gamma}}
\newcommand{\bfE}{{\bf E}}
\newcommand{\bfB}{{\bf B}}
\newcommand{\bfM}{{\bf M}}
\newcommand{\cl} {\cal l}
\newcommand{\ctilde}{\tilde{\chi}}
\newcommand{\ttilde}{\tilde{t}}
\newcommand{\ptilde}{\tilde{\phi}}
\newcommand{\utilde}{\tilde{u}}
\newcommand{\vtilde}{\tilde{v}}
\newcommand{\wtilde}{\tilde{w}}
\newcommand{\ztilde}{\tilde{z}}
\newcommand{\ket}[1]{\vert{#1}\rangle}
\newcommand{\bra}[1]{\langle{#1}\vert}


\hoffset 0.5cm
\voffset -0.4cm
\evensidemargin -0.2in
\oddsidemargin -0.2in
\topmargin -0.2in
\textwidth 6.3in
\textheight 8.4in

\begin{document}

\normalsize

\baselineskip 14pt

\begin{center}
{\Large {\bf Quantum Mechanics IIa 2021 \ \ \\ Solutions to Problem Set 3}} \\
Jake Muff
6/02/21
\end{center}
\section*{Problem 1}
Periodically Driven Harmonic Oscillator where $t < 0$ in the ground state and for $t >0$ we have perturbing potential 
$$ V(x,t) = F_0 x \cos (\omega t) $$
With Hamiltonian 
$$ H = \frac{p^2}{2m} + \frac{1}{2} m \omega_0^2 x^2 $$
In the interaction picture we have 
$$ \langle x \rangle = \langle \psi | x | \psi \rangle $$
$$ = \langle \psi | e^{i H_0 t} x  e^{-i H_0 t} | \psi \rangle $$
Where 
\begin{equation} \ket{\psi} = \sum_n c_n (t) \ket{n} \end{equation}
Starting at $t=0$ we have  $c_n^{(0)} (t) = \delta_{n0} $
$$ c_n^{(0)} = c_0^{(0)} = c_0 (t) -1 $$
$$ c_n^{(1)} (t) = \frac{-i}{\hbar} \int_{t_0}^t \langle n | V_I (t') | i \rangle dt' $$
$$ = \frac{-i}{\hbar} \int_{t_0}^t e^{i \omega_{ni} t'} V_{ni} (t') dt' $$
$$ = \frac{-i}{\hbar} \int_0^t V_{n0} (t') e^{in \omega_0 t'} $$
$$ = \frac{-i}{\hbar} \int_0^t e^{i(E_n - E_0)t' / \hbar} \langle n | F_0 x \cos (\omega t') | 0 \rangle dt' $$
Using the hint 
$$ \langle n' | x | n \rangle = \sqrt{\frac{\hbar}{2 m \omega}} ( \sqrt{n+1} \delta_{n', n+1} + \sqrt{n} \delta_{n', n-1} ) $$
So 
$$ \langle n | F_0 x \cos (\omega t' ) | 0 \rangle \equiv F_0 \langle n | x | 0 \rangle \cos (\omega t' ) $$
$$ = \sqrt{\frac{\hbar}{2m \omega}} ( \delta_{n,1} ) $$
Thus,
$$ c_n^{(1)} (t) = \frac{-i}{\hbar} \int_0^t e^{i \omega_0 t'} F_0 \cos( \omega t') \sqrt{\frac{\hbar}{2 m \omega_0}} \delta_{n1} dt' $$
$$ = \frac{-i}{\hbar} \sqrt{\frac{\hbar}{2m \omega_0}} F_0 \delta_{n1} \int_0^t e^{i \omega_0 t'} \cos (\omega t') dt' $$
$$ = \frac{-i}{\hbar} \sqrt{\frac{\hbar}{2m \omega_0}} F_0 \int_0^t e^{i \omega_o t'} \Big( \frac{e^{i \omega t'} + e^{-i \omega t'}}{2} \Big) dt' $$
$$ = \frac{-i}{\hbar} \sqrt{\frac{\hbar}{2m \omega_0}} F_0 \cdot \text{integral} $$
The integral is evaluated as 
$$ \int_0^t e^{i \omega_o t'} \Big( \frac{e^{i \omega t'} + e^{-i \omega t'}}{2} \Big) =  \Big[ \frac{ie^{-it'(\omega - \omega_0)}}{\omega - \omega_0} - \frac{i e^{it' (\omega + \omega_0)}}{\omega + \omega_0} \Big]^t_0 $$
$$ = -i \Big( \frac{1-e^{-it (\omega - \omega_0) }}{\omega - \omega_0} + \frac{e^{it(\omega + \omega_0) }-1}{\omega + \omega_0} \Big) $$
For $n>1$, $ c_n^{(1)} =0 $ clearly. 
Here I changed into the schrodinger picture, because it was easier to calculate (and understand what was going on). I still use the calculations above in the final answer. 

$$ \ket{\psi}_I = \sum_n c_n (t) \ket{n} $$
$$ = 1 \ket{0} + c_1 (t) \ket{1} $$
Therefore 
$$ \ket{\psi}_S = e^{-i H_0 t / \hbar} \ket{\psi}_I $$
For a simple harmonic Oscillator we have 
$$ H_0 \ket{0} = \frac{1}{2} \hbar \omega_0 \ket{0} $$
$$ H_0 \ket{1} = \frac{3}{2} \hbar \omega_0 \ket{1} $$
Thus 
$$ \ket{\psi}_S = e^{-i \omega_0 t /2 } \ket{0} + c_1 (t) e^{-3i \omega_0 t/2 } \ket{1} $$
$$ \langle x \rangle = _S\langle \psi | x | \psi \rangle_S $$
\begin{equation} = ( e^{i \omega_0 t/2 } \bra{0} + c_1^{\dagger} (t) e^{3i \omega_0 t/2} \bra{1} ) \cdot x \cdot ( e^{-i \omega_0 t / 2 } \ket{0} + c_1 (t) e^{-3i \omega_0 t /2 } \ket{1} ) \end{equation} 
$x$ here can be represented in ladder operator formalism with 
$$ x = \sqrt{\frac{\hbar}{2 m \omega_0 }} (a + a^{\dagger}) $$
Where 
$$ a = \sqrt{\frac{m \omega_0}{2}} (x + \frac{i}{m} \hat{p}) $$
$$ a^{\dagger} = \sqrt{\frac{m \omega_0}{2}} (x-\frac{i}{m} \hat{p} ) $$
Using this equation (2) can be split into two 
$$ c_1^{\dagger} e^{i \omega_0 t} \langle 1 | x | 0 \rangle = c_1^{\dagger} e^{i \omega_0 t} \sqrt{\frac{\hbar}{2m \omega_0} } $$
$$ c_1 e^{-i \omega_0 t} \langle 0 | x | 1 \rangle = c_1 e^{-i \omega_0 t} \sqrt{\frac{\hbar}{2 m \omega_0}} $$
So that 
$$ \langle x \rangle = \sqrt{\frac{\hbar}{2 m \omega_0 }} ( c_1 e^{-i \omega_0 t } + c_1^{\dagger} e^{i \omega_0 t} ) $$
We have from before: 
$$ c_1 (t) = \frac{-i}{\hbar} \sqrt{\frac{\hbar}{2 m \omega_0}} F_0 \Big( -i \Big( \frac{1-e^{-it(\omega - \omega_0)}}{\omega-\omega_0} + \frac{e^{it(\omega + \omega_0) } -1}{\omega+ \omega_0} \Big) \Big) $$
Substituting this in 
$$ \langle x \rangle = \frac{1}{\hbar} \frac{\hbar }{2 m \omega_0} F_0 \Big( e^{-i \omega_0 t} \Big( \frac{1-e^{i(\omega+ \omega_0)t }}{\omega+\omega_0} \Big) + e^{-i \omega_0 t} \Big( \frac{1-e^{i(\omega-\omega_0)t}}{\omega-\omega_0}  \Big)\Big) $$

$$ = \frac{1}{\hbar} \frac{\hbar }{2 m \omega_0} F_0 \Big( \Big( \frac{e^{-i \omega_0 t}-e^{i\omega t }}{\omega+\omega_0} \Big) + \Big( \frac{e^{-i \omega_0 t}-e^{-i\omega t}}{\omega-\omega_0}  \Big)\Big) $$

$$ = \frac{1}{\hbar}  \frac{\hbar}{2 m \omega_0 } F_0 \frac{\cos(\omega_0 t) - \cos (\omega t)}{\omega_0^2 - \omega^2} \Big((\omega - \omega_0 )+ (\omega + \omega_0) \Big) $$

$$ = \frac{1}{\hbar} \frac{\hbar}{2 m \omega_0} F_0 2 \omega_0 \Big( \frac{\cos(\omega_0 t) - \cos (\omega t) }{\omega_0^2 - \omega^2} \Big) $$

$$ = \frac{F_0}{m} \frac{\cos(\omega_0 t) - \cos(\omega t)}{\omega_0^2 - \omega^2} $$

Is this valid for $\omega = \omega_0 $? No. The equation becomes invalid when at resonance because as $\omega_0$ increases toward $\omega$ the soltion tends towards infinty, thus perturbation theory breaks down. 




\section*{Problem 2} 
Simple Harmonic Oscillator with 
$$ V(x,t) = Ax^2 e^{\frac{-t}{\tau}} $$
Probability that after $t >> \tau $ system transitions to a higher excited state. Transistion probability for $\ket{i} \rightarrow \ket{n}$ with $n \neq i$ is 

$$ P(i \rightarrow n) = | c_n^{(1)} (t) + c_n^{(2)} (t) + \ldots |^2 $$
For this we need 
$$ \langle n' | x^2 | n \rangle = \sqrt{\frac{\hbar}{2 m \omega_0}} \Big( \sqrt{n} \langle n' | x | n-1 \rangle + \sqrt{n-1} \langle n' | x | n+1 \rangle \Big) $$
$$ = \frac{\hbar}{2 m \omega_0} \Big( \sqrt{n(n-1) } \delta_{n-2, n'} + (2n +1 ) \delta_{nn'} + \sqrt{(n+1)(n+2) } \delta_{n+2, n'} \Big) $$
So 
$$ \langle n' | x^2 | 0 \rangle = \frac{\hbar}{2m\omega_0} ( \delta_{0n'} + \sqrt{2} \delta_{2n'} ) $$
Ignoring $\delta_{0n'}$ .
$$ c_n^{(0)} = \delta_{n0} $$
$$ c_n^{(1)} = \frac{-i}{\hbar} \int_0^t e^{i(E_n - E_0) t' / \hbar} \langle n' | Ax^2 e^{-t/ \tau} | 0 \rangle dt' $$

$$ = \frac{-i}{\hbar} A \int_0^t e^{i \omega_0 t'} e^{-t / \tau} \langle n' | x^2 | 0 \rangle dt' $$


$$ = \frac{-i}{\hbar} A \frac{\hbar}{2m \omega_0} \sqrt{2} \delta_{n2} \int_0^t e^{i \omega_0 t'} e^{-t / \tau} dt' $$

$$ = \frac{-i}{\hbar} A \frac{\hbar}{2 m \omega_0} \sqrt{2} \delta_{n2} \Big[ \frac{e^{i \omega_0 t'} - \frac{t'}{\tau} }{i \omega_0 - \frac{1}{\tau} } \Big]_0^t $$

$$ = \frac{-i}{\hbar} A \frac{\hbar}{2 m \omega_0} \sqrt{2} \delta_{n2} \Big[ \frac{e^{i \omega_0 t - \frac{t}{\tau} } }{i \omega_0 - \frac{1}{\tau} } - \frac{1}{i \omega_0 - \frac{1}{\tau}} \Big] $$

$$ = \frac{-i A}{m \omega_0} \frac{1}{\sqrt{2}} \delta_{n2} \Big[ \frac{e^{i \omega_0 t - \frac{t}{\tau}} }{i \omega_0 - \frac{1}{\tau} } \Big] $$

Trying for $c_0^{(1)}$, $c_{1}^{(1)} $ and  $ c_{2}^{(1)}$ we find that 
$$ c_0^{(1)} \neq 0 \rightarrow \delta_0 \neq 0 $$
$$ c_2^{(1)} \neq 0 \rightarrow \delta_{22} \neq 0 $$
So we can say $c_n^{(1)} = 0$ for $n \neq 0, 2$. This may be extended to $n \neq 0, 2n$ (aka even $n$) but I didn't test this. Therefore we have, 

$$ c_0^{(1)} = \frac{-i}{\hbar} \int_0^t \frac{\hbar}{2 m \omega_0} A e^{\frac{-t'}{\tau} } dt' $$
$$ = \frac{iA}{2m \omega_0} \tau (e^{- \frac{t}{\tau}} -1 ) $$
With $t >> \tau $ we have 
$$ c_0^{(1)} = - \frac{i A}{2 m \omega_0 }\tau $$

For $c_2^{(1)} $ we have 
$$ c_2^{(1)} = \frac{-i A}{m \omega_0}\frac{1}{\sqrt{2}} \frac{e^{i \omega_0 t - \frac{t}{\tau}} }{i \omega_0 - \frac{1}{\tau} } $$
The transistion to the $\ket{2}$ state is 
$$ | c_2^{(1)} |^2 = \frac{A^2 \tau^2 | e^{\frac{i \omega_0 t \tau - t}{\tau}} |^2}{2 m^2 \omega_0^2 ( \omega_0^2 \tau^2 +1) } $$
With $t >> \tau $ we have 
$$ | c_2^{(1)} |^2 = \frac{A^2 \tau^2}{2 m^2 \omega_0^2 ( \omega_0^2 \tau^2 +1) } $$

\section*{Problem 3} 
Hydrogen atom in ground state $(n l, m) = (1,0,0)$ with 
$$ \vec{E} = \begin{cases}
    0 & \quad t <0, \\
    \vec{E_0} e^{-t/ \tau} & \quad t>0.
    \end{cases} $$

We want to calculate the probability for atom to found at at $t >> \tau $ in 
$$ (n,l,m) = (2,1, \pm 1) $$
$$ (n,l,m) = (2,1, 0) $$
$$ (n,l,m) = (2,0,0) $$
The potential is 
$$ V = -eE_0 \hat{z} e^{-t / \tau} $$ 
So we have 
$$ c_n^{(1)} = \frac{-i}{\hbar} \int_0^t e^{i \omega_{ni} t'} \langle n | e E_0 \hat{z} e^{-t' / \tau} | i \rangle dt' $$
$$ = \frac{i}{\hbar} \int_0^t e^{i \omega_{ni} t'} \langle n | e E_0 \hat{z} | i \rangle e^{-t' / \tau} dt' $$

\begin{enumerate}
    \item For $(n,l,m) = (2,1,\pm 1) $ we have 
    $$ \langle 2,1, \pm 1 | \hat{z} | 1,0,0 \rangle =0 $$
    And because of the $ \Delta m = 0$ selection rule we have 
    $$ c_{2,1, \pm 1}^{(1)} = 0 $$ 
    
    \item For $(n,l,m) = (2,1,0) $ we have 
    $$ \langle 2,1,0 | \hat{z} | 1,0,0 \rangle $$
    We have a radial integral which we don't have to evaluate, which is 
    $$ \langle 210 | \hat{z} | 100 \rangle = \int_0^{\infty} dr r^3 R_{21}^* R_{10} \int_{-1}^1 d (\cos \theta ) \cos \theta Y_1^0 Y_0^0 $$
    $$ = \int_0^{\infty} R_{21}^* R_{10} r^3 dr $$
    Normalizing this gives 
    $$ = \frac{1}{\sqrt{3}} \int_0^{\infty} R_{21} R_{10} r^3 dr $$
    Let's call this $I_r$ for ease. So that 
    $$ c_{210}^{(1)} = I_r \cdot \frac{-i}{\hbar} \int_0^t e^{i \omega_{ni} t'} e E_0 e^{-t' / \tau } dt' $$
    $$ = \frac{i}{\hbar} e E_0 \tau \cdot I_r \cdot \frac{e^(i \omega_{ni} t - \frac{1}{\tau} ) \tau -1}{1- i \omega_{ni} \tau } $$ 
    With $ t >> \tau $ the probability transistion becomes 
    $$ | c_{210}^{(1)} |^2 = \frac{e^2 E_0^2 \tau^2}{\hbar^2} \cdot | I_r |^2 \cdot \frac{1}{1 + \omega_{ni}^2 \tau^2 } $$
    $$ \frac{e^2 E_0^2 \tau^2}{\hbar^2} | \int_0^{\infty} R_{21} R_{10} r^3 dr |^2 \cdot \frac{1}{1+ \omega_{ni}^2 \tau^2} $$

    \item Replacing the final state with 2s gives 
    $$ \langle 200 | \hat{z} | 100 \rangle = 0 $$
    Due to selection rules $\Delta l = + 1$ and the transistion probability is thus 0. 

    


\end{enumerate}

\section*{Problem 4} 
For $t <0$, $H=0$. For $t > 0$ 
$$ H = \Big( \frac{4 \Delta}{\hbar^2} \Big) \vec{S_1} \cdot \vec{S_2} $$
The state is initially in $\ket{ + - }$ for $ t \leq 0$. We need to find the probability of being in $\ket{++}, \ket{+-}, \ket{-+}, \ket{--}$ as a function of time. 

\begin{enumerate}
    \item Solving exactly. Firstly: 
    $$ \vec{S_1} \cdot \vec{S_2} = S_{12} S_{22} + \frac{1}{2} S_{1+} S_{2-} + \frac{1}{2} S_1 - S_{2+} $$
    So $H$ can be expanded using Spin Operator identities in a coupled basis, i.e 
    $$ S_+ \ket{-} = \hbar \ket{+} $$
    $$ S_- \ket{+} = \hbar \ket{-} $$
    $$ S_2 \ket{+} = \hbar{2} \ket{+} $$
    and we denote $\ket{1} = \ket{++}, \ket{2} = \ket{+-}, \ket{3} = \ket{-+}, \ket{4} = \ket{--} $. So the Hamiltonian can now be expanded as a matrix 
    $$ H = 4 \Delta \begin{pmatrix}
        \frac{1}{4} & 0 & 0 & 0 \\
        0 & - \frac{1}{4} & \frac{1}{2} & 0 \\
        0 & \frac{1}{2} & - \frac{1}{4} & 0 \\
        0 & 0 & 0 & \frac{1}{4} 
    \end{pmatrix}  $$
    This is worked from 11.230 of the QM 1 notes. The eigenvalues are 
    $$ E_{1} = \Delta, E_{0} = - 3 \Delta $$ 
    With $E_1$ for spin 1, $E_0$ for spin 0. From the Clebsch-Gordan coefficients we see that $\ket{++}, \ket{--}, \frac{1}{\sqrt{2}} ( \ket{+-}, + \ket{-+} )$ are all spin 1 and therefore have energy $\Delta$. $\frac{1}{\sqrt{2}} ( \ket{+-} - \ket{-+} ) $ is spin 0 and has energy $- 3 \Delta$. \\
    If we denote a new basis with 
    $$ \ket{1,0} = \frac{1}{\sqrt{2}} ( \ket{+-} + \ket{-+} ) $$
    $$ \ket{0,0} = \frac{1}{\sqrt{2}} ( \ket{+-} - \ket{-+} ) $$
    Then the initial state is 
    $$ \ket{+-} = \frac{1}{\sqrt{2}} ( \ket{1,0} + \ket{0,0} ) $$
    Solving exactly for $t >0$ we have 
    $$ U(t, t_0) = e^{\frac{-i}{\hbar} H (t-t_0) } $$
    $$ \ket{+-} = \frac{1}{\sqrt{2}} ( e^{-i \Delta t/ \hbar} \ket{1,0} + e^{i \Delta t 3 / \hbar} \ket{0,0} ) $$
    $$ = \frac{1}{\sqrt{2}} e^{-i \Delta t / \hbar} ( \ket{+-} + \ket{-+ } ) + \frac{1}{\sqrt{2}} e^{3i \Delta t / \hbar} ( \ket{+-} - \ket{-+} ) $$
    $$ = \frac{1}{2} \Big[ (e^{it \Delta / \hbar} + e^{3it \Delta / \hbar} ) \ket{+-} + ( e^{-i t \Delta / \hbar} + e^{3i \Delta t / \hbar} ) \ket{-+} \Big] $$
    The probability to find the system in state $\ket{i}$ is $| \langle i | +- \rangle |^2$, thus, clearly 
    $$ | \langle ++ | +- \rangle |^2 =0 $$
    $$ | \langle -- | +- \rangle |^2 = 0 $$ 
    The other two are 
    $$ | \langle +- | +- \rangle |^2 = \frac{1}{4} ( 2 + e^{4i \Delta t / \hbar} + e^{-4 i t \Delta / \hbar} ) $$
    $$ = \frac{1 + \cos( \frac{4 \Delta t}{\hbar} )}{2} $$
    $$ | \langle -+ | +- \rangle |^2 = \frac{1}{4} ( 2 - e^{4i \Delta t / \hbar} - e^{-4i \Delta t / \hbar} ) $$
    $$ = \frac{1 - \cos (\frac{4 \Delta t}{\hbar}  )}{2} $$

    \item Now solved using perturbation theory. 
    $$ c_n^{(0)} = \delta_{ni} $$
    $$ c_n^{(1)} = \frac{-i}{\hbar} \int_0^t \langle n | H | i \rangle dt' $$
    $$ = \frac{-it}{\hbar} \langle n | H | i \rangle $$
    Using the matrix form of $H$ from before we have 
    $$ \langle +- | H | +- \rangle =0 $$
    $$ \langle -+ | H | +- \rangle = 2 \Delta $$
    $$ \langle ++ | H | +- \rangle =0 $$
    $$ \langle -- | H | +- \rangle = 0 $$
    The bottom two agree with the exactly solved version. So we have 
    $$ c_{\ket{+-} }^{(0)} = 1 $$
    $$ c_{\ket{+-}}^{(1)} = 1 + \frac{it \Delta}{\hbar} $$
    $$ c_{\ket{-+}}^{(1)} = \frac{-2i \Delta t}{\hbar} $$
    $$ P_{\ket{+-} }^{(1)} = | c_{\ket{+-}}^{(1)} |^2 = 1 + \frac{t^2 \Delta^2}{\hbar^2} $$
    $$ P_{\ket{-+}}^{(1)} = | c_{\ket{-+} }^{(1)} |^2 = \frac{4 \Delta^2 t^2}{\hbar^2} $$
    Using the small angle approxmation for the exact solution we can then compare answers as 
    $$ = \frac{1 - \cos (\frac{4 \Delta t}{\hbar}  )}{2} \approx \frac{4 \Delta^2 t^2}{\hbar^2} $$
    Thus, for small times, first order perturbation theory gives the exact solution. The answer for the state $\ket{+-}$ does not match the exact solution, even when applying the same approxmation 
    $$ = \frac{1 + \cos( \frac{4 \Delta t}{\hbar} )}{2} \approx 1 - \frac{4 \Delta^2 t^2}{\hbar^2} $$
    However they do match in the sense that the probability must add up to equal 1
    $$ P_{\ket{+-}}^{(1)} + P_{\ket{-+}}^{(1)} = 1 $$
    Thus, 
    $$ P_{\ket{+-}}^{(1)} = 1- \frac{4 \Delta^2 t^2}{\hbar^2} $$
    I am not sure if I made a mistake somewhere, but it is interesting. 


\end{enumerate}




\end{document}


