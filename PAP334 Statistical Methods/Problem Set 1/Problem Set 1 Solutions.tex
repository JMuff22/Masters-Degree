
\documentclass[12pt]{article}
\usepackage[finnish]{babel}
\usepackage[T1]{fontenc}
\usepackage[utf8]{inputenc}
\usepackage{delarray,amsmath,bbm,epsfig,slashed}
\newcommand{\pat}{\partial}
\newcommand{\be}{\begin{equation}}
\newcommand{\ee}{\end{equation}}
\newcommand{\bea}{\begin{eqnarray}}
\newcommand{\eea}{\end{eqnarray}}
\newcommand{\abf}{{\bf a}}
\newcommand{\Zmath}{\mathbf{Z}}
\newcommand{\Zcal}{{\cal Z}_{12}}
\newcommand{\zcal}{z_{12}}
\newcommand{\Acal}{{\cal A}}
\newcommand{\Fcal}{{\cal F}}
\newcommand{\Ucal}{{\cal U}}
\newcommand{\Vcal}{{\cal V}}
\newcommand{\Ocal}{{\cal O}}
\newcommand{\Rcal}{{\cal R}}
\newcommand{\Scal}{{\cal S}}
\newcommand{\Lcal}{{\cal L}}
\newcommand{\Hcal}{{\cal H}}
\newcommand{\hsf}{{\sf h}}
\newcommand{\half}{\frac{1}{2}}
\newcommand{\Xbar}{\bar{X}}
\newcommand{\xibar}{\bar{\xi }}
\newcommand{\barh}{\bar{h}}
\newcommand{\Ubar}{\bar{\cal U}}
\newcommand{\Vbar}{\bar{\cal V}}
\newcommand{\Fbar}{\bar{F}}
\newcommand{\zbar}{\bar{z}}
\newcommand{\wbar}{\bar{w}}
\newcommand{\zbarhat}{\hat{\bar{z}}}
\newcommand{\wbarhat}{\hat{\bar{w}}}
\newcommand{\wbartilde}{\tilde{\bar{w}}}
\newcommand{\barone}{\bar{1}}
\newcommand{\bartwo}{\bar{2}}
\newcommand{\nbyn}{N \times N}
\newcommand{\repres}{\leftrightarrow}
\newcommand{\Tr}{{\rm Tr}}
\newcommand{\tr}{{\rm tr}}
\newcommand{\ninfty}{N \rightarrow \infty}
\newcommand{\unitk}{{\bf 1}_k}
\newcommand{\unitm}{{\bf 1}}
\newcommand{\zerom}{{\bf 0}}
\newcommand{\unittwo}{{\bf 1}_2}
\newcommand{\holo}{{\cal U}}
%\newcommand{\bra}{\langle}
%\newcommand{\ket}{\rangle}
\newcommand{\muhat}{\hat{\mu}}
\newcommand{\nuhat}{\hat{\nu}}
\newcommand{\rhat}{\hat{r}}
\newcommand{\phat}{\hat{\phi}}
\newcommand{\that}{\hat{t}}
\newcommand{\shat}{\hat{s}}
\newcommand{\zhat}{\hat{z}}
\newcommand{\what}{\hat{w}}
\newcommand{\sgamma}{\sqrt{\gamma}}
\newcommand{\bfE}{{\bf E}}
\newcommand{\bfB}{{\bf B}}
\newcommand{\bfM}{{\bf M}}
\newcommand{\cl} {\cal l}
\newcommand{\ctilde}{\tilde{\chi}}
\newcommand{\ttilde}{\tilde{t}}
\newcommand{\ptilde}{\tilde{\phi}}
\newcommand{\utilde}{\tilde{u}}
\newcommand{\vtilde}{\tilde{v}}
\newcommand{\wtilde}{\tilde{w}}
\newcommand{\ztilde}{\tilde{z}}

% David Weir's macros


\newcommand{\nn}{\nonumber}
\newcommand{\com}[2]{\left[{#1},{#2}\right]}
\newcommand{\mrm}[1] {{\mathrm{#1}}}
\newcommand{\mbf}[1] {{\mathbf{#1}}}
\newcommand{\ave}[1]{\left\langle{#1}\right\rangle}
\newcommand{\halft}{{\textstyle \frac{1}{2}}}
\newcommand{\ie}{{\it i.e.\ }}
\newcommand{\eg}{{\it e.g.\ }}
\newcommand{\cf}{{\it cf.\ }}
\newcommand{\etal}{{\it et al.}}
\newcommand{\ket}[1]{\vert{#1}\rangle}
\newcommand{\bra}[1]{\langle{#1}\vert}
\newcommand{\bs}[1]{\boldsymbol{#1}}
\newcommand{\xv}{{\bs{x}}}
\newcommand{\yv}{{\bs{y}}}
\newcommand{\pv}{{\bs{p}}}
\newcommand{\kv}{{\bs{k}}}
\newcommand{\qv}{{\bs{q}}}
\newcommand{\bv}{{\bs{b}}}
\newcommand{\ev}{{\bs{e}}}
\newcommand{\gv}{\bs{\gamma}}
\newcommand{\lv}{{\bs{\ell}}}
\newcommand{\nabv}{{\bs{\nabla}}}
\newcommand{\sigv}{{\bs{\sigma}}}
\newcommand{\notvec}{\bs{0}_\perp}
\newcommand{\inv}[1]{\frac{1}{#1}}
%\newcommand{\xv}{{\bs{x}}}
%\newcommand{\yv}{{\bs{y}}}
\newcommand{\Av}{\bs{A}}
%\newcommand{\lv}{{\bs{\ell}}}

%\newcommand\bsigma{\vec{\sigma}}
\hoffset 0.5cm
\voffset -0.4cm
\evensidemargin -0.2in
\oddsidemargin -0.2in
\topmargin -0.2in
\textwidth 6.3in
\textheight 8.4in

\begin{document}

\normalsize

\baselineskip 14pt

\begin{center}
{\Large {\bf Statistical Methods \ \ Fall 2020 \ \  Answers to Problem Set 1}}\\
{\large { Jake Muff}}\\
{Student number: 015361763}\\
{13/09/2020}
\end{center}



\begin{enumerate}

\item Exercise 1
%Question 1 Answer here
\begin{enumerate}
    \item Proof of $P(A \cup B)$
    \\ 
    For this we mainly use Kolmogorov's third axiom as well as the nature of associativity:
    \\
    Suppose there is 
    $$ A \cup (B \cap A^c) $$
    Where $A^c$ is the complementary to A such that $P(A^c) = 1-P(A) $. Then we can express $ A \cup B$ as the union of two disjoint sets:
    $$ A \cup (B \cap A^c) = (A \cup B) \cap (A \cup A^c) = A\cup B $$
    Which can also be written in terms of B as the union of two disjoint sets:
    $$ B = B \cap (A \cup A^c) = (B \cap A) \cup (B\cap A^c) $$
    Using Axiom 3 we can write:
    \begin{equation} \label{1}
        P(A\cup B) = P(A \cup (B \cap A^c)) = P(A) + P(B \cap A^c) 
    \end{equation}
    
    And for the $B$ part
    $$ P(B) = P(B \cap A) + P(B\cap A^c) $$ 
    Rearranged gives:
    $$ P(B \cap A^c) = P(B) - P(B \cap A)  $$
    Subsituting this into (1) gives 
    $$ P(A \cup B) = P(A \cap (B \cap A^c)) = P(A) + P(B \cap A^c) $$
    $$ = P(A) + P(B) - P(B \cap A) $$


    \item Probability that he/she is a MacOS user given that their computer has been infected with a virus $P(MacOS|Virus)$, which, using Bayes Theorem gives
    $$ P(MacOS|Virus) = \frac{P(Virus|MacOS) P(MacOS)}{P(V)} $$
    Now we know $P(Virus | MacOS) = 3\%$ and the probability of having MacOS $P(MacOS) = 14\%$ and from the 2nd axiom we can figure out $P(Virus)$ from
    $$ P(Virus) = \sum_i P(OS_i \cup Virus) + P(OS_i \cup No Virus) = 1 $$ 
    Where $i = 1,2, \ldots$ is each OS. Probability virus is then $P(Virus) = 6.96\%$. So we have 
    $$ P(MacOS|Virus) = \frac{3\% \times 12\%}{6.96\%} = 5.172\% $$

    \item Probability that he/she is not a Windows user given that they have the virus will be 
    $$ P(Not Windows | Virus) = \frac{P(Virus | Not windows)P(Not windows)}{P(Virus)} $$
    With 
    $$ P(Virus | Not windows) = P(Virus | MacOS) + P(Virus | Linux) + P(Virus | Chrome) = 9\%$$
    $$ P(Not windows) = P(MacOS) + P(Linux) + P(Chrome) = 1- P(Windows) = 20\%$$
    So we have 
    $$ P(Not Windows | Virus) = \frac{9\% \times 20\%}{6.96\%} = 25.86\% $$
    
\end{enumerate}

\item Exercise 2

For this question we have a uniform distribution:
$$   f(x;\alpha, \beta) = 
\begin{cases}
    \frac{1}{\beta-\alpha}&  \alpha \leq x \leq \beta\\
    0&\text{otherwise}\\
    \end{cases} $$
$$ E[x] = \int_\alpha^\beta \frac{x}{(\beta-\alpha)} dx $$
$$ V[x] = \int_\alpha^\beta \frac{[x-\frac{1}{2}(\alpha+\beta)^2  ]}{(\beta-\alpha)} dx$$
In our case we can take the halfway point between each strip and integrate over. In that case $\beta = \frac{d}{2}$ and $\alpha = \frac{-d}{2}$. We therefore have
$$ V[x] = \int_\frac{-d}{2}^\frac{d}{2} \frac{[x-\frac{1}{2}(\frac{-d}{2}+\frac{d}{2})^2  ]}{(\frac{d}{2}-\frac{-d}{2})} dx$$
$$ = \int_\frac{-d}{2}^\frac{d}{2} \frac{x^2}{d} dx $$
$$ = \frac{d^2}{12} = \frac{d}{\sqrt{12}} $$





\end{enumerate}


\end{document}

