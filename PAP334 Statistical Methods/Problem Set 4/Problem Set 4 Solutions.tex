
\documentclass[12pt]{article}
\usepackage[finnish]{babel}
\usepackage[T1]{fontenc}
\usepackage[utf8]{inputenc}
\usepackage{listings}
\usepackage{graphicx}
\usepackage{caption}
\usepackage{subcaption}
\usepackage{delarray,amsmath,bbm,epsfig,slashed}
\newcommand{\pat}{\partial}
\newcommand{\be}{\begin{equation}}
\newcommand{\ee}{\end{equation}}
\newcommand{\bea}{\begin{eqnarray}}
\newcommand{\eea}{\end{eqnarray}}
\newcommand{\abf}{{\bf a}}
\newcommand{\Zmath}{\mathbf{Z}}
\newcommand{\Zcal}{{\cal Z}_{12}}
\newcommand{\zcal}{z_{12}}
\newcommand{\Acal}{{\cal A}}
\newcommand{\Fcal}{{\cal F}}
\newcommand{\Ucal}{{\cal U}}
\newcommand{\Vcal}{{\cal V}}
\newcommand{\Ocal}{{\cal O}}
\newcommand{\Rcal}{{\cal R}}
\newcommand{\Scal}{{\cal S}}
\newcommand{\Lcal}{{\cal L}}
\newcommand{\Hcal}{{\cal H}}
\newcommand{\hsf}{{\sf h}}
\newcommand{\half}{\frac{1}{2}}
\newcommand{\Xbar}{\bar{X}}
\newcommand{\xibar}{\bar{\xi }}
\newcommand{\barh}{\bar{h}}
\newcommand{\Ubar}{\bar{\cal U}}
\newcommand{\Vbar}{\bar{\cal V}}
\newcommand{\Fbar}{\bar{F}}
\newcommand{\zbar}{\bar{z}}
\newcommand{\wbar}{\bar{w}}
\newcommand{\zbarhat}{\hat{\bar{z}}}
\newcommand{\wbarhat}{\hat{\bar{w}}}
\newcommand{\wbartilde}{\tilde{\bar{w}}}
\newcommand{\barone}{\bar{1}}
\newcommand{\bartwo}{\bar{2}}
\newcommand{\nbyn}{N \times N}
\newcommand{\repres}{\leftrightarrow}
\newcommand{\Tr}{{\rm Tr}}
\newcommand{\tr}{{\rm tr}}
\newcommand{\ninfty}{N \rightarrow \infty}
\newcommand{\unitk}{{\bf 1}_k}
\newcommand{\unitm}{{\bf 1}}
\newcommand{\zerom}{{\bf 0}}
\newcommand{\unittwo}{{\bf 1}_2}
\newcommand{\holo}{{\cal U}}
%\newcommand{\bra}{\langle}
%\newcommand{\ket}{\rangle}
\newcommand{\muhat}{\hat{\mu}}
\newcommand{\nuhat}{\hat{\nu}}
\newcommand{\rhat}{\hat{r}}
\newcommand{\phat}{\hat{\phi}}
\newcommand{\that}{\hat{t}}
\newcommand{\shat}{\hat{s}}
\newcommand{\zhat}{\hat{z}}
\newcommand{\what}{\hat{w}}
\newcommand{\sgamma}{\sqrt{\gamma}}
\newcommand{\bfE}{{\bf E}}
\newcommand{\bfB}{{\bf B}}
\newcommand{\bfM}{{\bf M}}
\newcommand{\cl} {\cal l}
\newcommand{\ctilde}{\tilde{\chi}}
\newcommand{\ttilde}{\tilde{t}}
\newcommand{\ptilde}{\tilde{\phi}}
\newcommand{\utilde}{\tilde{u}}
\newcommand{\vtilde}{\tilde{v}}
\newcommand{\wtilde}{\tilde{w}}
\newcommand{\ztilde}{\tilde{z}}

% David Weir's macros


\newcommand{\nn}{\nonumber}
\newcommand{\com}[2]{\left[{#1},{#2}\right]}
\newcommand{\mrm}[1] {{\mathrm{#1}}}
\newcommand{\mbf}[1] {{\mathbf{#1}}}
\newcommand{\ave}[1]{\left\langle{#1}\right\rangle}
\newcommand{\halft}{{\textstyle \frac{1}{2}}}
\newcommand{\ie}{{\it i.e.\ }}
\newcommand{\eg}{{\it e.g.\ }}
\newcommand{\cf}{{\it cf.\ }}
\newcommand{\etal}{{\it et al.}}
\newcommand{\ket}[1]{\vert{#1}\rangle}
\newcommand{\bra}[1]{\langle{#1}\vert}
\newcommand{\bs}[1]{\boldsymbol{#1}}
\newcommand{\xv}{{\bs{x}}}
\newcommand{\yv}{{\bs{y}}}
\newcommand{\pv}{{\bs{p}}}
\newcommand{\kv}{{\bs{k}}}
\newcommand{\qv}{{\bs{q}}}
\newcommand{\bv}{{\bs{b}}}
\newcommand{\ev}{{\bs{e}}}
\newcommand{\gv}{\bs{\gamma}}
\newcommand{\lv}{{\bs{\ell}}}
\newcommand{\nabv}{{\bs{\nabla}}}
\newcommand{\sigv}{{\bs{\sigma}}}
\newcommand{\notvec}{\bs{0}_\perp}
\newcommand{\inv}[1]{\frac{1}{#1}}
%\newcommand{\xv}{{\bs{x}}}
%\newcommand{\yv}{{\bs{y}}}
\newcommand{\Av}{\bs{A}}
%\newcommand{\lv}{{\bs{\ell}}}

%\newcommand\bsigma{\vec{\sigma}}
\hoffset 0.5cm
\voffset -0.4cm
\evensidemargin -0.2in
\oddsidemargin -0.2in
\topmargin -0.2in
\textwidth 6.3in
\textheight 8.4in

\begin{document}

\normalsize

\baselineskip 14pt

\begin{center}
{\Large {\bf Statistical Methods \ \ Fall 2020 \ \  Answers to Problem Set 4}}\\
{\large { Jake Muff}}\\
{Student number: 015361763}\\
{07/10/2020}
\end{center}

\section{Question 2: Test Statistic}
For this question we have two gaussian distributions. For the pions $\mu = 0$ as it is centered at 0 and $\sigma = 1$. For the kaons $mu = 3.0$ and $\sigma = 1$. \\
So for this question the hypothesis $H_0$ is for the pions with a probability distribution of 
$$ g(w | H_0) = \frac{1}{\sqrt{2 \pi}} e^{-\frac{1}{2}(w-0)^2} $$
And the hypothesis $H_1$ with a pdf 
$$ g(w | H_0) = \frac{1}{\sqrt{2 \pi}} e^{-\frac{1}{2}(w-3)^2} $$
So that 
$$ H_0 = \pi \ , \ H_1 = K $$
\begin{enumerate}
    \item What is the kaon selection efficiency when requiring $w > 2.0$? \\
    $$ \epsilon_K = \int_{- \infty}^2 g(w | K) dw $$
    $$ = \int_{- \infty}^2 \frac{1}{\sqrt{2 \pi}} e^{- \frac{1}{2}(w-3)^2} $$
    $$ = 0.158655 $$
    Or, alternatively calculate 
    $$ \alpha = \int_2^{\infty} \frac{1}{\sqrt{2 \pi}} e^{-\frac{1}{2}(w-3)^2} dw $$
    $$ = 0.841345 $$
    Where 
    $$ \epsilon_K = 1 - \alpha $$
    So 
    $$ \epsilon_K = 0.158655 $$

    \item What is the probability that a pion will be accepted as a kaon when requiring $w > 2.0$? 
    $$ \int_2^{\infty} \frac{1}{\sqrt{2 \pi}} e^{-\frac{1}{2} (w)^2} $$
    $$ = 0.0227501 $$

    \item Suppose a sample of particles consists of 95 \% pions and 5 \% kaons. What is the purity of the kaon sample selected by $w > 2.0$?
    \\
    Purity from GC can be defined as 
    $$ p_K = \frac{\int_{-\infty}^w a_K \cdot g(w | K) dw}{\int_{-\infty}^w (a_K g(w|K) + (1-a_K)g(w|\pi))dw} $$
    With $a_K = 5 \%$. This gives us 
    $$ p_K = 0.00847 $$
    $$ p_K = 0.847 \%$$

    \item Design a kaon selection for the same sample that gives real kaons in at least 4 cases out of 5. What will be the requirement on w?
    \\
    Question is basically asking what should the cut value be at a minimum for a sample of kaons with a purity of 80\%. To answer this use the equation from GC 
    $$ \frac{g(w|K)}{g(w|\pi)} > 0.8 $$
    $$ \frac{\frac{1}{\sqrt{2 \pi}} e^{-\frac{1}{2}(w-3)^2}}{\frac{1}{\sqrt{ 2 \pi}} e^{-\frac{1}{2} w^2}} > 0.8 $$
    Which reduces down to 
    $$ e^{\frac{6w -9}{2}} > 0.8 $$
    So 
    $$ w > 1.4256 $$
    
  
\end{enumerate}

\end{document}

